\section{Concluding Remarks}
\label{sec:concluding-remarks}

In this dissertation, a new group of hybrid methods called the $\Omega$-methods
were
proposed. The $\Omega$-methods are built on the foundational work of CADIS
and FW-CADIS to generate angle-informed variance reduction parameters. The two
new methods proposed were CADIS-$\Omega$ and FW-CADIS-$\Omega$. Both methods use
the $\Omega$-flux, a form of the adjoint scalar flux calculated by weighting the
adjoint angular flux with the forward angular flux, to generate source biasing and weight window
values. By using the forward angular flux
normalization, the importance map generated for the
$\Omega$-methods is adjusted to include the directionality of
the forward and the adjoint particles without explicitly including angle
in the source biasing or weight window values.

The $\Omega$-methods were implemented in two software packages developed at Oak
Ridge National Laboratory: Exnihilo and ADVANTG. The functionality to generate
the $\Omega$-fluxes were implemented in Exnihilo, which contains the
deterministic transport solver Denovo. The infrastructure to generate variance
reduction parameters consistent with CADIS and FW-CADIS was implemented in
ADVANTG. The development of these methods now allows for any user to use the
$\Omega$-methods, should they have access to the software.

In addition to the $\Omega$-methods method proposal and
implementation, CADIS-$\Omega$ has been
characterized on a wide variety of problems with flux anisotropies. The problems
were designed to understand the method's limitations and in what parameter space
the method can and should be used. To more fully understand the method's' behavior and how
flux anisotropy affected its ability to perform, a number of anisotropy metrics
were proposed. These metrics were then used
to investigate if performance improvement could be correlated with
anisotropy in any way.

The anisotropy metrics did not show significant trends with the FOM or the
solution relative error, but their distributions did help reveal more about the
distribution of anisotropy in the problems. In particular, it was easily
observable how the distribution of anisotropy changed between energy groups for
a particular problem. Future use of these metrics may also
aid us in more fully understanding other hybrid methods' performance.

CADIS-$\Omega$ is a promising hybrid method. If used with a well-suited problem,
it has the potential to improve the FOM over traditional methods by an order of
magnitude. This offers significant time and energy savings. However, the
$\Omega$-methods are not without their drawbacks. If used in a poorly-suited
problem they can take substantially more time to transport particles in
Monte Carlo. The $\Omega$-methods' characterization and performance study
presented in this dissertation have contributed a broader understanding of these types of
hybrid methods, and have created ample pathways forward for future hybrid
methods analysis.


