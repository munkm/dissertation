\section{Suggested Future Work}
\label{sec:futurewrk}

While this dissertation covered the characterization of CADIS-$\Omega$ over a
fairly broad spectrum of anisotropy-containing problems, there are a number of
fruitful pathways by which the method could be improved or characterization
expanded. Broadly, these fall into three categories: improvements to the
software implementation and algorithmic design, expansion of the
characterization space, and application to larger, real-life problems.
The next few
subsections addresses each one of these categories individually.

\subsection{Software Improvement}
\label{subsec:softwareimp}

This subsection addresses the improvements that could be
made to the software and analysis methods to enhance understanding of the omega
methods. A discussion on how improving software performance aids in
future work will start this section.
An explanation of how extending the breadth of analysis
helps to understand the $\Omega$-methods further will finish the section.

To calculate the $\Omega$ flux, a rotation of either the adjoint or forward
flux matrix is
required to ensure that the directional variable $\Omega$ is consistent between
the forward and adjoint,
or that $\Omega_{adjoint} = \Omega_{forward}$. Because quadrature sets
are not always straightforward to interpolate, a rotationally symmetric
quadrature set is currently required for computing the $\Omega$-fluxes in order
to perform this rotation.
Should a method
be developed that does have interpolatable quadrature points, it would be a good
candidate to calculate $\Omega$-fluxes for solutions that are not
rotationally symmetric.

The anisotropy metrics described in Section \ref{sec:anisotropy_quant} at this
point have not shown significant trends with either the relative error or figure
of merit improvement metrics (I$_{RE}$ and I$_{FOM}$). To filter out values of
each metric to regions more important to the problem solution, two filtering
algorithms were proposed: one that only uses values of metrics from cells with
contributon fluxes above the mean contributon flux, and the other that uses
values from cells that have a flux above the median contributon flux. Using
these filtering algorithms did show interesting features in the anisotropy
metric distributions as well as shifts in I$_{RE}$ and I$_{FOM}$. However,
trends were not apparent for the majority of the metrics.
A useful modification
to the filtering algorithm would be to select certain percentages of high-valued
contributon flux locations. For example, perhaps selecting out the cell locations
containing the top 10\% of contributon fluxes would reveal a trend in the
improvement metrics.
It is possible that too many values are being
selected from the entire problem even with the existing filters, so an even
stricter filtering algorithm may help.

To filter the anisotropy metrics, the contributon flux distribution was chosen
as the filter base. This is an intuitively good choice
because it will use values near both
the forward- and adjoint- sources, and also the values between them where
particles are most likely to flow. Further, the contributon flux is something
that is method agnostic. That is, it can be used as a filtering
algorithm for non-$\Omega$ methods and it will still reveal problem information.
However, an argument could also be made to
use the omega flux distribution as a filtering base, as that is the method in
which we are interested. Modifying the filtering algorithms to use the
$\Omega$-flux distribution may provide trends in the method
improvement metrics that are not apparent using the contributon flux.

The $\Omega$-methods, as currently implemented in both Exnihilo and ADVANTG, are
entirely serial. That is, there is no parallelization in any part of the
$\Omega$-flux calculation, or supporting code to that effect. In the results
presented in Chapter \ref{ch:charprobs}, T$_{hybrid}$ was calculated to remove
parallelization effects so that CADIS and CADIS-$\Omega$ were comparable. While
the results were adjusted accordingly, this is not the best implementation for
production software or more difficult use cases. As mentioned previously, the
ADVANTG software is entirely serial, so parallelization is not required for
VR parameter generation. However, Exnihilo/Denovo is parallelized. The
parallelization of the $\Omega$-flux calculation in this code
would significantly improve its
usability. Parallelization would reduce the actual time to calculate the
$\Omega$-fluxes and anisotropy metrics.

Another algorithmic improvement to the $\Omega$-methods is to reduce the
memory requirements for both the computation of the $\Omega$-fluxes and the
anisotropy metrics. Much of this could be accomplished with parallelization.
However, even the serial version of the $\Omega$-methods could be adjusted to
read in the angular flux data in ``chunks'' so as to not read in datasets larger
than the memory available on the system. As a first order approach, the angular
fluxes could be read
in serially by energy group. Depending on the energy group structure, this has
the potential to
reduce the memory load at a particular time by 20x or 200x. At present, the
$\Omega$-methods are limited by memory requirements. Without a large computing
cluster, there is no feasible way to calculate the $\Omega$-fluxes for a problem
of reasonable complexity.

Another alternative modification to the $\Omega$-methods is to bypass writing
the angular flux matrices entirely. This would reduce the I/O requirements for
the method, and also not demand as much disk space. However, this is a
non-trivial task, as the forward and angular fluxes for a cell must both be in
memory to compute the $\Omega$-method for that cell. To store the complete angular
flux matrices in memory will present the same memory limitations that the $\Omega$-methods
currently face, so some algorithmic challenges exist should this be a path of
future work.

The $\Omega$-methods are currently implemented on a localized
development version of both Exnihilo and ADVANTG. If a larger audience wishes to
use or access them, they would require support beyond that of a standard
software release. Depending on the continued characterization of the
$\Omega$-methods, integrating this software into future releases of Exnihilo and
ADVANTG may be useful.

Each of the areas proposed in the previous paragraphs are areas in which the
$\Omega$-methods can be improved upon or areas that may improve our understanding of
the $\Omega$-methods' behavior. Expanding the filtering algorithm
for the anisotropy metrics may
also help us to understand more broadly how anisotropy is distributed in
different problems. Expanding our understanding of the $\Omega$-methods'
strengths and deficiencies can also improve future hybrid methods.

\subsection{Characterization Problem Extension}
\label{subsec:extendcharprobs}

Broadening the scope
of the characterization problem study
is another fruitful avenue for exploration. In this
vein, there exits a two-pronged approach: first extending the types of problems
(more diverse materials, less air in problems, more diverse geometries) will
enhance knowledge of the methods. Next, extending the scope of the parametric
studies will help to inform how resilient the $\Omega$-methods might be to
changes in the solutions space that indirectly impacts angle.
In this realm, the deterministic calculation specifics, like
quadrature type will be addressed.

The characterization problems studied covered a broad range of
anisotropy-inducing physics. The geometries chosen were fairly
simple, with very few
materials. The majority of the problems used air in some portion of their
geometry to have streaming-induced anisotropy of the flux. Depending on their
geometries, this caused sampling issues and slowdown of the CADIS-$\Omega$ method.
For example, the problem variants of the
steel beam embedded in concrete geometry illustrated the
CADIS-$\Omega$ method's susceptibility to air. In the air-filled beam
variant of the problem, the $\Omega$-method had the lowest improvement margin
when compared to CADIS of the three matierial variants. A
beneficial extension of the characterization problem study would be to replace
the
air in this geometry with a high atomic mass
material that maintains scattering anisotropy but includes more
sampling interaction points. Using problems with greater material diversity
and more problems
with preferential flowpaths (that are not air),
would be an interesting extension
to the characterization problem materials.

While the characterization problems are fairly simple
geometrically, it may be advantageous to
investigate simpler problem geometries with even less
geometric complexity. In
comparing the single- and multiple-turn labyrinths, we observed that with too
little anisotropy in the problem, the $\Omega$ method's performance suffers.
However, a simpler
geometry of the labyrinth (perhaps an elbow bend), or a hallway in concrete with
no air rooms, can show if there is a turnover in labyrinth anisotropy in which
the $\Omega$ methods perform the best.

The results presented in Section \ref{sec:AngleResults} showed that
CADIS-$\Omega$ is generally more resilient than CADIS to changes in
quadrature and P$_N$ discretization.
As a result, CADIS-$\Omega$ can use a coarser problem
discretization to obtain variance reduction parameters, saving
computational cost in terms of both runtime and memory.
The results in Section \ref{sec:AngleResults} also
showed that CADIS-$\Omega$ was less
susceptible to large fluctuations in the relative
errors in intermediate energy energy bins.

Beyond sensitivity to quadrature order and P$_N$ order, it may be worth
investigating the sensitivity of each method to other deterministic calculation
parameters. If, like quadrature order and P$_N$ order, CADIS-$\Omega$ generates
better importance maps with lower-fidelity solutions in other deterministic
parameters, then even more
computational time could be saved.
For something like
mesh refinement, the number of mesh cells can significantly alter the
speed at which the deterministic solution converges.

Investigating the impact of quadrature type may also be an area of future work.
In Section \ref{sec:AngleResults}, it was observed that both CADIS and
CADIS-$\Omega$ showed greater sensitivity to changes in quadrature order than
P$_N$ order. CADIS showed a greater sensitivity to changes in quadrature order
than CADIS-$\Omega$. We expect that the behavior of other quadrature sets will
be similar, but this may be worth verifying in future use cases. It is possible
that the different properties of different quadrature sets may more strongly
affect the $\Omega$-methods' performance.

In addition to characterizing the performance of CADIS-$\Omega$, it will be
important to characterize FW-CADIS-$\Omega$. In Chapter \ref{ch:methodology},
the $\Omega$-method theory for both CADIS and FW-CADIS were presented. Indeed,
FW-CADIS-$\Omega$ has also been implemented in Exnihilo and ADVANTG. The scope
of this project did not extend to the characterization of FW-CADIS-$\Omega$,
though
it could prove useful to characterize for large, global calculations. A similar
set of characterization problems can be designed for FW-CADIS-$\Omega$, but with
global mesh tallies rather than small detectors.

It would also be beneficial to
perform a thorough investigation into the $\Omega$-methods' mitigation or
multiplication of ray effects. Both the forward and the adjoint angular fluxes
will have ray effects in problems with long mean free paths. As discussed in
Section \ref{sec:CharResults},
the ray effects may also be multiplied depending on the geometric
configuration of the problem. The degree to which the $\Omega$-flux exacerbates
or minimizes ray effects as a function of these locations would be an
interesting study and may help in further specifying to which problems the
$\Omega$-methods is suited. Further, the difference in the construction of the
adjoint between CADIS and FW-CADIS means that CADIS-$\Omega$ and
FW-CADIS-$\Omega$ will have different sensitivities to ray effects.

Further characterization of the $\Omega$ methods' performance with different
problem geometry and material configurations will deepen our understanding for
which applications the methods may be best suited. For large scale, high-impact,
high-complexity problems, issues observed in the characterization problem
studies may be exacerbated. Before applying this method to application problems,
it will be important to have confidence that the methods will achieve better
results than other methods.

\subsection{Application Problems}
\label{subsec:appprobs}

Based on the data presented in Sections \ref{sec:CharResults} and
\ref{sec:AngleResults} we believe that the
CADIS-$\Omega$-method has
the potential to be applied to a number of application problems.
These
problems include, but are not limited to: detectors near dry cask nuclear waste
storage, dry cask storage beds, nuclear containment buildings, and nuclear spent fuel
cooling pools.

The dry casks are a promising use case for the CADIS-$\Omega$-method because
they have small air
channels for ventilation, but their body is primarily metal tubes containing
nuclear fuel surrounded by concrete.
These rods are pointed towards the ventilation ducts, and so the
results from the steel bar embedded in concrete suggest that this may be a more
complex application of the physics it represents.

Further, a bed of dry cask storage containers will have several spaces through
which particles may travel. A use case of this may be to calculate the dose rate
standing at the boundary of such a facility, or to consider if the cask loading
matches the owner-provided loading list. Because this problem has so much
air, it may be more difficult for the $\Omega$-methods. However, with the thick
soil boundary in the $z-$plane the $\Omega$-methods may still perform well.

Nuclear spent cooling pools have used fuel rods clustered in assemblies arranged
in rows submerged in water. These rods emit a range of
highly energetic particles. Spent fuel cooling
pools will be an interesting extension of the steel beam in concrete, as water
is a highly moderating material not dissimilar to the concrete from the
characterization problem. The fuel rods act as both a source and a preferential
flowpath, so the differing source distribution in this
problem may yield interesting results.

Each of these application problems uses the physics modeled in the
characterization problems but applies them to a more geometrically and
materially complex problem. In extending the
$\Omega$-methods to these problems and comparing them to CADIS and FW-CADIS, we
can also understand how sensitive the $\Omega$-methods are to more difficult
problems. If, as noted in the characterization problems subsection, the
$\Omega$-methods are more resilient to deterministic problem solution fidelity
in larger more complex problems,
these problems will benefit significantly from the decreased deterministic
solution time and the lower computational burden demanded by the
$\Omega$-methods.
