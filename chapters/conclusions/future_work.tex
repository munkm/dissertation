\section{Suggested Future Work}
\label{sec:futurewrk}

\subsection{Software Improvement}
\label{subsec:softwareimp}

Subsection \ref{subsec:softwareimp} will address the improvements that could be
made to the software and analysis methods to enhance understanding of the omega
methods. First we will discuss how improving software performance will aid in
future work. Then we will discuss how extending the breadth of analysis will
help to understand the $\Omega$-methods further.

To calculate the $\Omega$ flux, an adjustment to either the adjoint flux is
required to ensure that the directional variable $\Omega$ is the same in both
fluxes, or that $\Omega_{adjoint} = \Omega_{forward}$. Because quadrature sets
are not always straightforward to interpolate, a rotationally symmetric
quadrature set is desirable for computing the $\Omega$ fluxes. Should a method
be developed that does have interpolatable quadrature points, it would be a good
candidate to calculate $\Omega$ fluxes for solutions that are not
rotationally symmetric.

- mention parallelization \\
- mention integration into production code \\
- mention analysis script specifics \\
* like the anisotropy measurements above the mean and the median \\
* like comparing them between problems (which is not done now) \\
* like

\subsection{Characterization Problem Extension}
\label{subsec:extendcharprobs}

Subsection \ref{subsec:extendcharprobs} will describe how broadening the scope
of characterization problems is another fruitful avenue for exploration. In this
vein, there is a two-pronged approach: first extending the types of problems
(more diverse materials, less air in problems, more diverse geometries) will
enhance knowledge of the methods. Next, extending the scope of the parametric
will be discussed. In this vein, the deterministic calculation specifics, like
quadrature type will be addressed.


\subsection{Application Problems}
\label{subsec:appprobs}

Based on the data presented in Section \ref{subsec:}, we believe that the
$\Omega$ methods can be applied to a number of application problems. These
problems include, but are not limited to BLAH BLAH BLAH because of the following reasons.
