\section{Assessment of the $\Omega$-methods}
\label{sec:assessments}

The results in Chapter \ref{ch:charprobs} showed that CADIS-$\Omega$ has varied
performance when compared to CADIS over the problem space investigated.
Depending on the geometric configuration, the material composition, and the
solver options used, the method can outperform CADIS by an order of magnitude or
it can be outperformed by an order of magnitude. This underscores the difficulty
of developing a method that is broadly applicable to a large subset of
application space. Further, it illustrates the necessity for further methods
development.

Several characterization problems were formulated to induce anisotropy in the
flux by different physical mechanisms. These mechanisms for flux anisotropy were
either from the source, or from physical interactions with the problem materials
and geometry. The success of the $\Omega$-methods was not directly correlated
with any single physical mechanism, but both CADIS and CADIS-$\Omega$ struggled
in problems comprised of primarily air. Further, CADIS-$\Omega$ struggled more
than CADIS in problems with air and fairly isotropic fluxes.

In the single turn labyrinth, CADIS-$\Omega$ achieved lower relative errors in
epithermal and fast energy groups. These groups were shown to have flux
anisotropies with anisotropy distributions that were clumped around a particular
anisotropy values.
For the multiple turn labyrinth, CADIS achieved uniformly lower relative errors
than CADIS-$\Omega$. For both the steel beam in concrete and the u-shaped bend,
CADIS-$\Omega$ achieved lower relative errors than CADIS but had runtimes 3-7x
longer than those of CADIS. For the geometrically complex rebar-embedded
concrete, CADIS-$\Omega$ had higher relative errors than CADIS. In high energy
regions the convergence for energy bins would take days of computational runtime
to get to a relative error of less than 10\%. For two heavily air-centered
problems, CADIS-$\Omega$ and CADIS both had comparable relative error
achievements.

In addition to checking the limitations of the $\Omega$-methods with respect to
geometry and material composition, the sensitivity of the methods to
deterministic parameter selection was studied also. In particular, the effect of
quadrature order and P$_N$ order were studied. For both CADIS and
CADIS-$\Omega$, the change in quadrature order had a stronger effect on the
change in the relative error and the FOM. CADIS showed stronger sensitivity to
both change in P$_N$ order and quadrature order over CADIS-$\Omega$. CADIS also
proved to have more and greater magnitude oscillations in the relative error
between different P$_N$ and quadrature orders. Spikes in the relative error
occurred in both methods, but more frequently in CADIS. Both
methods showed improvement in the FOM and relative error with increasing
quadrature order and P$_N$ order.
In high energies, CADIS-$\Omega$ achieved superior FOMS to CADIS for all P$_N$
orders and quadrature orders.

Chapter \ref{ch:charprobs} showed a few examples of the anisotropy metrics when
they showed promising trends with I$_{RE}$ or I$_{FOM}$. These metrics did
provide information on the relative distribution of anisotropy in the problem,
and they also showed some trends with the improvement factors. However, many
problems did not have significant trends, so more work must be done to fully
characterize hybrid methods using this novel analysis technique.

The $\Omega$-methods have been characterized with their sensitivity to
geometric and material configuration, as well as their sensitivity to
deterministic calculation parameter choice. It is clear from the results in
Sections \ref{sec:CharResults} and \ref{sec:AngleResults} that the
$\Omega$-methods are not always the best choice for variance reduction methods.
This is from a combination of many effects, but primarily the varied range of
runtimes when compared to CADIS. In many problems, CADIS-$\Omega$ was able to
obtain lower relative errors for tally bins than CADIS, but the runtimes were
significantly longer. The generally longer runtime for CADIS-$\Omega$
negatively impacts the FOMs that it is able to achieve, thus negating its more
effective transport of particles.
