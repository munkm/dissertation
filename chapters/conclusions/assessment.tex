\section{Assessment of the $\Omega$-methods}
\label{sec:assessments}

The results in Chapter \ref{ch:charprobs} showed that CADIS-$\Omega$ has varied
performance when compared to CADIS over the problem space investigated.
Depending on the geometric configuration, the material composition, and the
solver options used, the method

Summarize the results of the char probs

Summarize the results of the angle study

Summarize the effectiveness of the anisotropy metrics

Draw larger conclusions from it.

The $\Omega$-methods have been characterized with their sensitivity to
geometric and material configuration, as well as their sensitivity to
deterministic calculation parameter choice. It is clear from the results in
Sections \ref{sec:charresults} and \ref{sec:angleresults} that the
$\Omega$-methods are not always the best choice for variance reduction methods.
This is from a combination of many effects, but primarily the varied range of
runtimes when compared to CADIS. In many problems, CADIS-$\Omega$ was able to
obtain lower relative errors for tally bins than CADIS, but the runtimes were
significantly longer. The generally longer runtime for CADIS-$\Omega$
negatively impacts the FOMs that it is able to achieve, thus negating its more
effective transport of particles.
