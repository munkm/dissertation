\section{Method Recommendations}
\label{sec:method_recs}

% [Summarize results from characterization problems.] \\

In this chapter the

% [Summarize findings from angle sensitivity study.] \\

% [Based on the results, we recommend that the following parameters be used for
% the $\Omega$-methods.] \\

% [Address gaps in angle sensitivity study.] \\

The characterization problems that were run were heavily biased towards
low-density streaming to induce anisotropy in the flux. In retrospect, this subset of
problems was not ideal for a method so dependent on weight-window type biasing,
because particle streaming allowed for particles to cross several orders of
magnitude in the flux before re-sampling. This meant that in a high-importance
region a particle may split many thousands of times in a new splitting event.
Unfortunately, the $\Omega$-methods are not immune to this issue and so suffered
the same effects as CADIS, even with positive effects like the reduction of ray
effects. Further, with the strong dependence on angle, the $\Omega$-fluxes may
have exacerbated this streaming-sampling effect in regions with strong angular
dependence around the detector. In a problem like the single turn labyrinth,
where the $\Omega$-flux generated a strong line of importance between the exit
of the labyrinth and the detector and drastically dropped the importance behind
the detector, a particle has much more opportunity to cross several orders of
magnitude of importance than it does in CADIS. This is likely what caused
CADIS-$\Omega$ to take longer in Monte Carlo transport than CADIS in all of the
characterization problems.

% [Conclude chapter with potential application problems that the $\Omega$-methods
% might apply to.] \\
