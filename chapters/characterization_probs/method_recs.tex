\section{Method Recommendations}
\label{sec:method_recs}

The performance of CADIS-$\Omega$ has been characterized and compared against
CADIS and a standard, nonbiased analog Monte Carlo run for a series of problems.
Section \ref{sec:CharResults} showed how varying geometric configuration and
material composition of various problems with anisotropy affected the
performance of the $\Omega$ methods.
Subsections \ref{subsec:pnorder} and \ref{subsec:quadorder} showed how varying
P$_N$ order and quadrature order changed the tally results and tally
convergences for the steel beam problem embedded in concrete. In doing this
characterization, we sought to determine in which problems and with which solver
options the $\Omega$ methods were best suited. A secondary objective was to
determine the sensitivity of the $\Omega$-methods to changes in the solver
options. With these objectives in mind, we can evaluate the $\Omega$ methods'
performance based on the study performed in the preceding subsections.

\subsection{Problem Selection}
\label{sec:materials_recs}

\subsection{Deterministic Solver Choice}
\label{sec:deterministic_recs}

The angle-based parametric study provided a number of interesting obervations on
the performance of the $\Omega$ methods.
First, the effect of T$_{det}$ does not change the FOM with CADIS-$\Omega$ more
than CADIS. In
Section \ref{subsec:quadorder} the hypothesis that I/O requirements would severely
impact the FOM for CADIS-$\Omega$ was shown to not be as impactful as
hypothesized. The FOM change between
FOM$_{MC}$ and FOM$_{hybrid}$ was roughly the same for CADIS as CADIS-$\Omega$
because the CADIS-$\Omega$ runtimes are so much longer than CADIS.

Next, the only consistent region in which CADIS-$\Omega$ outperforms CADIS is in
high energies. For almost all P$_N$ orders and all quadrature orders, CADIS-$\Omega$
achieved lower relative errors and higher FOMs than CADIS. In high energy bins,
increasing quadrature order showed a decrease in I$_RE$, increasing P$_N$ order
did not show a large change in I$_RE$. In the same bins, I$_{FOM}$ values above
unity were observed for both P$_N$ and S$_N$ order, but no trends with changing
parameter value were observed.


By including the runtime to calculate
the FOM, the comparative performance of CADIS-$\Omega$ dropped when compared to
using the relative error. Several
energy bins in CADIS-$\Omega$--for quadrature orders and
P$_N$ orders--achieved better FOMs than
CADIS. However no P$_N$ order consistently outperformed the other, while low
S$_N$ orders generally achieved better FOMS for CADIS-$\Omega$ than CADIS.
However, despite the lack of consistent performance for a single P$_N$ order,
the raw values obtained with P$_N$ order are promising. With P$_N$ order
there were more energy bins that had high I$_{FOM}$ values than with quadrature
order.

Another observation that can be extended from Section \ref{sec:CharResults} is
that CADIS-$\Omega$ consistently biases
particles better than CADIS. For the same number of source particles,
CADIS-$\Omega$
achieves lower relative error than CADIS for most energy bins with both P$_N$
order and quadrature order. This means that while sampling may be slow, the
importance map generated with the $\Omega$ flux is generally
better at moving particles to the tally region than
CADIS.

Based on the results in Section \ref{sec:AngleResults}, a number of
recommendations can be made based on deterministic solver choice. First, the
best P$_N$ order choice is
dependent on the energy range in which one is tallying. For low energy regions,
P$_N$ order 1 will give the best FOMs relative to CADIS, for intermediate
energies P$_N$ 3 is a better choice, and for high energies any P$_N$ order is
satisfactory. In general, because lower P$_N$ orders have lower runtimes, these
will get the best results for CADIS-$\Omega$ the fastest, and have comparatively
the best relative errors and FOMs against CADIS. Next, the best S$_N$ order
choice is

If one has to choose between varying P$_N$ order and S$_N$ order to improve the
importance map for their method, varying S$_N$ order will have a greater impact.
This is the case for using either CADIS or CADIS-$\Omega$. However, both methods
have a turnarount point at which increasing S$_N$ order does not improve the
relative error enough to offset the time increase of the method. For
CADIS-$\Omega$, this occurs in bins above S$_N$ 15, and for CADIS it occurs in
bins aboe S$_N$ 12. For this type of problem, and using all energy bins in the
tally, CADIS-$\Omega$ will obtain the
best results with a lower P$_N$ order and intermediate S$_N$ orders.

\subsection{Lessons Learned}
\label{sec:lesslearn}

The characterization problems that were run were heavily biased towards
low-density streaming to induce anisotropy in the flux. This subset of
problems, though highly anisotropic, are not the best
for a method so dependent on weight-window type biasing,
because particle streaming allowed for particles to cross several orders of
magnitude in the flux before re-sampling. This meant that in a high-importance
region a particle may split many thousands of times in a new splitting event.
Unfortunately, the $\Omega$-methods are not immune to this issue and so suffered
the same effects as CADIS, even with positive effects like the reduction of ray
effects. Further, with the strong dependence on angle, the $\Omega$-fluxes may
have exacerbated this streaming-sampling effect in regions with strong angular
dependence around the detector. In a problem like the single turn labyrinth,
where the $\Omega$-flux generated a strong line of importance between the exit
of the labyrinth and the detector and drastically dropped the importance behind
the detector, a particle has much more opportunity to cross several orders of
magnitude of importance than it does in CADIS. This is likely what caused
CADIS-$\Omega$ to take longer in Monte Carlo transport than CADIS in many of the
characterization problems.

It should also be noted that while the angle-dependent parametric study revealed
how P$_N$ order and quadrature order may affect a problem's results, the
best parameter choices for this problem are by no means a prescriptive solution
for other problems. Section \ref{sec:CharResults}
showed how different the characterization problems' results were, depending on
the source definition, the material composition of the problem, and the
geometric configuration of the problem. Using the deterministic parameter
choices that appear the best for the steel beam in concrete may not be the best
for, say, a multi-turn labyrinth. From this study we have a good starting point
from which to further characterize the method for other application problems.

