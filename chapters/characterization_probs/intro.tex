The $\Omega$-methods have been presented in the previous chapters.
Foundational previous work and a description of other hybrid methods' successes
and deficiencies were presented in Chapter \ref{ch:lit_review}. The
theoretical basis for the $\Omega$-methods were described in Chapter
\ref{ch:methodology}. In this chapter, the $\Omega$-methods will be applied to a
number of small, anisotropy-inducing problems. Recall that the $\Omega$-methods
are a version of CADIS and FW-CADIS that used an adjusted contributon-based flux
rather than a pure-adjoint flux to generate biasing parameters.
The $\Omega$-methods' performance
will be compared to CADIS and analog Monte Carlo.
Because the $\Omega$-methods have been designed to generate variance reduction
parameters in problems
where there is a strong degree of anisotropy in the flux, their
characterization will also depend on testing them in anisotropic
problems.

This chapter will begin by presenting characterization problems
that have been designed to have anisotropic particle flux as a result of
different physical characteristics. The results from each problem will follow. Two problems that highlight interesting
aspects of the $\Omega$-methods will be used in a deeper parametric study to
determine the $\Omega$-methods' sensitivity to different angular flux
information. Using the results obtained from this study,
recommendations on favorable
parameters with which to run the $\Omega$-methods will be made. \\

