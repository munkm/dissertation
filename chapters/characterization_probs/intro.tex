The $\Omega$-methods have been presented in the previous chapters.
In this chapter, the CADIS-$\Omega$-method is applied to a
number of small, anisotropy-inducing problems. Recall that the $\Omega$-methods
are a version of CADIS and FW-CADIS that use an adjusted contributon-based flux
rather than a pure-adjoint flux to generate biasing parameters.
The CADIS-$\Omega$-method's performance
is compared to CADIS and standard nonbiased Monte Carlo.
Because the $\Omega$-methods have been designed to generate variance reduction
parameters in problems
where there is a strong degree of anisotropy in the flux, their
characterization is dependent on testing them in anisotropic
problems.
This chapter begins with a presentation of the characterization problems
that have been designed to induce anisotropy in the particle flux by
different physical mechanisms. The results of the $\Omega$-methods when
applied to these problems follows. Two problems that highlight interesting
aspects of the $\Omega$-methods are subsequently used in a deeper parametric study to
determine the $\Omega$-method's sensitivity to different angular flux
information. Using the results obtained from this study,
recommendations on favorable
parameters with which to run the $\Omega$-methods are made. \\

