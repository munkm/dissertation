\section{Sensitivity to Deterministic Angle Variations}
\label{sec:AngleResults}

Section \ref{sec:CharResults} showed that the $\Omega$-methods have a strong
weakness to ``thin'' materials, as CADIS and FW-CADIS do. It also showed that
the incorporation of the $\Omega$-flux into a problem with materials with
strongly different moderating properties, the steel plate in concrete, showed
strong improvement when compared to both CADIS and nonbiased Monte Carlo.

[Describe problem selection decision.] \\

[Describe physical reasons why these are important for testing the method.] \\

\subsection{Parametric Study Description}
\label{subsec:parstudy}

The angle sensitivity parametric study will cover the subset of computational parameters
that are most likely to influence the $\Omega$ method's solution. Because the
$\Omega$-flux is calculated from an angle integration of the forward and adjoint
flux, calculation parameters that are most likely to influence the angular flux
solution are the variables that were perturbed.
The two parameters that will be studied
are the quadrature order and the P$_N$ order.

The quadrature used in a deterministic solution is used do discretize the
problem in angle. Quadrature options are split into two separate selections: the
quadrature set or type, and the quadrature order. Because the $\Omega$-methods
require rotational symmetry, only quadrature sets that have rotational
symmetry (generally these are triangular quadrature sets)
can be used with the $\Omega$-methods. In ADVANTG/Denovo, the triangular
quadrature sets are: linear-discontinuous finite element, level-symmetric, and
quadruple range. Quadrature sets differ in
properties and are a realm of study unto their own. Quadrature orders specify
how fine of a resolution the quadrature set will be

Other deterministic parameters may influence the variance reduction parameters
calculated by the $\Omega$ methods.
The spatial discretization, while not a primary factor influencing
the angular flux, still may affect the $\Omega$-methods' performance.
A finer energy group structure may also influence the $\Omega$-method solution.
Finer energy groups will more effectively reflect resonance regions in
scattering and absorption. Scattering effects in certain energy regions will
have angular dependence and, thus, may be a stronger effect on the angular flux
than a coarser energy discretization. Because these particular solution effects
are not strongly tied to influencing the angular flux, they will not be included
in the angular sensitivity parametric study. This is because the effects on the
angular flux will be hard to isolate, and conclusions drawn
from them will be harder to make. However, a study extending to include the
energy group structure, the spatial discretization, and the quadrature type
certainly could be an area of future work.

Several factors in the deterministic calculation should not have a strong effect
on the angular flux whatsoever. These include the spatial solver, the
convergence limits of the solvers, and the within group solver types.
Because these factors should not influence the angular flux any more than any
other part of the solution, they will not be included in this parametric study.

\subsection{Quadrature Type}
\label{subsec:quadtype}

[Show FOMs as a function of quad type for first problem] \\

[Show interesting anisotropy plots for extreme-valued quad types.] \\

[Show flux maps for each extreme-valued problem in region of interest.] \\

[Describe results.] \\

[Show FOMs as a function of quad type for second problem] \\

[Show interesting anisotropy plots for extreme-valued quad types.] \\

[Show flux maps for each extreme-valued problem in region of interest.] \\

[Describe results.] \\

\subsection{Quadrature Order}
\label{subsec:quadorder}
[Show FOMs as a function of quad order for first problem] \\

[Show interesting anisotropy plots for extreme-valued quad types.] \\

[Show flux maps for each extreme-valued problem in region of interest.] \\

[Describe results.] \\

[Show FOMs as a function of quad order for second problem] \\

[Show interesting anisotropy plots for extreme-valued quad types.] \\

[Show flux maps for each extreme-valued problem in region of interest.] \\

[Describe results.] \\

\subsection{Scattering (P$_N$) Order}
\label{subsec:pnorder}
[Show FOMs as a function of pN order for first problem] \\

[Show interesting anisotropy plots for extreme-valued pN orders] \\

[Show flux maps for each extreme-valued problem in region of interest.] \\

[Describe results.] \\

[Show FOMs as a function of pN order for second problem] \\

[Show interesting anisotropy plots for extreme-valued pN orders] \\

[Show flux maps for each extreme-valued problem in region of interest.] \\

[Describe results.] \\

\subsection{Observations}
\label{subsec:observations}

[Go back through results and highlight benefits and pitfalls.] \\

[Describe reasons why these might have occurred. Follow up with a list of options
that could be done in further testing to confirm reasons.] \\
