\section{Description of the Characterization Problems}
\label{sec:AngleProbDesc}

[Introduce the types of problems that worked vs. didn't work in CADIS and other
methods.] \\

\subsection{Descriptors of Anisotropy-Inducing Physics}
\label{subsec:AngleProbDescriptors}

[From the problems that didn't work for other methods, extrapolate and describe
the physical reasons why these problems caused issues.]

[Make sure to talk about both physical effects and deterministically-introduced
issues, like ray effects.]

[Summarize physical effects, like streaming paths, highly heterogenous,
monodirectional sources, etc.]

\subsection{Problem Specifications}
\label{subsec:ProbSpecs}

With the anisotropy-inducing physics described in Section
\ref{subsec:AngleProbDescriptors}, a set of characterization problems that have
different combinations of each of these effects can be conceptualized. With
different anisotropy-inducing physics in the system, these problems can provide
an overview of how the $\Omega$-methods perform in an assortment of anisotropic
problems. The primary anisotropy-inducing features identified are: streaming
paths, material heterogeneity, materials with highly differing scattering
probabilities, and monodirectional sources. As previously described, these fall
into two broad categories: anisotropy caused by the problem materials and
geometry, and anisotropy caused by the source definition. In the next few
paragraphs, each problem will be described and will be accompanied by
a justification for which
anisotropy-inducing physics is in each problem. A summary of which physics are
in each problem is provided in Table \ref{tab:probphysics}.

% [Describe problems that have streaming paths.]

% [Describe problems that are highly heterogenous.]

% [Describe problems with monodirectional sources.]

% [Describe problems that have high scattering discontinuity.]

Now that the broad subset of characterization problems have been described,
the physics that each contains is summarized in Table \ref{tab:probphysics}.
Upon glancing at the table, one may see that it can be difficult to separate
out one physical mechanism from another when constructing problems. This is a
deficiency of the characterization problem construction, and is certainly an
area that may be improved upon in future work.

\begin{table}[h!]
  \centering
  \begin{tabular}{l|C{2cm}C{2cm}C{2cm}C{2cm}C{2cm}}
% \begin{tabular}{l|ccccc}
\toprule
\multirow{2}{*}{Problem Name} &  \multicolumn{4}{c}{Problem Coverage} \\
{} &  Streaming Paths & Highly Scattering & Material Heterogeneity &
Monodirectional Source & Ray \newline Effects \\
\midrule
Beamline              & x &   &   & x & x \\
Maze variants         &   &   &   &   &   \\
\textit{Single turn}  & x & x & x &   & x \\
\textit{Multi-turn}   & x & x & x &   & x \\
Steel plate           &   & x & x & x & x$^{\dagger}$  \\
U-shaped corridor     & x & x & x &   &   \\
Shielding with rebar  &   & x & x & x &   \\
Therapy Room          & x &   & x & x & x \\
\bottomrule
\end{tabular}
\begin{flushleft}
\footnotesize{
  $^{\dagger}$ May have ray effects in low density region exiting the metal
  plate, but effects will be less pronounced than other problems.
}
\end{flushleft}

  \caption[Anisotropy-inducing physics of each of the characterization problems.]
  {Anisotropy-inducing physics of each of the characterization problems.
  Each identified anisotropy-inducing physical metric is used in different
  combinations for the characterization problems. This will help to aid in
  extrapolating to which real problems the $\Omega$-methods may be applied.}
  \label{tab:probphysics}
\end{table}
