\section{Motivation}
\label{sec:motivation}

Radiation shielding is a realm of continued importance for nuclear engineering
and health physics applications. With the continued use of nuclear technology,
the potential proliferation of nuclear materials, and the expansion of
nuclear medicine applications, tools with which to predict the behavior
of these systems are
in ever-increasing demand. Over the course of several decades, radiation
transport methods has been developed in two primary areas: stochastic (Monte
Carlo) and
deterministic transport methods.

These tools have the potential to be immensely
powerful, but are not without their drawbacks. Monte Carlo methods have the
benefit of modeling transport that is continuous in energy, space, and angle.
A user can obtain results for any region in phase-space that they might desire.
However, Monte Carlo methods also require adequate sampling in order to obtain a
solution with sufficient precision. Depending on the complexity of the problem,
this may be difficult, computationally demanding, very time consuming,
or impossible. Deterministic
transport methods discretize the problem phase-space in space, energy, and
angle. They iteratively converge on a global problem solution that is equally
valid across the entire problem, rather than a potentially localized tally
location. They tend to be much faster than Monte Carlo methods, but also lose
the continuity in phase-space that is offered by Monte Carlo. Depending on how
coarse the problem is discretized, this may obfuscate regions or physics of
interest in the problem.

Hybrid methods leverage the speed and uniform solution validity of
deterministically-obtained transport solutions to bias Monte Carlo transport to
more effectively sample in regions of interest. Biasing Monte Carlo to move
particles to regions of interest is called variance reduction. Hybrid methods
generally automate the variance reduction process to speed up the time to a
desired solution or to achieve a more uniform uncertainty distribution in the
problem.
