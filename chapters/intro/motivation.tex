\section{Motivation}
\label{sec:motivation}

Radiation shielding is a realm of continued importance for nuclear engineering,
nuclear security, and health physics applications.
With the expansion of nuclear technology
applications,
the potential proliferation of nuclear materials, and the continued development of
nuclear medicine, tools with which to predict the behavior
of these systems are
in ever-increasing demand. Over the course of many decades, radiation
transport methods have been developed in two primary areas: stochastic (Monte
Carlo) and
deterministic.

These tools have the potential to be immensely
powerful, but are not without their drawbacks. Monte Carlo methods have the
benefit of modeling transport that is continuous in energy, space, and angle.
A user can obtain results for any region in phase-space that one might desire.
However, Monte Carlo methods also require adequate sampling in order to obtain a
solution with sufficient precision. Adequate sampling depends on the number of
particles transported to the tally region. The more particles that are run in a
problem, the
longer the computational time required. Depending on the complexity of the problem,
this may be difficult, computationally demanding, very time consuming,
or impossible.

Deterministic
transport methods discretize the problem phase-space in space, energy, and
angle. They iteratively converge on a global problem solution that is equally
valid across the entire problem space, rather than a potentially localized tally
location. Deterministic solvers tend to be much faster
than Monte Carlo methods, but also lose
the continuity in phase-space that is offered by Monte Carlo. Depending on the
coarseness of the problem discretization,
features of interest in the particle flux may be incorrect, obfuscated,
or missed entirely.

Hybrid methods leverage the speed and uniform solution validity of
deterministically-obtained transport solutions to bias Monte Carlo transport to
more effectively sample in regions of interest. Biasing Monte Carlo to move
particles to regions of interest more effectively
is called variance reduction. Many existing
implementations of hybrid methods
automate the variance reduction process to speed up the time to a
desired solution or to achieve a more uniform uncertainty distribution in the
problem.

Hybrid methods have been designed for an assortment of applications, and none are
universally applicable to all problem types. In particular, hybrid methods are
wanting for a method well-suited for
highly anisotropic, deep-penetration radiation transport applications. The work
presented herein endeavors to provide a potential solution for such applications.
