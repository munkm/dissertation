\section{Research Objectives}
\label{sec:objectives}

This dissertation has a number of objectives that it will address. First, the
$\Omega$-methods

Objectives: \\
- Use theoretical basis to propose new hybrid method \\
- Capitalize on knowledge obtained from existing hybrid methods. \\
- Implement the method in a usable and reproducible manner. \\
- Devise several problems that can characterize the method's strengths and
limitations. \\
- Devise a rigorous and consistent set of metrics with which to measure method
performance. \\

The next several chapters of this dissertation will cover the relevant
background, the pertinent theoretical basis, and the results of the research
contributing to this dissertation. Chapter \ref{sec:lit_review} will provide a
comprehensive background on the theoretical basis on which Monte Carlo methods,
deterministic radiation transport, and hybrid methods for radiation shielding
are founded. Chapter \ref{ch:methodology} will continue with an overview of the
theoretical basis of the method developed for this project. The theoretical work
presented in this chapter is novel and contributes to the larger body of hybrid
methods research. This chapter will also cover the software used for this
project, and in which ways it was modified to incorporate the novel theory
presented herein. Next, Chapter \ref{ch:charprobs} will present several problems
with which the method is to be characterized. The results from these problems
will inform a parametric angle-informed study, which will also be presented
later in the chapter. Finally, Chapter \ref{ch:conclusion} will draw from the
results presented in Chapter \ref{ch:charprobs} to discuss the performance of
the new method and what was learned from the characterization of the method.

