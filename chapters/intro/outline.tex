\section{Outline of the Dissertation}
\label{sec:outline}

The next several chapters of this dissertation covers the relevant
background, the pertinent theoretical basis, and the numerical results that
address the research objectives outlined in Section \ref{sec:objectives}.
Chapter \ref{ch:lit_review} provides a
comprehensive background on the theoretical basis on which Monte Carlo methods,
deterministic radiation transport, and hybrid methods for radiation shielding
are founded. In so doing, it provides context for the existing gaps for
generating variance reduction parameters in
highly anisotropic, deep-penetration radiation transport problems. It
further
highlights the most effective hybrid methods that can be applied to
non-anisotropic, deep-penetration radiation transport problems.

The conclusion of Chapter \ref{ch:lit_review} demarcates the transition from
theoretical background work to the novel contributions of this project.
Building on
the knowledge presented in Chapter \ref{ch:lit_review},
Chapter \ref{ch:methodology} presents an overview of the
theoretical basis of the method developed in this research. The theory
contained in this chapter contributes to the larger body of hybrid
methods research. This chapter also covers the software used for this
project, and how it was modified to incorporate the novel theory
presented herein. Next, Chapter \ref{ch:charprobs} presents several problems
with which the method is to be characterized. The results from these problems
inform a parametric angle-informed study, presented
in the latter portion of the chapter.
Finally, Chapter \ref{ch:conclusions} draws from the
results presented in Chapter \ref{ch:charprobs} to discuss the performance of
the new method, summarize what was learned from the method characterization,
and suggest future paths forward for future hybrid methods research.



