\section{Importance Functions for Variance Reduction}
\label{sec:Importance}

\subsection{The Concept of Importance}
\label{sec:Importance}

The effective use of variance reduction techniques can lead to a faster time to
a desired solution and a reduced variance in the specified tally. However,
specifying variance reduction parameters is not always a straightforward
procedure. In simple geometries, a user might intuitively understand which
regions of a problem may contribute more to a desired solution, but for more
complex geometries, this may not be so obvious. Thus, the concept of importance
is introduced: regions of the problem that are likely to contribute to a
response have higher importance than those that do not. If an importance
function for a system can be obtained, then that importance function can be
strategically used in variance reduction techniques to optimize the monte carlo
simulation.

A number of methods have been developed to obtain a measure of importance for
radiation transport problems. Here a selection of these methods will be
presented: first, obtaining a measure of importance with monte carlo sampling;
second, utilizing the adjoint equation to measure importance; and lastly,
utilizing the contributon solution for importance quantification.

\subsubsection{Quantifying Importance with Monte Carlo}

With the advent of the weight window technique, Booth
\cite{booth_automatic_1982} proposed
a method to quantify a cell's importance within a Monte Carlo simulation. In
this method, Booth suggested estimating the cell's importance using monte carlo
transport as:
\begin{equation}
  Importance  = \frac{\text{score of particles leaving the cell}}
                     {\text{weight leaving the cell}}
\label{eq:BoothImp}
\end{equation}

Booth went on to suggest that this importance generator could be used in a
complementary nature to the weight window variance reduction technique available
in MCNP at that time, where a weight window target value is inversely
proportional to the importance could be used for variance reduction.
In this case, the track weight times the expected score is approximately
constant in the problem.
In test problems, the importance generator had estimated the problem cells'
importances after only a few iterations.

Henricks \cite{hendricks_code-generated_1982}, concurrently to Booth, developed
an automated geometry- and energy-dependent weight window generator for MCNP.
Hendricks' method differed from Booth's in that the weight window generation was
nonspecific to a tally.
Rather, the generator populated all regions of phase space equally.
To do this, the generator set boundaries for the weight window bounds, $W_{low}$
and $W_{high}$, which were calculated based on the total weight entering and
exiting the weight window target region. The specifics of this calculation were
not reported in his paper. However, Hendricks' method likely proved to be
optimized for global tally calculations, while Booth's was for localized
tallies.

However, both Booth and Henricks' methods were reliant on the statistical
precision of the cells sampled to generate their weight window paramaters. For
deep-penetration problems, obtaining a "good" estimate of the cell importances
many mean free paths from the forward source likely took many iterations. With
large fluctuations between iterations, this could be a very slow and
computationally inefficient way to calculate importance in a problem.

\subsection{The Adjoint Solution for Importance}
\label{sec:AdjointImportance}

\subsubsection{Theory}

Utilizing the solution of the adjoint formulation of the neutron transport
equation is, perhaps, one of the most widely recognized methods to utilize as an
importance function.

The integral form of the forward, steady-state, neutron transport equation with
an external source can be given by:
\begin{multline}
\hat\Omega \cdot \nabla \psi
        (\vec {r} ,E,\:\hat\Omega)+\Sigma _{ t }
        (\vec{r},E)\psi (\vec { r } ,E,\:\hat\Omega) = \\
        \int _{ 4\pi  } \int _{ 0 }^{ \infty  } \Sigma _{ s }(E'\rightarrow E,
        \hat\Omega'\rightarrow\hat\Omega)\psi (\vec { r } ,E',\: \hat\Omega')dE'
        \:d\hat\Omega' + q_{e}(\vec { r } ,E, \:\hat\Omega)
\label{eq:F-NTE}
\end{multline}

Where $\vec { r }$, $E$, and $\hat\Omega$, are the direction, energy, and angle
independent variables, and $\psi$ is the neutron flux, $\Sigma$ is the neutron
interaction (scattering, absorption, total) cross section, and $q_{e}$ is the
external source. In general, this equation tells us where particles are moving
throughout the system that we define by equation \ref{eq:F-NTE}. Of note, the
particles move in the scattering term from $E'$ to $E$, and from $\hat\Omega'$
to $\hat\Omega$.

The integral form of the adjoint, steady-state, neutron transport equation with
an external source can be expressed as:
\begin{multline}
-\hat\Omega \cdot \nabla \psi^{\dagger}
        (\vec {r} ,E,\:\hat\Omega)+\Sigma _{ t }
        (\vec{r},E)\psi^{\dagger}  (\vec { r } ,E,\:\hat\Omega)
       = \\
        \int _{ 4\pi  } \int _{ 0 }^{ \infty  } \Sigma _{ s }(E\rightarrow E',
        \hat\Omega\rightarrow\hat\Omega')\psi^{\dagger}  (\vec { r } ,E',\:
        \hat\Omega')dE' \:d\hat\Omega' + q_{e}^\dagger(\vec { r } ,E, \:\hat\Omega)
\label{eq:A-NTE}
\end{multline}

% Make sure to emphasize somewhere that adjoint = importance, and the weights in
% MC need to be inversely proportional to importance (to make sure lots of
% sampling happens).

Note here that the particles in the adjoint equation move from $E$ to $E'$, and
from $\hat\Omega$ to $\hat\Omega'$, which indicates an upscattering in energy
and a reversal of direction. The external source, too, is different, changing
from $q_{e}$ to $q_{e}^\dagger$. For a simple source-detector problem, we set
$q_{e}^\dagger$ to be $\Sigma _{ d }$, or the tally/detector response function
for our system. Thus, the adjoint particles start at low energy at the detector
location, move away from the adjoint source (the detector location), and scatter
up in energy.

Goertzel \cite{goertzel_monte_1958} and Kalos \cite{kalos_importance_1963} first
showed that the solution to the adjoint equation was related to importance.
Further, they showed that if the adjoint solution is known exactly, the forward
problem could be solved with zero variance. Of course, finding an exact solution
for the adjoint equation is just as computationally expensive as an exact
solution to the forward equation, so instead an approximation is used to
optimize Monte Carlo simulations.

Also Lux and Koblinger \cite{lux_monte_carlo}

\subsubsection{Implementation}

Coveyou, Cain and Yost \cite{coveyou_adjoint_1967} expanded on Goertzel and
Kalos' work by interpreting in which ways the adjoint solution could be adapted
for Monte Carlo variance reduction. In particular, they investigated the choice
of biasing schemes and how effective they were at variance reduction for a
simple one-dimensional problem. They reiterated that the adjoint solution is a
good estimate for importance, but should not be calculated explicitly, and
rather estimated by a simpler model. The adjoint function is not necessarily the
most optimal importance function; however, it is likely the closest and most
obtainable estimate of importance that can be calculated. Further, they
concluded that source biasing by the solution to the adjoint equation or by the
expected response is an optimal choice for Monte Carlo variance reduction, as it
can be used independently of any other variance reduction technique, and
provides good results.

Both, Niemal, and Vergnaud \cite{both_automated_1990} derived VR parameters for
the exponential transform and for collision biasing based on the adjoint
solution as a measure of importance. They implemented their results in the
TRIPOLI-3 software.

Tsang and Hoffman \cite{tsang_monte_1988} built off of the parameters derived by
Coveyou et al \cite{coveyou_adjoint_1967} to generate variance reduction
parameters automatically for fuel cask dose rate analyses. In their work, Tsang
and Hoffman utilized a one-dimensional discrete ordinates code XSDRNPM-S to
calculate the adjoint fluxes for their shielding problems. The results from this
claculation were then used to generate biasing parameters for Monte Carlo;
specifically they aimed at generating parameters for energy biasing, source
biasing, russian roulette and splitting, and next event estimation
probabilities. They implemented their work in the SAS4 module in SCALE; it was
one of the earlier implementations of a fully automated biasing procedure for
Monte Carlo.
% Maybe I want to make a new generic "Automated VR" section that comes before the local, global, and angle sections. I could introduce the MCNP WWgenerator, and earlier work like Tsang and Hoffman's.....


The motivated reader may also consider Lewins' book
\cite{lewins_importance:_1965} as a
useful introduction importance functions and the adjoint equation. Lewins' work was
primarily oriented towards the use of the adjoint formulation in reactor physics and
perturbation theory, but that does not eliminate its relevance to this topic.


\subsection{The Contributon Solution for Importance}
\label{sec:ContributonImportance}

Over a number of years, Williams
\cite{williams_generalized_1991,williams_contributorn_1992,williams_contributon_study}
developed contributon theory and corresponding spatial channel theory. In this
area, Williams defined a pseudo-particle, the textit{contributon}, as a particle
that carries response from the radiation source to a detector. Furthermore, the
total number of contributons in a system are conserved by the textit{contributon
conservation principle}: all contributons that are emitted from the source
eventually arrive at the detector.  In his work, Williams noted that one could
go so far as to track contributons, rather than real particles in monte carlo.
Because every particle transported would eventually reach the detector, this
would lead to a zero variance solution. However, the nature of solving the
contributon equation with monte carlo (or any other transport mechanism)
involves knowing the exact solution to the adjoint equation, and so relies on
the same computational obstacles as solving the adjoint NTE.

The equivalent contributon transport equation can be derived in a form analogous
to the forward (eq. \ref{eq:F-NTE} ) and adjoint (eq. \ref{eq:A-NTE}) equations:

Defining the contributon flux as:
\begin{equation}
\Psi (\vec {r} ,E,\:\hat\Omega) = \psi^{\dagger} (\vec {r} ,E,\:\hat\Omega)
        \psi(\vec {r} ,E,\:\hat\Omega)
\label{eq.Cont-Flux}
\end{equation}


The contributon transport equation is:

\begin{multline}
\hat\Omega \cdot \nabla \Psi (\vec {r} ,E,\:\hat\Omega)
+\widetilde{\Sigma} _{ t }(\vec{r},E,\:\hat\Omega)\Psi (\vec { r } ,E,\:\hat\Omega)
     = \\
        \int _{ 4\pi  } \int _{ 0 }^{ \infty  }
        \widetilde{p}(\vec{r}, \hat\Omega', E'\rightarrow\hat\Omega, E)
        \widetilde{P}(\vec{r}, \hat\Omega',E')
        \widetilde{\Sigma} _{ t }(\vec{r}, E', \hat\Omega')
        \Psi (\vec { r } ,E',\: \hat\Omega')dE' \:d\hat\Omega'
        + \hat p(\vec { r } ,E, \:\hat\Omega) R
\label{eq:Cont-NTE}
\end{multline}

where the effective cross sections are given by:
\begin{equation}
\begin{aligned}
\widetilde{\Sigma}_{t}(\vec{r}, E, \hat\Omega) &=
        \widetilde{\Sigma}_{s}(\vec{r}, E, \hat\Omega) +
        \widetilde{\Sigma}_{a}(\vec{r}, E, \hat\Omega)    \\
     &= \frac{\iint \Sigma_{s}(\vec{r},\hat\Omega'\cdot\hat\Omega'',
        E'\rightarrow E'')}{\psi^{\dagger}(\vec{r}, E, \hat\Omega)}
        + \frac{Q^{\dagger}(\vec{r}, E, \hat\Omega)}
        {\psi^{\dagger}(\vec{r}, E, \hat\Omega)}
\end{aligned}
\end{equation}

The scattering probability of a contributon at position $\vec{r}$, $E'$, and
$\hat\Omega'$ is:
\begin{equation}
\widetilde{P}(\vec{r}, \hat\Omega',E') =
         \frac{\widetilde{\Sigma} _{ s }(\vec{r}, E', \hat\Omega')}
       {\widetilde{\Sigma} _{ t }(\vec{r}, E'Q, \hat\Omega')}
\end{equation}

and
\begin{equation}
\widetilde{p}(\vec{r}, \hat\Omega', E'\rightarrow\hat\Omega, E) =
       \frac{\Sigma_{s}(\vec{r},\hat\Omega'\cdot\hat\Omega,E'\rightarrow E)
       \psi^{\dagger} (\vec{r}, E, \hat\Omega)}
       {\iint \Sigma_{s}(\vec{r},\hat\Omega'\cdot\hat\Omega'',E'\rightarrow
       E'')\psi^{\dagger} (\vec{r}, E'', \hat\Omega'')d\hat\Omega'' dE''}
\end{equation}
is the probability that a contributon scattering at $\vec{r}$, $E'$, and $\hat\Omega'$ will scatter into $d\hat\Omega$ $dE$.

\begin{equation}
\hat p(\vec{r}, E, \hat\Omega) =
\frac{psi^{\dagger}(\vec{r}, E, \hat\Omega) Q(\vec{r},E,\hat\Omega)}
     {\int \int \int \psi^{\dagger}(\vec{r'},E',\hat\Omega')
     Q(\vec{r'},E',\hat\Omega') d\hat\Omega' dE' dV'}
\end{equation}
is the contributon source distribution.

and
\begin{equation}
R = \int \int \int \psi^{\dagger}(\vec{r},E,\hat\Omega)Q(\vec{r},E,\hat\Omega)
    d\hat\Omega dE dV
\end{equation}
is the reaction rate.


The derivation of eq. \ref{eq:Cont-NTE} and its corresponding variables is
available in a number of the sources referenced in this section, so we will
abstain from doing so at this time.

As mentioned in the previous section, the adjoint flux tells the user what the
importance of a particle is to a response function. Conversely, the contributon
flux tells the user what the importance of a particle is to the solution.
Becker's thesis \cite{becker_hybrid_2009} aptly points out that this is
illustrated most dramatically in a source-detector problem, where the forward
source has little importance to the adjoint source, but does have importance to
the problem solution.

Williams recognized the applications of the pseudo-particles, contributons, to
shield design and optimization. In particular, Williams noted that variables
relevant to contributon response were useful in determining transport paths
through media \cite{williams_contributon_study, williams_SCC_shielding}. A study
of the contributon density in a system could enlighten the user on locations
where particles would preferentially be transported, and so designers could
intelligently choose where to place detectors and material. In this way, the
user could find the particles most influential to the response of the system. A
region with high response is the most important to a detector tally. The
variables of response described by williams are the the response potential, the
response current, and the response vorticity \cite{williams_contributorn_1992}.

Contributon theory and spatial channel theory have been applied successfully to
shielding analyses \cite{seydaliev_contributon_2008, williams_SCC_shielding} due
to its ability to incorporate particle response throughout the entire system
effectively.

