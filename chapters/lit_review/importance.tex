\section{Importance Functions for Variance Reduction}
\label{sec:Importance}

The effective use of variance reduction techniques can lead to a faster time to
a desired solution and a reduced variance in the specified tally. However,
specifying variance reduction parameters is not always a straightforward
procedure. In simple geometries, a user might intuitively understand which
regions of a problem may contribute more to a desired solution, but for more
complex geometries, this may not be so obvious.
In the following subsections, the theory in determining which regions of a
problem are important to eliciting a tally response will be described.
The first topic discussed will be
the concept of importance and obtaining a measure of importance with
Monte Carlo sampling.
Second, the adjoint equation and its relation to importance will be introduced.
Last, the contributon solution and how its relation to tally responses is
reviewed.

\subsection{The Concept of Importance}
\label{sec:Importance}

The concept of importance is, in essence, a means of defining which regions of
a problem that are likely to contribute to a
response and which are less likely to contribute to a response. The regions
that are more likely to generate a response will have a
higher importance than those that do not. As one might imagine, this may be
intuitively deduced for a simple problem, but in a more complex problem
that might not be possible.
If an importance
function for a system can be obtained computationally, then that importance
function can be
strategically used in variance reduction techniques to speed up the Monte Carlo
calculations.

As described in Section \ref{subsec:AutomatedMCVR}, Booth
\cite{booth_automatic_1982} proposed
a method to quantify a cell's importance within a Monte Carlo simulation
(Eq. \eqref{eq:BoothImp}). In
this method, Booth suggested estimating the cell's importance using Monte Carlo
transport as:
\begin{equation*}
  Importance  = \frac{\text{score of particles leaving the cell}}
                     {\text{weight leaving the cell}}.
\end{equation*}
This particular calculation of importance
follows from the intuitive explanation for importance in the preceding
paragraph. Recall from Section \ref{subsec:MCVR} that in variance reduction
methods, the population of particles is increased in important regions such that
the number of samples or particles contributing to a tally increases, but the
total problem weight is conserved. More important regions should have many
lower-weight particles to reduce the tally variance. Using Booth's bookkeeping
method for estimating regional importance,
if a cell has a greater weight leaving the cell than the number
of particles, that means that the relative contribution of that cell to the
tally is likely to be lower than other regions. If, instead, the number of
particles leaving the cell is greater than the weight leaving the cell, then
that region is more important to the tally response, because that particle
population is higher than other cells.

While this
estimation of the importance requires only a Monte Carlo forward calculation of
the problem, it was referred to as the forward-adjoint importance generator
\cite{booth_automatic_1982, booth_deep_1982, booth_importance_1984} because the
bookeeping tracked by Eq. \eqref{eq:BoothImp} was a forward-approximation of the
adjoint. Adjoint theory and how it relates to importance will be discussed in
Section
\ref{sec:AdjointImportance}.
Booth's estimation of importance was used to generate
weight window target values inversely proportional to the importance.
In this case, the track weight times the expected score is approximately
constant in the problem. Choosing this inverse relationship between the weight
window and importance is common practice in variance reduction, and is often a
good choice because it is nearly optimal over a broad range of the problem
phase-space \cite{booth_common_2012}.

It should be noted that Booth's method is reliant on the statistical
precision of the cells sampled to generate their importances. For
deep-penetration problems, obtaining a ``good'' estimate of the cell importances
many mean free paths from the forward source took several iterations. With
large fluctuations between iterations, this had the potential to
be a very slow and
computationally inefficient way to calculate importance in a problem. Using a
solution of the adjoint that is equally valid across all of the problem space
would be more ideal for deep-penetration problems.

\subsection{The Adjoint Solution for Importance}
\label{sec:AdjointImportance}

Using the solution of the adjoint formulation of the neutron transport
equation is one of the most widely recognized methods for generating
an importance function. This subsection will begin with a brief summary of
adjoint theory. A discussion on how
the adjoint solution differs physically from the forward solution for a
particular problem will follow. Last, some early investigations on how the
adjoint
and importance are related will be summarized.

\subsubsection{Theory}

In previous sections we have reviewed the statistical precision of Monte
Carlo-based methods, and how sampled results from Monte Carlo can be sped up
with variance reduction methods. We have also briefly addressed the forward- and
the adjoint- solutions for a particular problem. In neutron transport, the
integral form of the forward, steady-state, particle transport equation can be
defined as:
\begin{multline}
\hat\Omega \cdot \nabla \psi
        (\vec {r} ,E,\:\hat\Omega)+\Sigma _{ t }
        (\vec{r},E)\psi (\vec { r } ,E,\:\hat\Omega) = \\
        \int _{ 4\pi  } \int _{ 0 }^{ \infty  } \Sigma _{ s }(E'\rightarrow E,
        \hat\Omega'\rightarrow\hat\Omega)\psi (\vec { r } ,E',\: \hat\Omega')dE'
        \:d\hat\Omega' + q_{e}(\vec { r } ,E, \:\hat\Omega),
\label{eq:F-NTE}
\end{multline}
where $\vec { r }$, $E$, and $\hat\Omega$ are direction, energy, and angle, respectively, giving six dimensions of phase space in total;
$\psi$ is the neutron flux, $\Sigma$ is the neutron
interaction (scattering, absorption, total) cross section, and $q_{e}$ is the
external fixed source. Alternatively, this can be written in operator form,
\begin{equation}
  H \psi = q_{e}\:,
\label{eq:F-NTE2}
\end{equation}
where $H$ represents the streaming, scattering, and absorptive terms from Eq.
\eqref{eq:F-NTE}, $\psi$ is the angular flux as it is in Eq. \eqref{eq:F-NTE}, and
$q_{e}$ is the source term. 

The forward transport equation tells us where particles are moving
throughout the system. Of note: the
particles move in the scattering term from $E'$ into $E$, and from $\hat\Omega'$
into $\hat\Omega$. Therefore, for a particular problem with a given $q_{e}$,
particles start at $q_e$ and move throughout the system,
either scattering in energy, streaming out of the problem, or
getting absorbed by the problem materials.

The adjoint equation of the same form, in contrast, can be expressed as:
\begin{multline}
-\hat\Omega \cdot \nabla \psi^{\dagger}
        (\vec {r} ,E,\:\hat\Omega)+\Sigma _{ t }
        (\vec{r},E)\psi^{\dagger}  (\vec { r } ,E,\:\hat\Omega)
       = \\
        \int _{ 4\pi  } \int _{ 0 }^{ \infty  } \Sigma _{ s }(E\rightarrow E',
        \hat\Omega\rightarrow\hat\Omega')\psi^{\dagger}  (\vec { r } ,E',\:
        \hat\Omega')dE' \:d\hat\Omega' + q_{e}^\dagger(\vec { r } ,E,
        \:\hat\Omega),
\label{eq:A-NTE}
\end{multline}
or in operator form as
\begin{equation}
  H^{\dagger} \psi^{\dagger} = q_{e}^{\dagger}.
\label{eq:A-NTE2}
\end{equation}
Note here that the particles in the adjoint equation move from $E$ into $E'$, and
from $\hat\Omega$ into $\hat\Omega'$.
% RNS: I'd explain that the forward problem is tracking how all parts of phase space contribute to a specific location and the adjoint is how a location is contributing to other parts of the problem. That's the difference in prime ordering (also it's not up vs. down scattering. maybe say in vs. outscattering?)

The external source, too, is different, changing
from $q_{e}$ to $q_{e}^\dagger$. This source should be chosen to be the objective of the forward problem. In Monte Carlo variance reduction, we often seek to obtain
information about a detector or similar response for the system.
The response for the forward problem
given a defined source distribution  $q(\vec{r}, E, \hat{\Omega})$ is
\begin{subequations}
\begin{equation}
  R_{tally} = \int_{4\pi} \int_{V} \int_{E} \psi(\vec{r}, E, \hat{\Omega})
  \Sigma_{tally}(\vec{r},E, \hat\Omega) dE dV d\hat\Omega ,
\end{equation}
which can be simplified using bracket notation: 
\begin{equation}
  R_{tally} = \langle \psi \Sigma_{tally} \rangle \:,
\end{equation}
\end{subequations}
where the brackets indicate an
integration over all phase-space and $\Sigma_{tally}$ is the effective tally
response function. 

For a simple source-detector problem, we choose
$q_{e}^{\dagger}$ to be $\Sigma_{tally}$, or that the adjoint source is the
tally/detector response function
for the system. Therefore the adjoint particles start at low energy at the detector
location, move away from the adjoint source (the detector location), and scatter
up in energy.
% RNS: How do you know which direction in energy they scatter? I don't think this is correct. 
 By making the choice that $q_{e}^{\dagger} = \Sigma_{tally}$, the
response function can be written as a product for the forward flux and the
adjoint source
\begin{equation}
  R_{tally} = \langle \psi q^{\dagger} \rangle .
  \label{eq:response1}
\end{equation}
By using the adjoint identity 
\begin{equation}
  \langle \psi, H^{\dagger} \psi^{\dagger} \rangle =
  \langle \psi^{\dagger}, H \psi \rangle ,
\end{equation}
Eq.\ \eqref{eq:response1} can be written to be a function of the
adjoint flux and the forward source distribution
\begin{equation}
  R = \langle \psi^{\dagger} q \rangle .
  \label{eq:response2}
\end{equation}

% RNS: I'm not sure if what is below has been deleted or moved? the merge for
% this kind of stuff is always a little tricky. You're going to have to double
At this point, we know that the adjoint problem transports particles from the adjoint source (which is the detector or tally
location) into the problem phase-space. However, it may not be immediately obvious how this adjoint solution
relates to importance for the forward solution. Let us start with a simple
illustrative example: a monoenergetic, monodirectional, point source. The
forward source takes the form of a delta function:
\begin{equation*}
  q(\vec{r}, E, \hat{\Omega}) = \delta(\vec{r}-\vec{r}_0) \delta(E-E_0)
  \delta(\hat{\Omega}-\hat{\Omega}_0) .
\end{equation*}

Using this definition of the forward source and evaluating Eq.
\eqref{eq:response2}, we obtain
\begin{equation*}
  \begin{split}
    R &= \langle \psi^{\dagger} q \rangle \\
    &= \int_{V} \int_{E} \int_{\Omega} \psi^{\dagger}(\vec{r}, E, \hat{\Omega})
       q(\vec{r}, E, \hat{\Omega}) d\hat\Omega dE dV \\
       & = \psi^{\dagger}(\vec{r}_0, E_0, \hat{\Omega}_0).
\end{split}
\end{equation*}
This result shows that the solution to the adjoint equation is the detector
response for the forward problem. As a result, the adjoint flux can be used as
an indicator of a particle produced in $\vec{r}, E, \hat{\Omega}$ contributing
to a response in the system. This indicator can be thought of as the particle's
importance to achieving the tally or response objective. Consequently, it is
often said that the adjoint is the importance function for the problem.

The adjoint solution is used in nuclear engineering for a number of
applications, including reactor physics and perturbation theory
\cite{lewins_importance_1965, lewins_developments_1968,
greenspan_developments_1976, lux_monte_1991}.
However, Goertzel and Kalos' early work recognized its
application for deep-penetration radiation shielding.
Goertzel and Kalos \cite{goertzel_monte_1958}
showed analytically
that the exact adjoint solution, if used as an importance or weighting
function for the forward Monte Carlo calculation, will result in a zero variance
solution for the forward Monte Carlo problem. Further, Kalos
\cite{kalos_importance_1963} showed in a 1D infinite hydrogen slab problem that
an analytically-derived adjoint importance function significantly helped neutron
transport for deep-penetration problems.

Goertzel and Kalos' finding that an exact adjoint can lead to a zero variance
solution means that if a single particle is transported with the adjoint
weighting function, its score will be the same as the total system response.
Only a single particle is required to get an exact solution for the forward
problem. This is prohibitive because obtaining
an exact adjoint solution is just as computationally expensive as getting
an exact forward solution. Instead, one seeks to obtain a good but inexpensive estimate of the
adjoint solution based on computational limitations. A good importance estimate
should help reduce the variance in a reasonable amount of time, and be
relatively computationally inexpensive. A Monte Carlo solution can
provide a continuous solution over the problem phase-space. However, as
discussed in Section \ref{subsec:MCVR}, the quality of this adjoint solution
relies on the number of samples used to calculate it and that may take a
significant amount of time. A deterministic solution could
offer equal or better solution confidence across the entire problem in much less time, though it is
discretized
in space, energy, and angle. For deep-penetration importance functions, we will
opt for deterministically-obtained solutions due to the solution's equally
distributed validity.

\subsubsection{Implementation}

Coveyou, Cain, and Yost \cite{coveyou_adjoint_1967} expanded on Goertzel and
Kalos' work by interpreting in which ways the adjoint solution could be adapted
for Monte Carlo variance reduction. In particular, they investigated the choice
of biasing schemes and how effective they were at variance reduction for a
simple 1D problem. They reiterated that the adjoint solution is a
good estimate for importance, but should not be calculated explicitly, and
rather estimated by a simpler model. The adjoint function is not necessarily the
most optimal importance function; however, it is likely the closest and most
obtainable estimate of importance that can be calculated
\cite{coveyou_adjoint_1967}. They
concluded that source biasing by the solution to the adjoint equation or by the
expected response is an best choice for Monte Carlo variance reduction, as it
can be used independently from any other variance reduction technique and
provides good results.

Tang and Hoffman \cite{tang_monte_1988} built off of the parameters derived by
Coveyou et al.\ \cite{coveyou_adjoint_1967} to generate variance reduction
parameters automatically for fuel cask dose rate analyses. In their work, Tang
and Hoffman utilized the 1D discrete ordinates code XSDRNPM-S to
calculate the adjoint fluxes for their shielding problems. The results from this
calculation were then used to generate biasing parameters for Monte Carlo;
specifically, they aimed at generating parameters for energy biasing, source
biasing, Russian roulette and splitting, and next event estimation
probabilities. They implemented their work in the SAS4 module in SCALE
\cite{SCALE6_1}; it was
one of the earlier implementations of a fully-automated deterministic
biasing procedure for Monte Carlo.

\subsection{The Contributon Solution for Importance}
\label{sec:ContributonImportance}

Contributon theory is another useful concept that can be used as a measure of
importance
\cite{williams_generalized_1991,williams_contributorn_1992,williams_contributon_study}.
However, contributon theory quantifies importance differently than
adjoint theory. In contributon transport, a pseudo-particle, the
\textit{contributon}, is defined. The contributon carries response in the problem
system from the radiation source to a dectector. The
total number of contributons in a system are conserved by the \textit{contributon
conservation principle}: all contributons that are emitted from the source
eventually arrive at the detector. 

Much of the work in this area has been done
by Williams and collaborators
\cite{williams_generalized_1991,williams_contributorn_1992,williams_contributon_study}.

We have generally introduced the concept of the contributon pseudo-particle and
the response that it carries. This will be continued with a brief mathematical
overview of contributon transport and contributon response.
The contributon transport equation can be derived in a form analogous
to the forward (Eq. \ref{eq:F-NTE}) and adjoint (Eq. \ref{eq:A-NTE}) equations.
The derivation of Eq. \ref{eq:Cont-NTE} and its corresponding variables is
available in a number of the sources referenced in this section, so we will
abstain from rederiving it here.
The angular contributon flux is defined
as the product of the forward- and
adjoint- angular- fluxes:
\begin{equation}
\Psi (\vec {r} ,E,\:\hat\Omega) = \psi^{\dagger} (\vec {r} ,E,\:\hat\Omega)
        \psi(\vec {r} ,E,\:\hat\Omega)\:.
\label{eq.Cont-Flux}
\end{equation}
The contributon transport equation is:
\begin{multline}
\hat\Omega \cdot \nabla \Psi (\vec {r} ,E,\:\hat\Omega)
+\widetilde{\Sigma} _{ t }(\vec{r},E,\:\hat\Omega)\Psi (\vec { r } ,E,\:\hat\Omega)
     = \\
        \int _{ 4\pi  } \int _{ 0 }^{ \infty  }
        \widetilde{p}(\vec{r}, \hat\Omega'\rightarrow\hat\Omega, E'\rightarrow E)
        \widetilde{P}(\vec{r}, \hat\Omega',E')
        \widetilde{\Sigma} _{ t }(\vec{r}, E', \hat\Omega')
        \Psi (\vec { r } ,E',\: \hat\Omega')dE' \:d\hat\Omega'
        + \hat p(\vec { r } ,E, \:\hat\Omega) R .
\label{eq:Cont-NTE}
\end{multline}
% RNS: don't worry about it for the draft to send, but while you're waiting on comments back you might want to have E, and omega in the same order everywhere. It's not actually important, but is easier to read if you're not familiar with the equations.
% RNS: another potential thing (though it's okay not to) is to look for where you can cut out unnecessary words and abbreviate things--makes it easier to read as well.
The units of phase-space are the same as observed in the forward and adjoint
transport equations. The symbols decorated with tildes denote variables that are
unique to the contributon equation. $\widetilde{p}$ and $\widetilde{P}$ are both
probability functions related to scattering and $\widetilde{\Sigma}$ are
effective cross sections.
The effective cross sections are given by:
\begin{equation}
\begin{aligned}
\widetilde{\Sigma}_{t}(\vec{r}, E, \hat\Omega) &=
        \widetilde{\Sigma}_{s}(\vec{r}, E, \hat\Omega) +
        \widetilde{\Sigma}_{a}(\vec{r}, E, \hat\Omega)    \\
     &= \frac{\iint \Sigma_{s}(\vec{r},\hat\Omega''\cdot\hat\Omega,
         E\rightarrow E'') \psi^{\dagger}
         (\vec{r}, \Omega'', E'') d\Omega'' dE''}
         {\psi^{\dagger}(\vec{r}, E, \hat\Omega)}
        + \frac{Q^{\dagger}(\vec{r}, E, \hat\Omega)}
        {\psi^{\dagger}(\vec{r}, E, \hat\Omega)}.
\end{aligned}
\end{equation}
% RNS: I don't think these equations are super clear.
Note here that the effective scattering and absorption cross sections are
adjoint flux-dependent. Where the adjoint flux becomes small, the interaction
probabilities will become large. As a result, regions where the adjoint flux
is high interaction probabilities become low, causing streaming. Conversely, regions with low adjoint
fluxes--like the problem boundary--will have a very high cross section, thus
encouraging particle transport back towards the adjoint source. This
increased probability of interaction in low flux regions encourages response
transport towards the detectector or tally, thus contributing to a response.

The scattering probability of a contributon at position $\vec{r}$, $E'$, and
$\hat\Omega'$ is:
\begin{equation}
\widetilde{P}(\vec{r}, \hat\Omega',E') =
         \frac{\widetilde{\Sigma} _{ s }(\vec{r}, E', \hat\Omega')}
       {\widetilde{\Sigma} _{ t }(\vec{r}, E'Q, \hat\Omega')}\:,
\end{equation}
% RNS: what is the Q in the denomiator about? Is that correct?
and
\begin{equation}
\widetilde{p}(\vec{r}, \hat\Omega', E'\rightarrow\hat\Omega, E) =
       \frac{\Sigma_{s}(\vec{r}, \hat\Omega'\rightarrow\hat\Omega, E'\rightarrow E)
       \psi^{\dagger} (\vec{r}, E, \hat\Omega)}
       {\iint \Sigma_{s}(\vec{r},\hat\Omega'\cdot\hat\Omega'',E'\rightarrow
       E'')\psi^{\dagger} (\vec{r}, E'', \hat\Omega'')d\hat\Omega'' dE''}.
\end{equation}
The distribution function governing the contributon source is
\begin{equation}
\hat p(\vec{r}, E, \hat\Omega) =
\frac{\psi^{\dagger}(\vec{r}, E, \hat\Omega) Q(\vec{r},E,\hat\Omega)}
     {\int \int \int \psi^{\dagger}(\vec{r'},E',\hat\Omega')
     Q(\vec{r'},E',\hat\Omega') d\hat\Omega' dE' dV'},
\end{equation}
note that the contributon source is actually defined in Eq.
\eqref{eq:Cont-NTE} by the product of $\hat{p}$
and $R$. $R$ is contributon production rate; it
is given by integral of the adjoint flux and the forward source
\begin{equation}
  \begin{split}
R &= \int \int \int \psi^{\dagger}(\vec{r},E,\hat\Omega)Q(\vec{r},E,\hat\Omega)
    d\hat\Omega dE dV \\
  & = \langle \psi^{\dagger}Q  \rangle
\end{split}
\end{equation}
which is recognizable as the system response described
in Section \ref{sec:AdjointImportance}. It can also be shown by integrating Eq.
\eqref{eq:Cont-NTE} over all phase space and ensuring that the function
$\hat{p}$ is normalized, that
\begin{equation}
  R = \langle \widetilde{\Sigma}_a \Psi \rangle,
  \label{eq:contribprod}
\end{equation}
or that the rate at which contributons that die in the detector is the same as
the rate at which they are produced by the contributon source.
Knowing that $R$ is the contributon production rate, let us consider the
probability that a particle will be absorbed in the detector, or $P$, given by
\begin{equation}
  P  = \langle \Sigma_a \psi \rangle.
\end{equation}
Adding a factor of $\psi^{\dagger}/\psi^{\dagger}$ to the terms on the right hand
side, this becomes
\begin{equation}
  P = \Big \langle \frac{\Sigma_a}{\psi^{\dagger}} \psi \psi^{\dagger} \Big
  \rangle.
\end{equation}
By using the identities from the contributon equation, this is also
\begin{equation}
  P = \langle \widetilde{\Sigma_a} \Psi \rangle .
\end{equation}
Next, substituting the definition from Eq. \eqref{eq:contribprod} into this
equation, it follows that
\begin{equation}
  P = R.
  \label{eq:PR}
\end{equation}
This is the same
\textit{contributon conservation principle} introduced at the beginning of this
section.
Williams noted that one could
go so far as to transport contributons rather than real particles with Monte Carlo.
Because every particle transported would eventually reach the detector and give
an exact value for R (as shown by Eq. \eqref{eq:PR}), this
would lead to a zero variance solution. However, the nature of solving the
contributon equation with Monte Carlo (or any other transport mechanism)
involves knowing the exact solution to the adjoint equation, and so relies on
the same computational obstacles as solving the adjoint transport equation and
using that alone for an importance measure.

As mentioned in the previous section, the adjoint flux is an indicator of a
particle's importance to inducing a response. Conversely, the contributon
flux describes the importance of a particle to the solution.
Becker's thesis \cite{becker_hybrid_2009} aptly points out that this is
illustrated most dramatically in a source-detector problem, where the forward
source has little importance to the adjoint source, but does have importance to
the problem solution. As a result, both the contributon solution and the adjoint
solution can be considered importance functions for a problem, but the
importance that they describe differs.

Williams recognized the applications of contributons to
shield design and optimization 
 in an extension of contribution theory called
spatial channel theory. In particular, Williams noted that variables
relevant to contributon response were useful in determining transport paths
through media \cite{williams_contributon_study, williams_SCC_shielding}. A study
of different contributon values throughout the system could enlighten users on
regions with higher response potential. This could then be used to intelligently
choose regions for detector locations or add to shielding.
The contributon values in this theory include the contributon flux, the
contributon density, the contributon current, or the contributon velocity
\cite{williams_relations_1977}.
In this way, the
user could find the particles most influential to the response of the system. A
region with high response potential is the most important to a detector tally. The
variables of response described by Williams are the response potential, the
response current, and the response vorticity \cite{williams_contributorn_1992}.

Contributon theory and has been applied successfully to
shielding analyses \cite{seydaliev_contributon_2008, williams_SCC_shielding} due
to its ability to show particle flow between a source and response effectively.
% RNS: what is spatial channel theory? First time it's mentioned... I'd remove.
%Williams and Engle showed that
%spatial channel theory can be used in reactor shielding analyses.
%In their work, they used
%contributon currents to determine preferential flow paths through the Fast Flux
%Test Facility (FFTF) \cite{williams_SCC_shielding}.
Seydaliev \cite{seydaliev_contributon_2008} used angle-dependent forward and
adjoint
fluxes and currents to visualize the contributon flux for
simple source-detector problems. In this work, he showed that contributon flow
in the system behaves much like a fluid between the source and detector,
following preferential flow paths more densely.
Seydaliev also observed ray effects in the
contributon flux for high energy photons, and traditional methods like using a
first collision source, did not remedy the issue. The contributon formulation of
particle transport can show important particle flow paths between a source and a
detector, but it is still not immune to computational obstacles that exist for
standard forward- and adjoint- transport.

The past few subsections have described the different means by which importance
can be defined or quantified for a problem.
As discussed in Section \ref{sec:Importance}, generating an importance function
with Monte Carlo is limited in that the quality of the importance map is only as
good as the regions that are sampled. For deep-penetration problems, it may be
prohibitively difficult to obtain adequate importance sampling with traditional
Monte Carlo methods. 

Deterministically-obtained importance functions, however,
offer the benefit of a solution that is equally valid across all of the problem's
solution space. This is because the deterministic solution's precision is
limited to convergence criteria, not sampling of individual particles. Using a
deterministic solution is often faster and much less computationally-intensive
than Monte Carlo for importance quantification.
As a result, many hybrid methods opt
to use deterministically-obtained importance functions to generate variance
reduction parameters for Monte Carlo transport.
