\section{Automated Angle-Informed Variance Reduction Methods}
\label{sec:AngleVR}

In a number of problems, the angular dependence of the flux is significant enough that
biasing by space and energy, but not angle, as detailed in the previous sections, is not
sufficient. As a result, a subset of hybrid methods were developed to incorporate some
degree of the flux anisotropy in variance reduction parameters. Without explicitly
calculating the angular flux, which is memory- and storage- intensive, many methods
approached angular biasing without explicitly utilizing the angular flux. Initial approaches
to angular biasing focused on approximating the angular flux $\psi$ as a separable
function of the scalar flux and an angle-dependent multiplier.

\subsection{AVATAR}

The AVATAR \cite{van_riper_generation_1995, van_riper_avatarautomatic_1997} method generates
three-dimensional, space-, energy- and angule-dependent weight windows for monte carlo by
utilizing a relatively course-mesh and few-angle deterministic calculation in THREEDANT
[REFERENCE FOR CODE] and using Justine [REFERENCE FOR CODE] to generate Monte Carlo weight
windows for MCNP. AVATAR avoided generating explicit angular fluxes with THREEDANT by assuming
that the adjoint angular flux is symmetric about the average adjoint current vector,
$J^{\dagger}$ :

\begin{subequations}
\label{avatareqns}
\begin{equation}
\psi^{\dagger}(\hat \Omega \cdot n) = \phi^{\dagger} f(\hat\Omega \cdot n)
\end{equation}
where
\begin{equation}
n = \frac{J^{\dagger}}{ \left\| J^{\dagger} \right\| }
\end{equation}
and f is derived from the maximum entropy distribution:
\begin{equation}
f(\hat\Omega \cdot n) = \frac{\lambda e^{(\hat\Omega \cdot n)\lambda}} {2 \sinh {\lambda} }
\end{equation}
and $\lambda$ is a function of the average cosine $\bar\mu$
\begin{equation}
\begin{split}
\lambda  & = \frac{2.99821 \bar\mu - 2.2669248 {\bar\mu}^2 }{1 - 0.769332 \bar\mu -0.519928 {\bar\mu}^2 + 0.2691594 {\bar\mu}^3} \\
        & = \frac{1}{1-\bar\mu} \\
\end{split}
\end{equation}
for $ 0 \le \bar\mu < 0.8001$ and $0.8001 \le \bar\mu < 1.0 $, respectively. Also,
\begin{equation}
\bar\mu(\vec{r} E) = \frac{\left\| J^{\dagger}(\vec{r} E) \right\|}{\phi^{\dagger}(\vec{r} E)}
\end{equation}
\end{subequations}

AVATAR's space- energy- and angle-dependent weight window could then be calculated by:
\begin{equation}
\bar {w} (\vec{r},E,\hat\Omega) = \frac{k}{\phi^{\dagger}(\vec{r},E) f(\hat\Omega \cdot n)}
\end{equation}
where k was a constant that could be adjusted to match the source distribution.
AVATAR exclusively generated weight windows, and did not attempt to bias the source
distribution. Physically, the assumption behind AVATAR is that the adjoint angular flux
is locally one-dimensional, so azimuthal symmetry is assumed.

In implementation, the authors of AVATAR showed that AVATAR's angularly-dependent
weight windows improved the FOM (from 2- to 5-times) for a multiple-tally well-logging
problem compared to the MCNP weight window generator. The user input for AVATAR was
also minimal, but did require that the user specify the adjoint source spectrum and
strength manually.
%
%
%----------------------------------------------------------------------------------------

\subsection{Simple Angular CADIS}

Simple Angular CADIS \cite{peplow_consistent_2012} built off of existing implementations
of CADIS and FW-CADIS by employing the method used by AVATAR and consistently biasing the
source distribution with the weight windows. In their work, Peplow et al implemented their
work in MAVRIC [CODE REFERENCE] to incorporate directionally-dependent weight windows
both with- and without directionally-dependent source biasing.

The Simple Angular CADIS approach includes the azimuthal distribution in its approximation
of the angular flux, such that:
\begin{equation}
\psi^{\dagger}(\vec{r}, E, \hat \Omega) \cong \phi^{\dagger}(\vec{r}, E) \frac{1}{2\pi} f(\hat\Omega \cdot n)
\end{equation}
and the angle-dependent weight windows are then given by:
\begin{equation}
\bar {w} (\vec{r},E,\hat\Omega) = \frac{2 \pi k}{\phi^{\dagger}(\vec{r},E)
                                  f(\hat\Omega \cdot n)}
\end{equation}

For the first method, with directionally-dependent weight windows and without directional
source biasing, the biasing parameters are:
\begin{subequations}
\label{SA-CADIS1}
\begin{equation}
\begin{split}
\hat{q}(\vec{r},E,\hat\Omega) & = \frac{1}{R} q(\vec{r},E) \phi^{\dagger}
                                  (\vec{r},E) \frac{1}{2 \pi} q(\hat\Omega \cdot \hat d)\\
                              & = \hat{q}(\vec{r},E) \frac{1}{2 \pi} q(\hat\Omega \cdot \hat d)
\end{split}
\end{equation}
\begin{equation}
\begin{split}
w_0  & = \frac{q(\vec{r},E,\hat\Omega)}{\hat{q}(\vec{r},E,\hat\Omega)} \\
     & = \frac{R}{\phi^{+}(\vec {r} ,E)}
\end{split}
\end{equation}
\begin{equation}
\begin{split}
\bar{w} (\vec{r},E,\hat\Omega)  & = \frac{R}{\phi^{\dagger}(\vec{r},E)}
                                    \frac{f(\hat{\Omega_0}\cdot n(\vec{r_0}, E_0))}
                                    {f(\hat\Omega \cdot n)} \\
                                & = \bar{w}(\vec{r},E) \frac{f(\hat{\Omega_0}\cdot
                                    n(\vec{r_0}, E_0))}{f(\hat\Omega \cdot n)}
\end{split}
\end{equation}
\end{subequations}
Note that the biased source distribution, $\hat{q}(\vec{r},E,\hat\Omega)$, is a function of
the biased source distribution from traditional space- energy- CADIS and of the original
directional source distribution; which is why this method has directional weight windows,
but not directional source biasing.

For the second method, with directionally-dependent weight windows and with directional
source biasing, the biasing parameters are given by:

\begin{subequations}
\label{SA-CADIS2}
\begin{equation}
\begin{split}
\hat{q}(\vec{r},E,\hat\Omega) & = \frac{1}{Rc} q(\vec{r},E,\hat\Omega)
                                  \phi^{\dagger}(\vec{r},E,\hat\Omega) \\
                              & = \left[ \frac{1}{R}q(\vec{r},E) \phi^{\dagger}(\vec{r},E)
                                  \right] \left[\frac{1}{c}\frac{1}{2 \pi} q(\hat\Omega
                                  \cdot \hat d) \frac{1}{2 \pi}
                                  f (\hat\Omega \cdot n_0)\right]  \\
                             & = \hat{q}(\vec{r},E) \left[\frac{1}{c}\frac{1}{2 \pi}
                                 q(\hat\Omega \cdot \hat d) \frac{1}{2 \pi}
                                 f(\hat\Omega \cdot n_0) \right]
\end{split}
\end{equation}
\begin{equation}
c = \int{\frac{1}{2 \pi} q(\hat\Omega \cdot \hat d) \frac{1}{2 \pi}
    f (\hat\Omega \cdot n_0} d\hat\Omega
\end{equation}
\begin{equation}
\begin{split}
w_0  &= \frac{q(\vec{r},E,\hat\Omega)}{\hat{q}(\vec{r},E,\hat\Omega)} \\
     &= \frac{R}{\phi^{+}(\vec {r} ,E)} \frac{2 \pi c}{f(\hat\Omega \cdot n_0)}
\end{split}
\end{equation}
\begin{equation}
\begin{split}
\bar{w} (\vec{r},E,\hat\Omega)  &= \frac{R}{\phi^{\dagger}(\vec{r},E)}
                                   \frac{2 \pi c}{f(\hat\Omega \cdot n_0)} \\
                                &= \bar{w}(\vec{r},E) \frac{2 \pi c}{f(\hat\Omega \cdot n)}
\end{split}
\end{equation}
\end{subequations}

To test these two modifications of CADIS, the authors ran a number of test problems and
compared them against standard implementations of CADIS and analog monte carlo runs. Simple
angular CADIS with- and without- source biasing showed improvement for some problems, but not
others. Notably, the modified CADIS methods performed well in a simple duct streaming problem,
but for many problems did not sufficiently increase the FOM to compensate for the increase in
calculation time to generate the variance reduction parameters.

%
%
%----------------------------------------------------------------------------------------

\subsubsection{Angular Biasing Using the Exponential Transform}

In 1985, Henricks and Carter \cite{hendricks_anisotropic_1985} described a method by which
photon transport could be biased in angle with an exponential transform adjustment factor.
In this study, the authors performed studies on three test problems with the exponential
transform adjustment factor and with a synergistic angular bias and exponential transform
adjustment.
In all studies, the synergistic biasing outperformed the exponential transform adjustment
alone.
However, their method performed best in highly absorbing media.
The authors noted that this performance was due to the fact that the biasing could be made
strong without undersampling scattering in the problem.
They also pointed out that while the weight window method was comparable in efficiency to the
method described, their method avoided the choice of choosing importances and weight window
values for biasing. Their method was limited to exclusively photon transport biasing, and not
neutron transport; the authors were optimistic that the method could be extended to neutron
transport.

The LIFT \cite{turner_automatic_1997} method, developed by Turner and Larsen, is a modification
of the zero variance solution that utilizes a deterministic calculation to generate biasing
parameters for the exponential transform and weight window variance reduction techniques.
First, the adjoint angular flux is approximated, piecewise continuous in space in angle,
with the following equations:

\begin{subequations}
\label{LIFT}
\begin{equation}
\psi^{\dagger}_{g,n} (\vec{r},\Omega) \approx \phi^{\dagger}_{g,n}V_n \left[ \beta_{g,n}
                \frac{\sigma_{s_0,g \rightarrow g,n} b_{g,n}(\Omega)}{\sigma_{t,g,n} -
                \rho_{g,n}\cdot \Omega} e^{\rho_{g,n}\cdot (r-r_n)} \right]
\end{equation}
where the physical system is comprised of $V_n$ regions, and $\psi^{\dagger}_{g,n}$ is the
approximation of the angular flux for group g and region n. Further, $\beta$, the normalization
factor is given by:
\begin{equation}
\beta_{g,n} = \frac{1}{\int_{V_n} e^{\rho_{g,n} \cdot (r-r_n)} dr \int_{4\pi}
              \frac{\sigma_{s_0,g \rightarrow g,n} b_{g,n}(\Omega)}{\sigma_{t,g,n} -
              \rho_{g,n}\cdot \Omega} d\Omega}
\end{equation}
b, the linearly anisotropic factor, is
\begin{equation}
b_{g,n}(\hat\Omega) = 1 + 3\mu_{g\rightarrow g,n}\frac{\sigma_{t,g,n} -
                      \sigma_{s_0,g\rightarrow g,n}}{{\left| \rho_{g,n}
                      \right|}^2}\rho_{g,n}\cdot \Omega
\end{equation}
and the biasing parameter is given by
\begin{equation}
\rho_{g,n} = \sigma_{t,g,n} \lambda_{g,n}
\end{equation}
\end{subequations}

The first equation in this series is an adjustment of the exponential transform. However,
rather than relying upon an isotropic scattering law, like earlier implementations, the LIFT
method adjusts the transform to instead be linearly anisotropic in angle. The parameters
$\beta_{g,n}$, $b_{g,n}$ and $rho_{g,n}$ are calculated from values obtained from the
deterministic calculation and are used to calculate $\psi^{\dagger}_{g,n}$.

In addition to utilizing the exponential transform to bias the particles in angle, the LIFT
method also utilized weight windows for particle weight adjustment. However, the computational
cost of generating angle-dependent weight windows from the previous equations led the authors
to choose space-energy exclusive weight windows. The weight window target values were
calculated to be inversely proportional to the adjoint solution, as with other methods.
\begin{equation}
ww_{center,g,n} = \frac{\phi^{\dagger}_{g,src}}{\phi^{\dagger}_{g,n}}
\end{equation}

Turner compared a number of variants of LIFT \cite{turner_automatic_1997-1} against AVATAR to
determine the efficiency of LIFT. In his investigation, Turner compared LIFT and AVATAR using
approximations for the adjoint solution with diffusion and $S_N$ transport calculations, and
various methods to calculate weight window paramaters, including utilizing LIFT combined
with AVATAR's weight window parameters. In most cases, LIFT outperformed AVATAR. In problems
with voids and low-density regions, the efficiency of the LIFT method decreased, but so did
AVATAR. However, an important note that Turner mentioned was that a while increasing the
accuracy of the deterministic solution may decrease the variance, it is not necessarily the
best for the FOM. This is a valuable lesson for all automated variance reduction methods: an
overly accurate solution for the adjoint problem may reduce the variance but come at such
a high computational cost that it will decrease the FOM.

[Kendra's addition to LIFT]

Cooper and Larsen, in addition to generating global isotropic weight windows, as described
in the previous section, also developed angle-dependent weight windows
\cite{cooper_automated_2001}. Here, the forward angular flux is calculated in a similar
manner as the AVATAR method:
\begin{subequations}
\label{LIFT}
\begin{equation}
\psi(\vec{r},\hat\Omega) \approx A(\vec{r})e^{\vec{B}_r \cdot \hat\Omega}
\end{equation}
where $A(\vec{r})$ and $\vec{B}(\vec{r})$ are given by:
\begin{equation}
A(\vec{r}) = \frac{\phi(\vec{r})}{4 \pi}\frac{B(\vec{r})}{sinh B(\vec{r})}
\end{equation}
\begin{equation}
\vec{B}(\vec{r}) = B(\vec{r}) \frac{\vec{\lambda}}{\left| \vec{\lambda} \right|}
\end{equation}
B can be calculated with
\begin{equation}
\lambda(\vec{r}) = coth B(\vec{r}) - \frac{1}{B(\vec{r})}
\end{equation}
Cooper noted that $\lambda(\vec{r})$ could be estimated with either the scalar fluxes
and currents from the quasidiffusion calculation, or with the scalar fluxes and currents
from the Monte Carlo solution. Cooper noted that because Monte Carlo robustly calculated
these values, it was the more optimal choice. Knowing how to calculate these parameters
from the deterministic calculation, the angular weight window can be calculated with:
\begin{equation}
ww_{i,j} (\hat\Omega)= \frac{\psi_{i,j}(\vec{r},\hat\Omega)} {max\phi_{i',j'}/4\pi}
\end{equation}
\end{subequations}

As mentioned in the previous section, Cooper's method was limited in that it used an
iterative quasidiffusion / Monte Carlo solution to generate the biasing parameters for
the problem. This method was not automated; and the ideal frequency between iterations
was never explored. However, Cooper showed that the angularly dependent weight windows
significantly improved the figure of merit as compared to analog Monte Carlo. Further, the
angular weight windows performed slightly better than the isotropic weight windows in
evenly distributing the problems in problems with anisotropy.
MCNP Weight Window Generator \textbf{cite MCNP manual here}


The AVATAR method
\cite{van_riper_generation_1995, van_riper_avatarautomatic_1997} used an
approximation of the angular flux--without explicitly calculating it--to
generate angle-dependent weight windows. It operated under the approximation
that
the angular flux was separable and symmetric about the average current vector.
The angular flux was then calculated using
a product of a deterministically-obtained
scalar flux and an exponential function, derived from the
maximum entropy distribution, that was a function of the scalar flux and the
current. Space-, energy-, and angle-dependent weight windows for
the Monte Carlo problem were then generated from the inverse of the angular
flux. AVATAR improved the FOM for sample problems from 2 to 5 times, but did not
apply to problems where the flux was not azimuthally symmetric.

The LIFT method \cite{turner_automatic_1997, turner_automatic_1997-1}, like
AVATAR, calculated the angular flux for a region by assuming the angular flux
was a product of the scalar flux and an exponential function. The angular flux
values were then used to generate values for the exponential transform variance
reduction
technique to bias the particles in space, energy, and angle. Like AVATAR, LIFT
also generated weight window parameters. However, generating a full
angle-dependent weight window map and running Monte Carlo transport with those
weight windows was computationally limiting, and the authors chose to only
generate space- and energy- dependent weight windows. Turner showed that LIFT
outperformed AVATAR for several example problems, but both methods performed
poorly in voids and low-density regions.

In an attempt to adapt CADIS and FW-CADIS to include angular information into
the variance reduction parameters,
Peplow et. al. formulated an adjustment to CADIS in the ORNL
code suite \cite{peplow_consistent_2012}. Two different
methods to generate weight windows and source biasing parameters
were investigated:
CADIS with directional source biasing, and CADIS without directional source
biasing. These methods were referred to as Simple Angular CADIS. Like AVATAR and
LIFT, Simple Angular CADIS approximated the angular flux as a product of the
scalar flux and an exponential. Like AVATAR, the angular flux values
were used to
generate angle-dependent weight windows but also consistently generated source
biasing parameters. For the method without
directional source biasing, the biased source distribution matched that of the
original CADIS, but the weight window values were directionally-dependent. The
method with directional source biasing used the transform function to obtain
directionally-dependent weight windows and directional source biasing.
Peplow and his colleagues found
that these methods generally increased the FOM by a factor of 1-5 as compared to
traditional CADIS, but in some
cases decreased the FOM. This was attributed to the P$_1$ type assumption used
to calculate the angular flux, which limited the physical applicability of the
method, as it did with AVATAR.

\subsection{Implementation and Performance of Angle-Informed Methods}
\label{sec:resultsangle}

In Mavric \cite{peplow_advanced_2007}

An internal feature of MCNP, the deterministic adjoint weight window generator (DAWWG), utilizes the discrete ordinates code PARTISN \cite{sweezy_automated_2005}.

ADVANTG \cite{mosher_new_2010, wagner_review_2011, bevill_new_2012}, which uses Denovo \cite{evans_denovo:_2010-1} and MCNP.

Tortilla \cite{somasundaram_implementation_2013}

With spent fuel storage containers \cite{chen_surface_2011}

IFSI site with MAVRIC \cite{sheu_dose_2011}


