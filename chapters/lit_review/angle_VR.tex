\section{Automated Angle-Informed Variance Reduction Methods}
\label{sec:AngleVR}

In a number of problems, the angular dependence of the flux is significant enough that
biasing by space and energy, but not angle, as detailed in the previous
sections, is not
sufficient. As a result, a subset of hybrid methods were developed to incorporate some
degree of the flux anisotropy in variance reduction parameters. Without explicitly
calculating the angular flux, which is memory- and storage- intensive, methods
attempted to approximate the angular flux using other information more readily
accessible to them.
Initial approaches
to angular biasing focused on approximating the angular flux $\psi$ as a separable
function of the scalar flux and an angle-dependent multiplier.
These approximations were then used to generate biasing
parameters dependent on angle for highly angular-dependent problems.

\subsection{Angular Biasing with Weight Windows}

\subsubsection{AVATAR}

The AVATAR \cite{van_riper_generation_1995, van_riper_avatarautomatic_1997}
(Automatic Variance and Time of Analysis Reduction) method generates
three-dimensional, space-, energy- and angle-dependent weight windows for Monte
Carlo by
utilizing a relatively course-mesh and few-angle deterministic calculation in
THREEDANT, approximating the angular flux as a function of the scalar flux, and
then subsequently passing those flux values through a postprocessing
code, Justine, to generate
weight windows for MCNP \cite{mcnp_manual_v1}. The AVATAR approach to
determining the angular flux used an approximation of the angular flux based on
the maximum entropy distribution, which will be briefly summarized in the next
few paragraphs.

\subsubsection*{Information Theory}

First, for a given set of discrete values $x_i (i=1,2, \cdots n)$
that are passed to a function, $f(x)$, the expectation value of that function is given
by
\begin{equation}
  \big\langle f(x) \big\rangle = \sum_{i=1}^n p_if(x_i).
\end{equation}
For the probability distribution $p_i = p(x_i)$, $(i=1,2, \cdots n)$ the entropy of
$p$ is defined as
\begin{equation}
  H(p) = - K \Sigma_i p_i \ln p_i .
  \label{eq:entropy}
\end{equation}
where K is a positive constant. A proof that this is indeed the associated
maximal entropy associated with all $p_i$ is given in
\cite{jaynes_information_1957}.
For a continuous probability density function $p(x)$ over the interval I, the
entropy of the continuous function is
\begin{equation}
  H(p) = - K \int_I\ p(x) \ln p(x) dx .
\end{equation}
To maximize either of these distributions, while also maintaining that $\Sigma
p_i = 1$, one can use Lagrangian multipliers $\lambda$ and $\mu$
\begin{equation}
  p_i = e^{-\lambda-\mu f(x_i)} .
\end{equation}
This can be solved using
\begin{subequations}
  \begin{equation}
    \big\langle f(x) \big\rangle = - \frac{\partial}{\partial \mu} \ln Z(\mu)
  \end{equation}
  \begin{equation}
    \lambda = \ln\big[Z(\mu)\big]
  \end{equation}
  where
  \begin{equation}
    Z(\mu) = \Sigma_i e^{-\mu f(x_i)} .
  \end{equation}
\end{subequations}

Jaynes \cite{jaynes_information_1957} showing that the maximum
entropy probability distribution function corresponding to the previous
equations is given by
\begin{equation}
  p_i= \exp \big[ - \big( \lambda_0 + \lambda_1f_1(x_1) + \cdots + \lambda_m f_m(x_i) \
    \big) \big] ,
\end{equation}
and the entropy of this distribution is given by
\begin{equation}
  S_{max} = \lambda_0 + \lambda_1 \big\langle f_1 (x) \big\rangle + \cdots + \
  \lambda_m \big\langle f_m(x) \big\rangle .
\end{equation}

In this case, the constant K from Eq. \eqref{eq:entropy} has been set to 1.
The maximum entropy approach to calculating a probability distribution function
is an attractive option given limited information about that
distribution. This method's power lies in that it
deduces a function given limited information, but does not place too great of an
importance on missing or unwarranted information. Furthermore, a distribution
ascertained from this methodology will encompass all distributions with smaller
entropies that satisfy the same constraints. Thus it provides the most widely
applicable probability distribution function for the system that has been
defined.

Moskalev showed that by using the maximum entropy approach, a distribution function
could be reconstructed from a (truncated) Legendre expansion
\cite{moskalev_reconstruction_1993}. This is particularly
applicable to radiation transport, because often scattering terms are truncated
Legendre expansions. In his application, Moskalev derived a generalized form of
reconstructing a probability distribution from a truncated expansion, where the
true function represented by a Legendre polynomial series
\begin{equation}
  f(L,\mu) = \sum_{l=0}^L -\frac{2l+1}{2} f_l P_l(\mu)
\end{equation}
could be associated with an adjusted function (obtained from maximizing the
entropy of the known values)
\begin{equation}
  \tilde{f}(L,\mu) = \exp \big( \sum_{l=0}^{L} \lambda_l P_l(\mu) \big)
\end{equation}
such that
\begin{equation}
  (f,P_l) = (\tilde{f}, P_l); \quad l = 0, \cdots, L .
\end{equation}
Where $\lambda_l$ are the Lagrange multipliers, $\tilde{f}$ and $f$ are
$\epsilon \phi$, and are assumed to be a function of $\mu$ such that $f(\mu)
\geq 0, \mu \epsilon [-1, 1]$.
These generalized equations were then applied to
group-to-group scattering probability distribution functions, as well as
reconstructing a $L=3$ function. The reconstruction showed agreement except near
the extrema of $\mu$.

Walters and Wareing \cite{walters_nonlinear_1994, walters_accurate_1996} showed
that the angle-dependent source definition for a discrete ordinates transport problem
could be calculated using Moskalev's approach
\cite{moskalev_reconstruction_1993}. In their method, they used this approach to
reconstruct the source distribution of particles in each cell from the source
moments. For standard methods, the source in a cell expanded in Legendre moments is
\begin{equation}
  S_m(x) = S_{m,j}\bigg[ P_0(x) + \frac{S_{m,j}^x}{S_{m,j}}P_1(x)\bigg]
\end{equation}
where $S_{m,j}$ is the average source in cell j, direction m, $S_{m,j}^x$ is the
$P_1(x)$ moment of the source and the $P_0$ and $P_1$ are the associated
Legendre polynomials. Using a normalized source distribution $s_m(x)$ where
\begin{equation*}
  S_m(x) = s_m(x) S_{m,j},
\end{equation*}
the normalized distribution is
\begin{equation}
  s_m(x) = \big[ s_0 + s_1 P_1(x)\big]
\end{equation}
where $s_0$ and $s_1$ are the zeroth and first Legendre moments of the source.
The source distribution derived from the maximum entropy distribution was found
to be
\begin{equation}
  \tilde{s}(x) = \frac{\lambda_{1,k}}{\sinh (\lambda_{1,j})} \
  e^{\lambda_{1,j}P_1(x)} .
\end{equation}
$\tilde{s}$ has normalized Legendre moments $s_0$ and $s_1$ that match
$s_m(x)$. Because $\tilde{s}$ satisfies the information that can be obtained
about $s_m$, it can be used to reconstruct $S_m(x)$ by way of
\begin{equation}
  S_m(x) = \tilde{s}_m(x)S_{m,j}.
\label{eq:Walt_reconstruct}
\end{equation}
$\lambda_{1,j}$ can be found from
\begin{equation}
  s_1 = 3\bigg[ \coth(\lambda_{1,j} - \frac{1}{\lambda_{1,j}})\bigg] .
  \label{eq:WW_lambda}
\end{equation}
It should be noted that the same methodology that Walters and Wareing use to reconstruct the source
distribution from the source moments can be used to reconstruct the angular flux
in cells based on moments of the angular flux (i.e. the scalar flux and
current) \cite{walters_nonlinear_1994}.

In their paper, Walters and Wareing \cite{walters_accurate_1996} suggested that in place of
solving Eq. \eqref{eq:WW_lambda} for $\lambda_{1,j}$, that a rational polynomial
be used in place of it to reduce computational time. The
suggested polynomial for $0 \leq \lambda_{1,j} \leq 5$ was:
\newcommand{\lamvar}{\dfrac{s_{1,j}}{3}}
\begin{equation}
  \lambda_{1,j} =  \dfrac{2.99821 (\lamvar) - 2.2669248 (\lamvar)^2 }{1 - 0.769332
             (\lamvar) -0.519928 (\lamvar)^2 + 0.2691594 (\lamvar)^3} \\
\end{equation}
and for $\lambda \geq 5$:
\begin{equation}
  \lambda_{1,j} = \frac{1}{1-\bar\mu}
\end{equation}
A full derivation of Eq. \eqref{eq:WW_lambda} and how it satisfies the maximum
entropy requirements can be found in Appendix A of \cite{walters_accurate_1996}.

In their application, Walters and Wareing found that this method was accurate
over a fairly course mesh for the problems analyzed, and the computed fluxes
remined positive over the solution space. When compared to other methods, this
approach performed much better over coarse meshes. However, the analysis was
limited to one-dimensional problems. As mesh size grew finer, the method
performed similarly to other methods. However, near vacuum boundary conditions
$\lambda_{1,j} \rightarrow \infty$ at the cell boundary, causing issues in
calculating the flux in these cells.

\subsubsection*{AVATAR Implementation}

AVATAR used a deterministic

AVATAR built off of the methodology described by Walters and Wareing
\cite{walters_nonlinear_1994, walters_accurate_1996}, but instead of
reconstructing the source distribution inside the cell, the maximum entropy
method was used to reconstruct the angular fluxes. Thus the angular flux, $\psi$
was reconstructed with the scalar flux, $\phi$ and the current, $J$.
AVATAR avoided generating explicit angular fluxes with
THREEDANT by assuming
that the adjoint angular flux is symmetric about the average adjoint current vector,
$J^{\dagger}$ :
\begin{subequations}
\label{avatareqns}
\begin{equation}
  \psi^{\dagger}(\hat\Omega) = \psi^{\dagger}(\hat\Omega\cdot n)
  \label{eq:Av1}
\end{equation}
where
\begin{equation}
  n = \frac{J^{\dagger}}{ \left\| J^{\dagger} \right\| }.
\end{equation}
Note that $n, J, \psi,$ and $\phi$ all have implied dependence on $(\vec{r},
E)$.
The angular flux could then be reconstructed assuming that the angular flux is a
product of the scalar flux and some angle-dependent function
\begin{equation}
  \psi^{\dagger}(\hat \Omega \cdot n) = \phi^{\dagger} f(\hat\Omega \cdot n).
  \label{eq:Av2}
\end{equation}
Note that Eq. \eqref{eq:Av2} takes the form of Eq. \eqref{eq:Walt_reconstruct}.
Thus f is derived from the maximum entropy distribution:
\begin{equation}
f(\hat\Omega \cdot n) = \frac{\lambda e^{(\hat\Omega \cdot n)\lambda}}
                             {2 \sinh {\lambda} }
\end{equation}
and $\lambda$ is a function of the average cosine $\bar\mu$
\begin{equation}
\begin{split}
\lambda  & = \frac{2.99821 \bar\mu - 2.2669248 {\bar\mu}^2 }{1 - 0.769332
             \bar\mu -0.519928 {\bar\mu}^2 + 0.2691594 {\bar\mu}^3} \\
         & = \frac{1}{1-\bar\mu} \\
  \label{eq:Av3}
\end{split}
\end{equation}
for $ 0 \le \bar\mu < 0.8001$ and $0.8001 \le \bar\mu < 1.0 $, respectively.
Also, $\mu$ is given by
\begin{equation}
\bar\mu(\vec{r} E) = \frac{\left\| J^{\dagger}(\vec{r} E)
                     \right\|}{\phi^{\dagger}(\vec{r} E)}.
  \label{eq:Av4}
\end{equation}
\label{eq:AvEqns}
\end{subequations}
Equations \eqref{eq:Av3} and \eqref{eq:Av4} are exact in both isotropic and
streaming conditions \cite{van_riper_generation_1995}.

Using the calculation of the angular flux described in Eqs. \eqref{eq:Av1}
through \eqref{eq:Av4}, angle-dependent weight windows could be constructed.
AVATAR's space- energy- and angle-dependent weight window was calculated by:
\begin{equation}
\bar {w} (\vec{r},E,\hat\Omega) = \frac{k}{\phi^{\dagger}(\vec{r},E)
                                  f(\hat\Omega \cdot n)}
\end{equation}
where k was a constant that could be adjusted to match the source distribution.
In the case of AVATAR, k was used as a normalization factor to ensure that
source particles are born with weights within the weight window.
AVATAR exclusively generated weight windows, and did not attempt to consistently
bias the source
distribution. Physically, the assumption behind AVATAR is that the adjoint
angular flux
is locally one-dimensional, so azimuthal symmetry is assumed.

\subsubsection*{AVATAR Results}

The authors of AVATAR showed that AVATAR's angularly-dependent
weight windows improved the FOM (from 5- to 7-times) for a multiple-tally well-logging
problem compared to the MCNP weight window generator. Additionally, because AVATAR
was not fully automated, the user had to have knowledge on the use of the $S_N$
deterministic solver in addition to the Monte Carlo methods they were trying to
optimize. As a result, the user needed to adequately prepare the deterministic
inputs, correctly specify the adjoint source for the deterministic solve, and
then pass these values to postprocessing software \cite{peplow_consistent_2012}.
%
%
%----------------------------------------------------------------------------------------

\subsubsection{Simple Angular CADIS}

Simple Angular CADIS \cite{peplow_consistent_2012} built off of existing
implementations
of CADIS and FW-CADIS by but attempted to incorporate angular information in the
methods without explicitly using angular flux solutions from the deterministic
solution. This was done by employing the method used by AVATAR and consistently
biasing the
source distribution with the weight windows (recall that the original
implementation of AVATAR did not have consistent source biasing).
In their work, Peplow et al. implemented their
work in MAVRIC [CODE REFERENCE] with two different approaches:
directionally-dependent weight windows with- and without directionally-dependent
source biasing.

\subsubsection*{Theory}

The Simple Angular CADIS approach includes the azimuthal distribution in its
approximation
of the angular flux, such that:
\begin{equation}
\psi^{\dagger}(\vec{r}, E, \hat \Omega) \cong \phi^{\dagger}(\vec{r}, E)
\frac{1}{2\pi} f(\hat\Omega \cdot n)
\end{equation}
and the angle-dependent weight windows are then given by:
\begin{equation}
\bar {w} (\vec{r},E,\hat\Omega) = \frac{2 \pi k}{\phi^{\dagger}(\vec{r},E)
                                  f(\hat\Omega \cdot n)}
\end{equation}

For the first method, with directionally-dependent weight windows and without
directional
source biasing, the biasing parameters are:
\begin{subequations}
\label{SA-CADIS1}
\begin{equation}
\begin{split}
\hat{q}(\vec{r},E,\hat\Omega) & = \frac{1}{R} q(\vec{r},E) \phi^{\dagger}
                                  (\vec{r},E) \frac{1}{2 \pi} q(\hat\Omega \cdot
                                  \hat d)\\
                              & = \hat{q}(\vec{r},E) \frac{1}{2 \pi}
                              q(\hat\Omega \cdot \hat d)
\end{split}
\end{equation}
\begin{equation}
\begin{split}
w_0  & = \frac{q(\vec{r},E,\hat\Omega)}{\hat{q}(\vec{r},E,\hat\Omega)} \\
     & = \frac{R}{\phi^{+}(\vec {r} ,E)}
\end{split}
\end{equation}
\begin{equation}
\begin{split}
\bar{w} (\vec{r},E,\hat\Omega)  & = \frac{R}{\phi^{\dagger}(\vec{r},E)}
                                    \frac{f(\hat{\Omega_0}\cdot n(\vec{r_0}, E_0))}
                                    {f(\hat\Omega \cdot n)} \\
                                & = \bar{w}(\vec{r},E) \frac{f(\hat{\Omega_0}\cdot
                                    n(\vec{r_0}, E_0))}{f(\hat\Omega \cdot n)}
\end{split}
\end{equation}
\end{subequations}
Note that the biased source distribution, $\hat{q}(\vec{r},E,\hat\Omega)$, is a
function of
the biased source distribution from traditional space- energy- CADIS and of the
original
directional source distribution; which is why this method has directional weight
windows,
but not directional source biasing.

For the second method, with directionally-dependent weight windows and with
directional
source biasing, the biasing parameters are given by:

\begin{subequations}
\label{SA-CADIS2}
\begin{equation}
\begin{split}
\hat{q}(\vec{r},E,\hat\Omega) & = \frac{1}{Rc} q(\vec{r},E,\hat\Omega)
                                  \phi^{\dagger}(\vec{r},E,\hat\Omega) \\
                              & = \left[ \frac{1}{R}q(\vec{r},E)
                                \phi^{\dagger}(\vec{r},E)
                                  \right] \left[\frac{1}{c}\frac{1}{2 \pi}
                                    q(\hat\Omega
                                  \cdot \hat d) \frac{1}{2 \pi}
                                  f (\hat\Omega \cdot n_0)\right]  \\
                             & = \hat{q}(\vec{r},E) \left[\frac{1}{c}\frac{1}{2 \pi}
                                 q(\hat\Omega \cdot \hat d) \frac{1}{2 \pi}
                                 f(\hat\Omega \cdot n_0) \right]
\end{split}
\end{equation}
\begin{equation}
c = \int{\frac{1}{2 \pi} q(\hat\Omega \cdot \hat d) \frac{1}{2 \pi}
    f (\hat\Omega \cdot n_0} d\hat\Omega
\end{equation}
\begin{equation}
\begin{split}
w_0  &= \frac{q(\vec{r},E,\hat\Omega)}{\hat{q}(\vec{r},E,\hat\Omega)} \\
     &= \frac{R}{\phi^{+}(\vec {r} ,E)} \frac{2 \pi c}{f(\hat\Omega \cdot n_0)}
\end{split}
\end{equation}
\begin{equation}
\begin{split}
\bar{w} (\vec{r},E,\hat\Omega)  &= \frac{R}{\phi^{\dagger}(\vec{r},E)}
                                   \frac{2 \pi c}{f(\hat\Omega \cdot n_0)} \\
                                &= \bar{w}(\vec{r},E) \frac{2 \pi
                                    c}{f(\hat\Omega \cdot n)}
\end{split}
\end{equation}
\end{subequations}

\subsubsection*{Results}

To test these two modifications of CADIS, the authors ran a number of test
problems and
compared them against standard implementations of CADIS and analog Monte Carlo
runs. Simple
angular CADIS with- and without- source biasing showed improvement for some
problems, but not
others. Notably, the modified CADIS methods performed well in a simple duct
streaming problem,
but for many problems did not sufficiently increase the FOM to compensate for
the increase in
calculation time to generate the variance reduction parameters.

%
%
%----------------------------------------------------------------------------------------

\subsection{Angular Biasing Using the Exponential Transform}

In 1985, Henricks and Carter \cite{hendricks_anisotropic_1985} described a
method by which
photon transport could be biased in angle with an exponential transform
adjustment factor.
In this study, the authors performed studies on three test problems with the
exponential
transform adjustment factor and with a synergistic angular bias and exponential
transform
adjustment.
In all studies, the synergistic biasing outperformed the exponential transform
adjustment
alone.
However, their method performed best in highly absorbing media.
The authors noted that this performance was due to the fact that the biasing
could be made
strong without undersampling scattering in the problem.
They also pointed out that while the weight window method was comparable in
efficiency to the
method described, their method avoided the choice of choosing importances and
weight window
values for biasing. Their method was limited to exclusively photon transport
biasing, and not
neutron transport; the authors were optimistic that the method could be extended
to neutron
transport.

The LIFT \cite{turner_automatic_1997} method, developed by Turner and Larsen, is
a modification
of the zero variance solution that utilizes a deterministic calculation to
generate biasing
parameters for the exponential transform and weight window variance reduction
techniques.
First, the adjoint angular flux is approximated, piecewise continuous in space
in angle,
with the following equations:

\begin{subequations}
\label{LIFT}
\begin{equation}
\psi^{\dagger}_{g,n} (\vec{r},\Omega) \approx
                \phi^{\dagger}_{g,n}V_n \left[ \beta_{g,n}
                \frac{\sigma_{s_0,g \rightarrow g,n} b_{g,n}(\Omega)}{\sigma_{t,g,n} -
                \rho_{g,n}\cdot \Omega} e^{\rho_{g,n}\cdot (r-r_n)} \right]
\end{equation}
where the physical system is comprised of $V_n$ regions, and
$\psi^{\dagger}_{g,n}$ is the
approximation of the angular flux for group g and region n. Further, $\beta$,
the normalization
factor is given by:
\begin{equation}
\beta_{g,n} = \frac{1}{\int_{V_n} e^{\rho_{g,n} \cdot (r-r_n)} dr \int_{4\pi}
              \frac{\sigma_{s_0,g \rightarrow g,n} b_{g,n}(\Omega)}{\sigma_{t,g,n} -
              \rho_{g,n}\cdot \Omega} d\Omega}
\end{equation}
b, the linearly anisotropic factor, is
\begin{equation}
b_{g,n}(\hat\Omega) = 1 + 3\mu_{g\rightarrow g,n}\frac{\sigma_{t,g,n} -
                      \sigma_{s_0,g\rightarrow g,n}}{{\left| \rho_{g,n}
                      \right|}^2}\rho_{g,n}\cdot \Omega
\end{equation}
and the biasing parameter is given by
\begin{equation}
  \rho_{g,n} = \sigma_{t,g,n} \lambda_{g,n}
\end{equation}
\end{subequations}

The first equation in this series is an adjustment of the exponential transform.
However,
rather than relying upon an isotropic scattering law, like earlier
implementations, the LIFT
method adjusts the transform to instead be linearly anisotropic in angle. The
parameters
$\beta_{g,n}$, $b_{g,n}$ and $rho_{g,n}$ are calculated from values obtained from the
deterministic calculation and are used to calculate $\psi^{\dagger}_{g,n}$.

In addition to utilizing the exponential transform to bias the particles in
angle, the LIFT
method also utilized weight windows for particle weight adjustment. However, the
computational
cost of generating angle-dependent weight windows from the previous equations
led the authors
to choose space-energy exclusive weight windows. The weight window target values were
calculated to be inversely proportional to the adjoint solution, as with other
methods.
\begin{equation}
  ww_{center,g,n} = \frac{\phi^{\dagger}_{g,src}}{\phi^{\dagger}_{g,n}}
\end{equation}

Turner compared a number of variants of LIFT \cite{turner_automatic_1997-1}
against AVATAR to
determine the efficiency of LIFT. In his investigation, Turner compared LIFT and
AVATAR using
approximations for the adjoint solution with diffusion and $S_N$ transport
calculations, and
various methods to calculate weight window paramaters, including utilizing LIFT
combined
with AVATAR's weight window parameters. In most cases, LIFT outperformed AVATAR.
In problems
with voids and low-density regions, the efficiency of the LIFT method decreased,
but so did
AVATAR. However, an important note that Turner mentioned was that a while
increasing the
accuracy of the deterministic solution may decrease the variance, it is not
necessarily the
best for the FOM. This is a valuable lesson for all automated variance reduction
methods: an
overly accurate solution for the adjoint problem may reduce the variance but
come at such
a high computational cost that it will decrease the FOM.

[Kendra's addition to LIFT]

Cooper and Larsen, in addition to generating global isotropic weight windows, as
described
in the previous section, also developed angle-dependent weight windows
\cite{cooper_automated_2001}. Here, the forward angular flux is calculated in a
similar
manner as the AVATAR method:
\begin{subequations}
\label{LIFT}
\begin{equation}
\psi(\vec{r},\hat\Omega) \approx A(\vec{r})e^{\vec{B}_r \cdot \hat\Omega}
\end{equation}
where $A(\vec{r})$ and $\vec{B}(\vec{r})$ are given by:
\begin{equation}
A(\vec{r}) = \frac{\phi(\vec{r})}{4 \pi}\frac{B(\vec{r})}{sinh B(\vec{r})}
\end{equation}
\begin{equation}
\vec{B}(\vec{r}) = B(\vec{r}) \frac{\vec{\lambda}}{\left| \vec{\lambda} \right|}
\end{equation}
B can be calculated with
\begin{equation}
\lambda(\vec{r}) = coth B(\vec{r}) - \frac{1}{B(\vec{r})}
\end{equation}
Cooper noted that $\lambda(\vec{r})$ could be estimated with either the scalar fluxes
and currents from the quasidiffusion calculation, or with the scalar fluxes and
currents
from the Monte Carlo solution. Cooper noted that because Monte Carlo robustly
calculated
these values, it was the more optimal choice. Knowing how to calculate these
parameters
from the deterministic calculation, the angular weight window can be calculated with:
\begin{equation}
ww_{i,j} (\hat\Omega)= \frac{\psi_{i,j}(\vec{r},\hat\Omega)} {max\phi_{i',j'}/4\pi}
\end{equation}
\end{subequations}

As mentioned in the previous section, Cooper's method was limited in that it used an
iterative quasidiffusion / Monte Carlo solution to generate the biasing parameters for
the problem. This method was not automated; and the ideal frequency between iterations
was never explored. However, Cooper showed that the angularly dependent weight windows
significantly improved the figure of merit as compared to analog Monte Carlo.
Further, the
angular weight windows performed slightly better than the isotropic weight windows in
evenly distributing the problems in problems with anisotropy.
MCNP Weight Window Generator \textbf{cite MCNP manual here}



\subsection{Implementation and Performance of Angle-Informed Methods}
\label{sec:resultsangle}

In Mavric \cite{peplow_advanced_2007}

An internal feature of MCNP, the deterministic adjoint weight window generator
(DAWWG), utilizes the discrete ordinates code PARTISN
\cite{sweezy_automated_2005}.

ADVANTG \cite{mosher_new_2010, wagner_review_2011, bevill_new_2012}, which uses
Denovo \cite{evans_denovo:_2010-1} and MCNP.

Tortilla \cite{somasundaram_implementation_2013}

With spent fuel storage containers \cite{chen_surface_2011}

IFSI site with MAVRIC \cite{sheu_dose_2011}


