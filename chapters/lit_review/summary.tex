\section{Variance Reduction in Large Application Problems}
\label{sec:litsummary}
%not exactly sure what to call this last section still. I need to summarize
%results and also summarize the literature. Hmmm.

Variance reduction methods exist for Monte Carlo methods to
achieve a more accurate answer in a
shorter amount of time. Automated variance reduction methods have been designed
to aid users in generating variance reduction parameters where it might not be
intuitive or obvious what variance reduction parameters are best for a problem.
The most successful variance reduction methods construct or estimate an
importance function for the desired response from a preliminary calculation.
This importance function may be derived from the adjoint solution to the
transport equation, or it may be derived from contributon theory.

The methods described in Sections \ref{sec:localVR} through \ref{sec:AngleVR}
have been implemented and tested in a number of software packages. The problem
spaces over which they have been applied is extensive, and show that a large
subset of application problems can be successfully simulated with the assistance
of existing variance reduction techniques. Local variance reduction methods can
be used to reduce the variance in source-detector problems where the detector
constitutes a small subset of the problem phase-space. Global variance reduction
methods can be used to distribute response sampling equally throughout several
tallies
or a problem-wide tally. Angle-based variance reduction methods are used in
problems where space- and energy- variance reduction methods alone are not
sufficient. For large and complex problems, automated versions of each of these
methods are required as the user expertise to obtain even remotely adequate
parameters is significant. Here, the existing state of automated variance
reduction methods and the applications on which they have been tested will be
summarized.

Presently, numerous hybrid methods packages that use the methods described in
the preceeding sections are available. These packages are targeted towards
deep-penetration radiation transport and shielding applications.
The CADIS and FW-CADIS methods are distributed with MAVRIC \cite{SCALE6_1,
peplow_advanced_2007} and
ADVANTG \cite{mosher_automated_2009} from Oak Ridge National Laboratory (ORNL),
which
use the discrete ordinates code Denovo \cite{evans_denovo:_2010} to make VR
parameters for the Monte Carlo codes
Monaco \cite{SCALE6_1} and MCNP\cite{mcnp_manual_v1},
respectively. CADIS and FW-CADIS are also available in Tortilla
\cite{somasundaram_implementation_2013},
which uses the hybrid methods software using the deterministic code Attilla
\cite{lucas_applications_2004}. Tortilla also includes a version of LIFT and
LIFT-based weight windows. The Deterministic Adjoint Weight Window
Generator (DAWWG) from Los Alamos National Laboratory (LANL)
\cite{sweezy_automated_2005} uses the adjoint solution
from a deterministic solve in
PARTISN \cite{alcouffe_partisn_2002}
to generate biasing parameters for MCNP, and also includes AVATAR
functionality. MCNP \cite{mcnp_manual_v1} is distributed with a weight window
generator (WWG) that uses a preliminary Monte Carlo solution to estimate an
importance function for the problem. Though this list is not exhaustive, it
illustrates the present ubiquity and need for hybrid methods to analyze
realistic problems. In the analysis of realistic problems, ensuring that a
``good'' answer is achieved is necessary for safety and security. In the next
few paragraphs, how and how effectively each of these methods have been
applied to application problems
is summarized.
The degree to which each is successful is also discussed.

CADIS and FW-CADIS have been used for a number of studies of spent fuel storage
facilities.
Radescleu et al.\ used FW-CADIS in MAVRIC to calculate spent fuel
dose rates of a single dry cask
with finely detailed geometry and spent fuel isotopic compositions
\cite{radulescu_dose_2013}.
Chen et al.\ used MAVRIC \cite{SCALE6_1} to analyze dose rates on spent fuel
storage containers \cite{chen_surface_2011}. The fueled region of the storage
container was homogenized into an effective fuel region.
They found that in a coarse energy
group calculation (27G19N) MAVRIC underestimated neutron dose rates at high
energies. However, MAVRIC's ability to generate importances in three dimensions
allowed it to have better problem-wide results, while the compared to methods
(SAS4) struggled generating satisfactory results in the axial direction. This
was demonstrated to a greater extent in an analysis of an independent spent
nuclear fuel storage installation (ISFSI) \cite{sheu_dose_2011} by Sheu et al.
The FOM
achieved by MAVRIC appeared inferior to those obtained with SAS4 or TORT/MCNP in
a single cask. However, when applied to a storage bed of 30 casks MAVRIC was able to
generate VR parameters at all, which were unfeasible for the other two methods. These
studies demonstrated that CADIS and FW-CADIS are desirable methods for which to
obtain global and three-dimensional variance reduction parameters for realistic
problems.

ADVANTG \cite{mosher_automated_2009},
developed at ORNL \cite{mosher_new_2010,
wagner_review_2011, bevill_new_2012} is a hybrid methods package
for automated variance reduction of
the Monte Carlo transport package, MCNP \cite{mcnp_manual_v2}.
ADVANTG uses the deterministic
transport code Denovo \cite{evans_denovo:_2010} to perform the forward and
adjoint calculations for CADIS and FW-CADIS. At its inception, ADVANTG was used
to analyze various threat-detection nonproliferation problems
\cite{mosher_automated_2009}. FOM improvements on the order of $10^2$
to $10^4$ when compared with analog Monte Carlo have been observed.
However, Mosher et al.\ noted
that the methods struggled with problems exhibiting strongly anisotropic
behavior. In particular, they noted that low-density materials and strongly
directional sources posed issues. This indicated that while CADIS and FW-CADIS
are very useful methods, they have limitations in highly angle-dependent
applications.

The deterministic adjoint weight window generator
(DAWWG), utilizes the discrete ordinates code PARTISN
\cite{sweezy_automated_2005} to generate space- energy- and angle-dependent
weight windows. It is an internal feature of MCNP.
The angle-dependent weight windows are calculated with the same
methodology as AVATAR \cite{sweezy_automated_2005, van_riper_generation_1995}.
Sweezy and colleagues compared
DAWWG to the standard MCNP WWG on an oil well logging problem, a shielding
problem, and a dogleg neutron void problem. The deterministic weight window
generator obtained similar relative errors as the standard WWG for the first two problems,
but in a fraction of the time. However, for the dogleg void problem, which
exhibited strong angular dependence in the neutron flux,
the authors noted that DAWWG was not as effective as the
standard MCNP WWG. This was attributed to ray effects from the $S_N$ transport
influencing the weight windows obtained by DAWWG, which is not an issue for the
standard WWG.

A variety of automated variance reduction methods, including CADIS and LIFT have
been implemented into the Attila / Tortilla deterministic and hybrid transport
code packages \cite{somasundaram_implementation_2013}. These methods
were used on several
nonproliferation test problems. For the most part, LIFT and LIFT combined with
weight windows outperformed CADIS' weight windows and source biasing, indicating
that the addition of angular information was of benefit for these more realistic
nonproliferation application
problems.

Peplow et. al.\ formulated an adjustment to CADIS in the ORNL
code suite \cite{peplow_consistent_2012} to incorporate angular information into
the VR parameters (see Section \ref{sec:simpAngCADIS}.
Two different
methods to generate weight windows and source biasing parameters
were investigated:
CADIS with directional source biasing, and CADIS without directional source
biasing.
For the method without
directional source biasing, the biased source distribution matched that of the
original CADIS, but the weight window values were directionally-dependent. The
method with directional source biasing used the transform function to obtain
directionally-dependent weight windows and directional source biasing.
Peplow and his colleagues found
that these methods generally increased the FOM by a factor of 1-5 as compared to
traditional CADIS, but in some
cases decreased the FOM. This was attributed to the P$_1$ approximation used
to calculate the angular flux, which limited the physical applicability of the
method, just as with AVATAR.

CADIS and FW-CADIS have shown to be the existing ``gold standard'' of local and
global variance reduction methods for large application problems, a selection
of which were described in the preceding paragraphs.
These problems include active interrogation of cargo containers
\cite{mosher_automated_2009}, spent fuel storage casks \cite{chen_surface_2011,
radulescu_dose_2013}
and beds \cite{sheu_dose_2011}, and other nonproliferation and shielding
applications
\cite{somasundaram_implementation_2013}. For additional
applications, one may refer to \cite{wagner_review_2011}.
In some of these application problems, the
parameters generated by CADIS or FW-CADIS were sufficient for the problem
application. However, for other problems that had
strong angular dependence or geometric
complexity, the parameters were insufficient \cite{chen_surface_2011,
somasundaram_implementation_2013, peplow_consistent_2012}.
This can be remedied with additional angular
information in the variance reduction parameters, such as LIFT
\cite{somasundaram_implementation_2013}, but the benefits of consistent source
biasing are lost in this case. Alternatively, the angular flux can be
reconstructed in a manner similar to AVATAR
\cite{sweezy_automated_2005, peplow_consistent_2012} to generate angle-dependent
weight windows, but this approximates the angular flux to be linearly
anisotropic in angle
(from the $P_1$ reconstruction), and is also dependent on
the deterministic flux not having ray effects \cite{sweezy_automated_2005}.

Although numerous methods have been
proposed and implemented to obtain adequate angle-informed variance reduction
parameters for application problems, they have limited applicability,
and determining in which problems they will be
useful is not always straightforward. No single method has been successful
for problems with all types of anisotropy, and no existing angle-informed method
captures the anisotropy in the flux without significant approximation. For
large-scale, highly anisotropic, deep-penetration radiation transport problems,
there exists a need for improvements in hybrid methods.

