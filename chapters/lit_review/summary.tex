\section*{Summary of the Literature}
\label{sec:litsummary}

The AVATAR method
\cite{van_riper_generation_1995, van_riper_avatarautomatic_1997} used an
approximation of the angular flux--without explicitly calculating it--to
generate angle-dependent weight windows. It operated under the approximation
that
the angular flux was separable and symmetric about the average current vector.
The angular flux was then calculated using
a product of a deterministically-obtained
scalar flux and an exponential function, derived from the
maximum entropy distribution, that was a function of the scalar flux and the
current. Space-, energy-, and angle-dependent weight windows for
the Monte Carlo problem were then generated from the inverse of the angular
flux. AVATAR improved the FOM for sample problems from 2 to 5 times, but did not
apply to problems where the flux was not azimuthally symmetric.

The LIFT method \cite{turner_automatic_1997, turner_automatic_1997-1}, like
AVATAR, calculated the angular flux for a region by assuming the angular flux
was a product of the scalar flux and an exponential function. The angular flux
values were then used to generate values for the exponential transform variance
reduction
technique to bias the particles in space, energy, and angle. Like AVATAR, LIFT
also generated weight window parameters. However, generating a full
angle-dependent weight window map and running Monte Carlo transport with those
weight windows was computationally limiting, and the authors chose to only
generate space- and energy- dependent weight windows. Turner showed that LIFT
outperformed AVATAR for several example problems, but both methods performed
poorly in voids and low-density regions.

In an attempt to adapt CADIS and FW-CADIS to include angular information into
the variance reduction parameters,
Peplow et. al. formulated an adjustment to CADIS in the ORNL
code suite \cite{peplow_consistent_2012}. Two different
methods to generate weight windows and source biasing parameters
were investigated:
CADIS with directional source biasing, and CADIS without directional source
biasing. These methods were referred to as Simple Angular CADIS. Like AVATAR and
LIFT, Simple Angular CADIS approximated the angular flux as a product of the
scalar flux and an exponential. Like AVATAR, the angular flux values
were used to
generate angle-dependent weight windows but also consistently generated source
biasing parameters. For the method without
directional source biasing, the biased source distribution matched that of the
original CADIS, but the weight window values were directionally-dependent. The
method with directional source biasing used the transform function to obtain
directionally-dependent weight windows and directional source biasing.
Peplow and his colleagues found
that these methods generally increased the FOM by a factor of 1-5 as compared to
traditional CADIS, but in some
cases decreased the FOM. This was attributed to the P$_1$ type assumption used
to calculate the angular flux, which limited the physical applicability of the
method, as it did with AVATAR.

