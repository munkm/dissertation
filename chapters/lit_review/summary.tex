\section{Summary}
\label{sec:litsummary}

The methods described in Sections \ref{sec:localVR} through \ref{sec:AngleVR}
have been implemented and tested in a number of software packages. The problem
space over which they have been tested is extensive, and shows that a large
subset of application problems can be successfully simulated with the assistance
of existing variance reduction techniques. Local variance reduction methods can
be used to reduce the variance in source-detector problems where the detector
constitutes a small subset of the problem phase-space. Global variance reduction
methods can be used to distribute response sampling equally throughout several-
or a problem-wide tally. Angle-based variance reduction methods are used in
problems where space- and energy- variance reduction methods alone are not
sufficient. For large and complex problems, automated versions of each of these
methods are required, as the user expertise to obtain even remotely adequate
parameters is significant. Here the existing state of automated variance
reduction methods and the applications on which they have been tested will be
summarized.

Numerous hybrid methods packages for radiation shielding exist.
The CADIS and FW-CADIS are distributed with Mavric \cite{SCALE6_1,
peplow_advanced_2007} and
ADVANTG \cite{mosher_automated_2009} from Oak Ridge National Laboratry, and Tortilla
\cite{somasundaram_implementation_2013},
the hybrid methods software using the deterministic code Attilla
\cite{lucas_applications_2004}. Tortilla also includes a version of LIFT and
LIFT-based weight windows as a variance reduction generator.
The Deterministic Adjoint Weight Window
Generator (DAWWG) from Los Alamos National Laboratory
\cite{sweezy_automated_2005} uses the adjoint solution
from a deterministic solve in
PARTISN \cite{alcouffe_partisn_2002}
to generate biasing parameters for MCNP, and also includes AVATAR
functionality.

An internal feature of MCNP, the deterministic adjoint weight window generator
(DAWWG), utilizes the discrete ordinates code PARTISN
\cite{sweezy_automated_2005} to generate space- energy- and angle- dependent
weight windows. The angle-dependent weight windows are done with the same
methodology as AVATAR \cite{sweezy_automated_2005, van_riper_generation_1995}.
In their work, Sweezy and colleagues compared
DAWWG to the standard MCNP WWG on an oil well logging problem, a shielding
problem, and a dogleg neutron void problem. The deterministic weight window
generator obtained similar FOMS as the standard WWG for the first two problems,
but in a fraction of the time. However, for the dogleg void problem, which
exhibited strong angular dependence in the neutron flux,
the authors noted that DAWWG was not as effective as the
standard MCNP WWG. This was attributed to ray effects from the $S_N$ transport
influencing the weight windows obtained by DAWWG, which is not an issue for the
standard WWG.

ADVANTG \cite{mosher_automated_2009}
has been and is currently actively
developed at Oak Ridge National Laboratory \cite{mosher_new_2010,
wagner_review_2011, bevill_new_2012} for automated variance reduction of
the Monte Carlo transport package, MCNP \cite{mcnp_manual_v2}.
ADVANTG uses the deterministic
transport code Denovo \cite{evans_denovo:_2010-1} to perform the forward and
adjoint calculations for CADIS and FW-CADIS. At its inception, ADVANTG was used
to analyze various threat-detection nonproliferation problems
\cite{mosher_automated_2009}, and FOM improvements were on the order of $10^2$
to $10^4$ when compared with analog Monte Carlo. However, Mosher et al. noted
that the methods struggled with problems exhibiting strongly anisotropic
behavior. In particular, they noted that low-density materials and strongly
directional sources posed issues.

A variety of automated variance reduction methods, including CADIS and LIFT have
been implemented into the Attila / Tortilla deterministic and hybrid transport
code packages \cite{somasundaram_implementation_2013}. These were used on several
nonproliferation test problems. For the most part, LIFT and LIFT combined with
weight windows outperformed CADIS' weight windows and source biasing, indicating
that the addition of angular information was of benefit for these more realistic
nonproliferation application
problems.

CADIS and FW-CADIS have been used for a number of studies on spent fuel storage
facilities.
Radescleu et al. used FW-CADIS in MAVRIC to calculate spent fuel
dose rates of a single dry cask
with finely detailed geometry and spent fuel isotopic compositions
\cite{radulescu_dose_2013}.
Chen et al. used MAVRIC \cite{SCALE6_1} to analyze dose rates on spent fuel
storage containers \cite{chen_surface_2011}. The fueled region of the storage
container was homogenized into an effective fuel region.
They found that in a coarse energy
group calculation (27G19N) MAVRIC underestimated neutron dose rates at high
energies. However, its ability to generate importances in three dimensions
allowed it to have better problem-wide results, while the compared methods
(SAS4) struggled generating satisfactory results in the axial direction. This
was demonstrated to a greater extent in an analysis of an independent spent
nuclear fuel storage installation (ISFSI) \cite{sheu_dose_2011} by Sheu et al..
While the FOM
achieved by MAVRIC appeared inferior to those obtained with SAS4 or TORT/MCNP in
a single cask, when applied to a storage bed of 30 casks MAVRIC was able to
generate VR parameters which was unfeasible for the other two methods. These
studies demonstrated that CADIS and FW-CADIS are desirable methods for which to
obtain global and three-dimensional variance reduction parameters for realistic
problems.

The AVATAR method
\cite{van_riper_generation_1995, van_riper_avatarautomatic_1997} used an
approximation of the angular flux--without explicitly calculating it--to
generate angle-dependent weight windows. It operated under the approximation
that
the angular flux was separable and symmetric about the average current vector.
The angular flux was then calculated using
a product of a deterministically-obtained
scalar flux and an exponential function, derived from the
maximum entropy distribution, that was a function of the scalar flux and the
current. Space-, energy-, and angle-dependent weight windows for
the Monte Carlo problem were then generated from the inverse of the angular
flux. AVATAR improved the FOM for sample problems from 2 to 5 times, but did not
apply to problems where the flux was not azimuthally symmetric.

The LIFT method \cite{turner_automatic_1997, turner_automatic_1997-1}, like
AVATAR, calculated the angular flux for a region by assuming the angular flux
was a product of the scalar flux and an exponential function. The angular flux
values were then used to generate values for the exponential transform variance
reduction
technique to bias the particles in space, energy, and angle. Like AVATAR, LIFT
also generated weight window parameters. However, generating a full
angle-dependent weight window map and running Monte Carlo transport with those
weight windows was computationally limiting, and the authors chose to only
generate space- and energy- dependent weight windows. Turner showed that LIFT
outperformed AVATAR for several example problems, but both methods performed
poorly in voids and low-density regions.

In an attempt to adapt CADIS and FW-CADIS to include angular information into
the variance reduction parameters,
Peplow et. al. formulated an adjustment to CADIS in the ORNL
code suite \cite{peplow_consistent_2012}. Two different
methods to generate weight windows and source biasing parameters
were investigated:
CADIS with directional source biasing, and CADIS without directional source
biasing. These methods were referred to as Simple Angular CADIS. Like AVATAR and
LIFT, Simple Angular CADIS approximated the angular flux as a product of the
scalar flux and an exponential. Like AVATAR, the angular flux values
were used to
generate angle-dependent weight windows but also consistently generated source
biasing parameters. For the method without
directional source biasing, the biased source distribution matched that of the
original CADIS, but the weight window values were directionally-dependent. The
method with directional source biasing used the transform function to obtain
directionally-dependent weight windows and directional source biasing.
Peplow and his colleagues found
that these methods generally increased the FOM by a factor of 1-5 as compared to
traditional CADIS, but in some
cases decreased the FOM. This was attributed to the P$_1$ type assumption used
to calculate the angular flux, which limited the physical applicability of the
method, as it did with AVATAR.

CADIS and FW-CADIS have shown to be the existing ``gold standard'' of local- and
global- variance reduction methods for large application problems.
These problems include active interrogation of cargo containers
\cite{mosher_automated_2009}, spent fuel storage casks \cite{chen_surface_2011,
radulescu_dose_2013}
and beds \cite{sheu_dose_2011}, and other nonproliferation applications
\cite{somasundaram_implementation_2013}. In some of these problems, the
parameters generated by CADIS or FW-CADIS were sufficient for the problem
application. Other problems that had strong angular dependence or geometric
complexity the parameters were insufficient \cite{chen_surface_2011,
somasundaram_implementation_2013}. This can be remedied with additional angular
information in the variance reduction parameters, such as LIFT
\cite{somasundaram_implementation_2013}, but the benefits of consistent source
biasing are lost in this case. Alternatively, the angular flux can be
reconstructed in a manner similar to AVATAR
\cite{sweezy_automated_2005, peplow_consistent_2012} to generate angle-dependent
weight windows, but this approximates the angular flux to be linearly
anisotropic in angle (from the $P_1$ reconstruction), and is also dependent on
the deterministic calculation not having apparitions in the flux from ray
effects \cite{sweezy_automated_2005}. Although numerous remedies have been
proposed and implemented to obtain adequate angle-informed variance reduction
parameters for application problems, they have limited applicability
\cite{peplow_consistent_2012} and determining in which problems they will be
useful is not always straightforward. No single method has been successful
for problems with all types of anisotropy, and no existing angle-informed method
captures the anisotropy in the flux without significant approximation. For
large-scale, highly anisotropic, deep-penetration radiation transport problems,
there exists a need for expansion in existing hybrid methods.

