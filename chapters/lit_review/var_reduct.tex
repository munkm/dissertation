\section{Monte Carlo Variance Reduction}
\label{sec:MCvar}

Monte Carlo methods for radiation transport are used in the nuclear
engineering community for a wide spectrum of application problems. Without any
variance reduction techniques, Monte Carlo methods aim to emulate
the transport of a particle from birth, through physical interaction, to death
by randomly sampling probabilities particle production, elastic and inelastic
scattering, and absorption for that particle.
This process of transporting a single particle
is repeated millions of times, which is analogous to transporting
millions of particles throughout the problem. When the user achieves a
sufficient
number of particles to sample to reach the desired statistical precision for
the region of interest, the
simulation will be complete. However, this naive approach to simulating each
particle, no matter whether it is likely to contribute to the tallied result,
can be extraordinarily computationally inefficient depending on the problem. A
user could waste time simulating millions of unusable particles and still not
reach the desired statistical precision for the tally. Variance
reduction techniques were developed to address this issue. In general, these
techniques bias the Monte Carlo transport to more effectively
contribute to a particular result, while not fundamentally changing the nature
of the problem being solved.

\subsection{Statistical Background}
\label{subsec:StatBkgnd}

Variance reduction techniques are rooted in statistics, so we will begin our
discussion of variance reduction techniques with a brief primer on the
statistical background relevant to Monte Carlo radiation transport. Monte Carlo
methods transports many randomly sampled
particles, and when those particles reach a region of interest, they are scored
in a tally. The statistical precision of the tally
will reflect the total number of particles that were sampled in- or at- this
region or surface.
The reliability of the answer obtained in this region is then dependent
on the quantity of these particles,
and the amount of time taken to move the particles on
their respective random walks through space to the region.

\subsubsection{Population Statistics}

Estimate of the variance:
\begin{equation}
S^{ 2 }=\sum _{ i=1 }^{ N }{ \frac { (x_{ i }-\bar { x } )^{ 2 } }{ N-1 }  }
             \cong \bar{x^2}-\bar{x}^2
\label{eq:Var}
\end{equation}

The variance of the mean value, or variance of $\bar{x}$, follows as:
\begin{equation}
S^{ 2 }_{ \bar { x }  }=\frac{S^2}{N}
\label{eq:VarMean}
\end{equation}

From \ref{eq:VarMean}, one can see that relationship between variance and N follows:
\begin{equation}
S_{ \bar { x }  }=\sqrt { \frac { S^{ 2 } }{ N }  } =\frac { S }{ \sqrt { N }  }
\label{eq:VarN}
\end{equation}

The relative error normalizes the variance by the mean value.
\begin{equation}
R = \frac{S_{ \bar { x }  }}{\bar{x}}
\label{eq:RelativeErr}
\end{equation}

S to R relation
\begin{equation}
S^2\:\propto\: R^2\:\propto\:\frac{1}{N}
\label{eq:S to R}
\end{equation}

\subsubsection{The Central Limit Theorem}

For
\textbf{describe the central limit theorem and how it relates to Monte Carlo
statistics here }

\subsubsection{Other Relevant Monte Carlo Factors}

The figure of merit (FOM) is often used as a success metric for variance
reduction techniques.
This is because
\begin{equation}
FOM=\frac { 1 }{ R^{ 2 }T }
\label{eq:FOM}
\end{equation}

\subsection{Variance Reduction Methods for Monte Carlo Radiation Transport}
\label{subsec:MCVR}
MCNP \cite{hendricks_mcnp_1985, brown_mcnp_2002} has a number of techniques for
variance
reduction that are accessible to users. Overall, these variance reduction
techniques fall
into four general categories: truncation methods, population control methods, modified
sampling methods, and partially-deterministic methods. Of importance for this
project are
population control methods and modified sampling methods, which are discussed in
a number
of the papers referenced herein. Truncation methods and partially-deterministic
methods are also very useful, but do not contribute to and
are not the focus of this work, so will only be touched upon briefly.

Note that while this
discussion will tend to focus towards the variance reduction methods in MCNP,
these
methods are by no means limited to this single software package. Indeed, a
number of other Monte Carlo radiation transport packages also include these
methods.

Population control methods adjust the particle population in the problem to
obtain better sampling in regions of interest by preferentially increasing or
decreasing the particle population.
These methods
are called splitting and rouletting.
Splitting is a method by which the particle population can be increased by
splitting a single higher-weight particle into several lower-weight particles.
Rouletting, conversely, reduces the particle population by stochastically
killing particles. Particles that survive a rouletting routine have their weight
adjusted higher, thereby conserving weight in the routine.
Both splitting and roulette maintain a
fair game by adjusting the particle weights as splitting and rouletting are
performed; the sum of the child particle weights is the same as the parent
weight as it entered the routine.
To use population control methods effectively as a variance reduction technique,
splitting is performed in high-importance regions and roulette is performed in
 low-importance
  regions. Population control methods can be applied to
  include geometry, energy and time,
  as well as a weight cutoff.

Rouletting and splitting can be combined in a
  single method, generally referred to as weight windows.

Modified sampling methods adjust transport by sampling from a different probability
distribution function than the actual distribution for the problem. This is
possible only
 if, as with population control methods, the particle weights are adjusted
 accordingly.
 The new probability distribution function should bias particles in regions of high
 importance to the problem tallies. In MCNP, a number of modified sampling
 methods exist.
 These include: the exponential transform, implicit capture, forced collisions, source
 biasing, and neutron-induced photon production biasing.

Truncation methods stop tracking particles in a region of phase-space that is of
low-importance to the tally. These methods can be used in space (a vacuum boundary
condition), energy (eliminate particles above or below a specified energy), or
time (stop
tracking after a given time). To effectively use these methods, the user must be
aware of
particles' importance to a tally result. If particles that are important to a
result are
 eliminated with a truncation method, the tally will be underestimated, and the
 solution
  will not incorporate the problem physics into the solution effectively.

DESCRIBE WEIGHT WINDOWS HERE

DESCRIBE SOURCE BIASING HERE

The remainder of this literature review will focus on efforts to utilize population
control methods and modified sampling methods for variance reduction.

\subsection{Automated Variance Reduction Methods for Monte Carlo Radiation
Transport}
\label{subsec:AutomatedMCVR}

It is important, in utilizing any variance reduction technique, to ensure that a
fair game
is being played. The user must ensure that the fundamental nature of the problem
is not
being changed by using a variance reduction technique, or the answer will not be
representative of the original problem. Automated variance reduction techniques aim to
eliminate this uncertainty for the user by estimating the importance of
particles in some
 way and then setting up variance reduction parameters automatically.

Early on, a number of methods explored automating some aspect of variance
reduction for
Monte Carlo. Some of these methods used a fast, low-particle Monte Carlo
transport run to gain an initial estimate for a cell's importance and generate
variance reduction parameters from them to bias a more computationally-intensive
run. Naturally, the variance reduction parameters generated by utilizing this
technique were limited by the statistical uncertainty in the figures used to
generate them. Furthermore, for deep-penetration particle transport, the
variance reduction parameters for low flux regions were exceedingly difficult to
generate.

As with any variance reduction technique, the goal to be achieved is a faster
time to achieve a solution for a desired uncertainty, and to reduce the variance
for the tally.
Thus the metrics for success are often the figure of merit (FOM) and the
variance for the tally obtaining the solution.
