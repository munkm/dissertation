\section{Monte Carlo Variance Reduction}

\subsection{Variance Reduction Techniques}
\label{sec:MCvar}

To begin a discussion on variance reduction techniques for monte carlo particle transport, it will be prudent to first discuss the nature of variance reduction itself. 

\textbf{describe the central limit theorem and how it relates to monte carlo statistics here }

Estimate of the variance: 
\begin{equation}
S^{ 2 }=\sum _{ i=1 }^{ N }{ \frac { (x_{ i }-\bar { x } )^{ 2 } }{ N-1 }  } \cong \bar{x^2}-\bar{x}^2
\end{equation}
\label{eq:Var}

The variance of the mean value, or variance of $\bar{x}$, follows as:
\begin{equation}
S^{ 2 }_{ \bar { x }  }=\frac{S^2}{N}
\end{equation}
\label{eq:VarMean}

From \ref{eq:VarMean}, one can see that relationship between variance and N follows: 
\begin{equation}
S_{ \bar { x }  }=\sqrt { \frac { S^{ 2 } }{ N }  } =\frac { S }{ \sqrt { N }  }
\end{equation} 
\label{eq:VarN}

The relative error normalizes the variance by the mean value. 
\begin{equation}
R = \frac{S_{ \bar { x }  }}{\bar{x}}
\end{equation}
\label{eq:RelativeErr}

S to R relation
\begin{equation}
S^2\:\propto\: R^2\:\propto\:\frac{1}{N}
\end{equation}
\label{eq:S to R}

The figure of merit (FOM) is often used as a success metric for variance reduction techniques. 
This is because
\begin{equation}
FOM=\frac { 1 }{ R^{ 2 }T } 
\end{equation}
\label{eq:FOM}

MCNP \cite{hendricks_mcnp_1985, brown_mcnp_2002} has a number of techniques for variance
reduction included in its distribution. Overall, these variance reduction techniques fall
 into four general categories: truncation methods, population control methods, modified 
 sampling methods, and partially-deterministic methods. Of note for this discussion are 
 population control methods and modified sampling methods, which are discussed in a number 
 of the papers referenced herein. 

Population control methods either increase the particle population by splitting or
decrease the particle population by rouletting. Both splitting and roulette maintain a 
fair game by adjusting the particle weights as splitting and rouletting are performed. 
To use population control methods effectively as a variance reduction technique, splitting
 is performed in high-importance regions and roulette is performed in low-importance
  regions. MCNP's population control methods include geometry, energy and time splitting
  and roulette, as well as a weight cutoff and weight windows. Weight windows utilize both
 splitting and roulette to keep particles within a desired weight range. 

Modified sampling methods adjust transport by sampling from a different probability 
distribution function than the actual distribution for the problem. This is possible only
 if, as with population control methods, the particle weights are adjusted accordingly. 
 The new probability distribution function should bias particles in regions of high 
 importance to the problem tallies. In MCNP, a number of modified sampling methods exist.
 These include: the exponential transform, implicit capture, forced collisions, source 
 biasing, and neutron-induced photon production biasing. 

Truncation methods stop tracking particles in a region of phase-space that is of
low-importance to the tally. These methods can be used in space (a vacuum boundary
condition), energy (eliminate particles above or below a specified energy), or time (stop
tracking after a given time). To effectively use these methods, the user must be aware of 
particles' importance to a tally result. If particles that are important to a result are
 eliminated with a truncation method, the tally will be underestimated, and the solution
  will not incorporate the problem physics into the solution effectively. 
  
The remainder of this literature review will focus on efforts to utilize population 
control methods and modified sampling methods for variance reduction.

DESCRIBE WEIGHT WINDOWS HERE

DESCRIBE SOURCE BIASING HERE

\subsection{Automated Variance Reduction Using Monte Carlo}
\label{sec:AutomatedVR}

It is important, in utilizing any variance reduction technique, to ensure that a fair game
is being played. The user must ensure that the fundamental nature of the problem is not 
being changed by using a variance reduction technique, or the answer will not be 
representative of the original problem. Automated variance reduction techniques aim to 
eliminate this uncertainty for the user by estimating the importance of particles in some
 way and then setting up variance reduction parameters automatically. 

Early on, a number of methods explored automating some aspect of variance reduction for 
Monte Carlo. Some of these methods used a fast, low-particle Monte Carlo transport run to gain an initial estimate for a cell's importance and generate variance reduction parameters from them to bias a more computationally-intensive run. Naturally, the variance reduction parameters generated by utilizing this technique were limited by the statistical uncertainty in the figures used to generate them. Furthermore, for deep-penetration particle transport, the variance reduction parameters for low flux regions were exceedingly difficult to generate.  

With the advent of the weight window technique, Booth \cite{booth_automatic_1982} proposed 
a method to calculate a cell's importance. In this method, Booth suggested estimating the cell's importance using monte carlo transport as:
\begin{equation}
Importance  = \frac{\text{score of particles leaving the cell}}{\text{weight leaving the cell}} 
\end{equation}
\label{eq:BoothImp}

Booth went on to suggest that the importance generator could be used in a complementary nature to the weight window variance reduction technique available in MCNP at that time, where a weight window value inversely proportional to the variance could be used for variance reduction.
In this case, the track weight times the expected score is approximately constant in the problem. 
In test problems, the importance generator had estimated the problem cells' importances after only a few iterations. 

Henricks \cite{hendricks_code-generated_1982}, concurrently to Booth, developed an automated geometry- and energy-dependent weight window generator for MCNP. 
Hendricks' method differed from Booth's in that the weight window generation was nonspecific to a tally. 
Rather, the generator populated all regions of phase space equally. 
To do this, the generator set boundaries for the weight window bounds, $W_{low}$ and $W_{high}$. These were calculated based on the total weight entering and exiting the weight window target region. 
Thus, the procedure 

Tsang and Hoffman \cite{tsang_monte_1988}

Hybrid methods, in particular, attempt to use the speed of a determinstic calculation to generate variance reduction parameters for Monte Carlo software. 

As with any variance reduction technique, the goal to be achieved is a faster time to achieve a solution for a desired uncertainty, and to reduce the variance for the tally. 
Thus the metrics for success are often the figure of merit (FOM) and the variance for the tally obtaining the solution. 