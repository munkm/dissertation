\section{Automated Variance Reduction Methods for Global Solutions}
\label{sec:GlobalVR}

Variance reduction methods for global solutions are designed to obtain an even
distribution of error across several tallies or a tally map that spans the
entire problem phase-space.
The previous section detailed several methods that automate variance reduction
for localized tallies. However, for global solutions these methods do not work
well. The global tally suffers from a large range in variance across the
physical problem space, and the solution is dependent on the flux distribution
throughout the problem.

This section describes several methods that provide
automated variance reduction for global solutions or multiple tallies. The
general principle that these methods follow is that by distributing particles
evenly throughout the Monte Carlo problem, a global tally will have
approximately the same sample size in each region, resulting in a uniform
variance across the tally. This often requires a forward deterministic solution
to determine the density of forward particles throughout the problem, and
subsequently using that forward distribution to aid in generating an
importance map. This
section summarizes the theory behind a number of existing global variance
reduction methods. The section is concluded with a summary of how the
methods performed and in which problems they performed well.

\subsection{Cooper's Isotropic Weight Windows}
\label{subsec:CooperGlobal}

Cooper and Larsen developed a weight window technique to reduce the variance of
Monte Carlo global solutions \cite{cooper_automated_2001} using a calculation of
the forward flux from solutions obtained from diffusion, quasidiffusion
\cite{miften_quasi-diffusion_1993}, or pure
Monte Carlo. In their work, Cooper
and Larsen utilized a forward solution to the transport equation to generate
weight window values to uniformly distribute particles throughout the problem.
By doing so, the variance in the
scalar flux remained relatively constant throughout the problem for a
problem-wide tally, rather than
rising significantly with increasing distance from the forward source. Cooper's
``isotropic'' weight windows (named because they were not dependent on $\hat\Omega$ )
dependent on $\vec{r}$ are given by:
\begin{subequations}
\label{coopers}
\begin{equation}
  \bar{ww}(\vec{r}) = \frac{\phi(\vec{r})}{\max \phi(\vec{r})} \:,
  \label{eq:coopertarget}
\end{equation}
\begin{equation}
  ww(\vec{r})_{top} = \rho \bar{ww}(\vec{r}) \:,
\end{equation}
and
\begin{equation}
  ww(\vec{r})_{bottom} = \frac{\bar{ww}(\vec{r})}{\rho} \:,
\end{equation}
where $\rho$ is the weight window scaling factor.
\end{subequations}
Note that by setting the weight window target value to be inversely
proportional to the total flux in the cell, the density of particles throughout
the problem ends up as roughly constant. Also note from Eq. \
\eqref{eq:coopertarget} that the weight windows are depend on space only.

In practice, Cooper's algorithm iteratively switches between solving the
diffusion equation with transport correctors (Eddington factors described by
\cite{goldin_quasi-diffusion_1964}), and Monte Carlo solutions; this process is
known as quasidiffusion \cite{goldin_quasi-diffusion_1964,
miften_quasi-diffusion_1993}. An initial
quasidiffusion solution is used to generate weight windows, and then after a
relatively short runtime, the Monte Carlo solution is used to
generate updated Eddington factors for
the quasidiffusion solution.

Because Cooper's method depends on Monte Carlo to generate the Eddington
factors for the quasidiffusion problem, this method is limited by the iterative
switch between the quasidiffusion solution and the Monte Carlo solution. The
frequency with which this switching happens is entirely up to the user, but may
drastically affect the efficiency of the method. Further, Cooper notes that we
do not know at what point in time (for which number of N particles) the Monte
Carlo solution becomes more accurate than the quasidiffusion solution, which is
an important issue in choosing solution parameters.

\subsection{Becker's Global Weight Windows}
\label{subsec:BeckerGlobal}

Becker, in addition to developing the local VR method discussed in Section
\ref{sec:beckerlocal}, developed a global space-energy
weight correction method both with (Section \ref{sec:AngleVR}) and
without directional biasing \cite{becker_hybrid_2007, becker_hybrid_2009}.
\label{eq.beckerglobal}
Becker's global method uses a formulation of the space-dependent
contributon flux, as with the local weight windows described in Section
\ref{sec:beckerlocal}. For reference, those are
defined in Eqs.\ \eqref{eq:beckerconributon} and
\eqref{eq:beckerconributonspace}.

Becker defines the spatially-dependent
contributon flux parameter as
$B(\vec{r})$, where
\begin{equation}
  B(\vec{r}) = \tilde{\phi^c}(\vec{r}).
\end{equation}
Becker's method defines a different adjoint source
distribution depending on the response desired for the calculation.
To optimize the flux the adjoint source is defined as:
\begin{subequations}
\begin{equation}
  q^{\dagger}(\vec{r},E) = \frac{1}{\phi(\vec{r},E)}.
\end{equation}
If the detector response is desired then
\begin{equation}
  q^{\dagger}(\vec{r},E) = \frac{\sigma_d(\vec{r},E)}
  {\int_0^{\infty} \phi(\vec{r},E) \sigma_d(\vec{r},E) dE},
\end{equation}
can be used instead.
\end{subequations}
The space- and energy-dependent weight windows are then a function of the
contributon flux, where
\begin{equation}
  \bar{w}(\vec{r},E) = \frac{B(\vec{r})}{\phi^{\dagger}(\vec{r},E)} \:.
\label{eq:beckerglobalww}
\end{equation}

The process followed by Becker's global method uses two deterministic
calculations to generate weight windows for the Monte Carlo calculation. First,
the forward flux is calculated deterministically and used to construct the
adjoint source distribution. After the adjoint solution
is run, the contributon flux is calculated. The contributon flux and the adjoint
flux are then used to construct the weight windows.

Becker's method aims to distribute response evenly throughout the problem.
However, like FW-CADIS
(discussed below in Section \ref{subsec:FWCADIS}),
the global response weight windows are proportional to the forward response,
\begin{equation}
  \bar{w}(\vec{r},E) \propto \frac{\int \sigma(\vec{r},E) \phi (\vec{r},E) dE}
                                  {\sigma (\vec{r},E)}
\end{equation}
rather than the forward flux as in Cooper's method, where
$\bar{w}(\vec{r},E) \propto \phi(\vec{r},E)$ .

In implementation, both Becker and Cooper's global methods
undersampled the source (in comparison to FW-CADIS, which will be described in
Section \ref{subsec:FWCADIS}) for a specified calculation
time. However, Becker's method sampled $\sim$1/3 the number of particles that
Cooper's method did. Notably, Becker's method did obtain better relative
uncertainties for low flux-regions in the problem.

\subsection{FW-CADIS}
\label{subsec:FWCADIS}

In 2007, Peplow, Blakeman, and Wagner \cite{peplow_advanced_2007} proposed three
methods by which variance reduction could be decreased in global mesh tallies in
deep-penetration radiation transport problems. The first method, using a CADIS
calculation where the adjoint source is defined at the problem boundary, aimed
at moving particles outward to the problem edges.
The second method used standard CADIS, but instead
defined each cell as equally important, so the adjoint source was defined
equally throughout the problem phase-space. The last method, called Forward-Weighted
CADIS (FW-CADIS), distributed the adjoint source across mesh cells as an inverse
relation to the forward response of the cell. In their work, Peplow et al.\ found
that the first method had large uncertainties in areas of the problem far
from the boundary; the second method performed slightly, but not significantly,
better than analog; and the third method had the most uniform uncertainty
distribution.

FW-CADIS
\cite{wagner_forward-weighted_2007,wagner_forward-weighted_2009,wagner_forward-weighted_2010}
built off of the work by Cooper and the CADIS method.
Like Becker's method, FW-CADIS uses a forward deterministic calculation to
determine the source distribution for the adjoint calculation. Unlike Becker's
method, which used contributon fluxes to construct weight windows,
the CADIS method uses adjoint fluxes as the basis of the weight window values.
Similar to
Cooper's method, however, FW-CADIS uses the forward calculation to determine how to
evenly distribute particles throughout the problem. Like CADIS, FW-CADIS uses
the adjoint solution from the deterministic calculation to generate consistent
source biasing, weight windows, and particle birth weights.

The adjoint source for the adjoint calculation is dependent on the
desired response for the system. The generic description for the adjoint source
is given by Eq. \eqref{eq:adjointsourcesimple} and more specific parameters are
given by Eqs. \eqref{eq:FWCadglobaldose}-\eqref{eq:FWCadglobalflux2}. First, we
can describe a general form of the adjoint source definition for all
phase-space, $P$, as:
\begin{equation}
  { q^{\dagger}} (P)=\frac{\sigma_d(P)}{R}.
\label{eq:adjointsourcesimple}
\end{equation}
Thus the adjoint source is dependent on the detector (or tally)
cross-section and whatever response is being calculated in the system. Depending
on whether the response is a flux or a dose rate, the adjoint source will
differ. For example,
the adjoint source for the spatially dependent global dose, $\int
\phi(\vec{r},E)\sigma_d(\vec{r},E) dE$ is:
\begin{subequations}
\begin{equation}
  { q^{\dagger} }(\vec { r } ,E)= \frac { \sigma _{ d }(\vec { r } ,E) }
       { \int { \sigma _{ d }(\vec { r } ,E)\phi (\vec { r } ,E,) } dE }.
\label{eq:FWCadglobaldose}
\end{equation}
The adjoint source for the spatially dependent total flux $\int \phi(\vec{r},E)
dE $ is:
\begin{equation}
  { q^{\dagger} }(\vec { r }) = \frac { 1 }{ \int { \phi (\vec { r } ,E) } dE }.
\label{eq:FWCadglobalflux}
\end{equation}
Last,
the adjoint source for the energy- and spatially- dependent flux
$\phi(\vec{r},E)$ is:
\begin{equation}
  { q^{\dagger} }(\vec { r } ,E) = \frac { 1 }{\phi (\vec { r } ,E) }.
\label{eq:FWCadglobalflux2}
\end{equation}
\end{subequations}

The process followed by FW-CADIS is to initially run a deterministic forward
calculation to obtain the forward response. This solution is used to create the
source distribution for the adjoint
problem. A second deterministic calculation is run to obtain the adjoint
solution. The adjoint solution is then used to generate variance reduction
parameters in the same manner as CADIS.

\subsection{Other Notable Methods}

Baker and Larsen showed that the exponential transform can be used to generate
VR parameters for global low-variance solutions in Monte Carlo
\cite{baker_localexponential_1993}. In this work, Baker used a forward diffusion
solution to generate parameters for a combination of VR techniques: implicit
capture and weight cutoff, geometry splitting / rouletting with implicit
capture and weight cutoff, and the exponential transform combined with
implicit capture and a weight cutoff. The exponential transform method was then
compared to the other combinations of VR techniques to quantify its success.
In their work, Baker and Larsen found that
the exponential transform approach did not work well for highly
scattering problems, where geometry
splitting and Russian roulette were generally better options. Their work did not
focus on generating weight window values, nor was it tested on deep-penetration
shielding problems.

While the aforementioned methods in this and the previous sections use
deterministically-obtained solutions to generate importance maps, it should be
noted that not all methods use this approach. Booth and Hendricks' methods used
initial Monte Carlo calculations to reduce the relative error in tallies. Two
methods in the global variance reduction realm are notable in that they too use
Monte Carlo estimates of the flux to generate variance reduction parameters
\cite{van_wijk_easy_2011, davis_comparison_2011}.
Van Wijk et al. \cite{van_wijk_easy_2011} developed an automated weight
window generator that used a Monte
Carlo calculation of the forward flux to generate weight window values. The
weight window target values could be determined based on either a flux-centered
scheme like Cooper's (Eq. \eqref{eq:coopertarget})
or by using a ratio of the square roots of the fluxes. The
second method is a combination of Cooper's weight window target values and
knowing that the relative error in a region
is proportional to the square root of the number
of particles.
Van Wijk et al.\ applied their methods to a PWR facility and observed a FOM
increase by a factor of $>$200 when compared to analog Monte Carlo. However, as
with other Monte Carlo-generated VR parameters, for deep-penetration problems
this approach relies on adequate sampling of all phase-space, which could be
computationally prohibitive.

The Method of Automatic Generation of Importances by Calculation (MAGIC) method
was proposed in parallel by Davis and Turner \cite{davis_comparison_2011}.
As with Van Wijk's method, the MAGIC method uses an analog forward Monte Carlo
--potentially with several iterations--calculation to generate weight windows.
The initial
Monte Carlo runs used to generate the importance map took less time to converge
by
using multigroup (rather than continuous energy) cross section data as well
as energy cutoffs. MAGIC
converged on a finalized importance map by iteratively running several
lower-fidelity
Monte Carlo calculations.

Davis and Hendricks compared three variants of MAGIC
to FW-CADIS in ITER fusion energy systems.
These three variants used different weight window adjustments for importances:
weight windows in cells based on existing weight information, weight windows in
mesh cells based on flux information, and weight windows in cells based on
population density.
It was concluded that the most effective method for variance
reduction of those proposed in the paper
was MAGIC's weight window in mesh cells based on flux
information. In this case, FW-CADIS' FOM was 65\% that of MAGIC's. This compared
similarly to Van Wijk's method, where the flux-based results continued improving
the FOM as the computational time increased.
The authors did not make it clear how many iterations were required, on average,
to generate the finalized weight window map or if the time to iteratively
generate the importance map was included in the FOM.
While FW-CADIS' FOM was lower
than MAGIC's, FW-CADIS had the highest fraction of cell voxels with very low
relative errors.

Peplow et al.\ \cite{peplow_comparison_2012} compared the performance of
Cooper's method, Van Wijk's method, Becker's method, and FW-CADIS
across a number of shielding calculations. For a simple shielding
problem, FW-CADIS had the shortest runtime, which included the forward and
adjoint deterministic runtimes, and had a FOM 80x higher than the analog
calculation, and more than 3x higher than the next best hybrid method. Van Wijk's method was
the only method other than FW-CADIS to pass all statistical convergence checks
for the problem, but its reported FOM was lower than either Becker's method or
FW-CADIS. In a second deep penetration shielding problem, FW-CADIS was the
only method
that passed all statistical convergence checks. FW-CADIS also had the highest
reported FOM for this problem. The timing for all of the methods were
comparable. Peplow et al.\ also ran two ``challenge'' problems. As with the first
two problems, FW-CADIS outperformed the other methods and passed all statistical
checks. Becker's method was
consistently comparable to FW-CADIS in reported FOMs, but only passed all of the
statistical checks in a single challenge problem. Becker's method also performed
relatively better than the other methods in deep-penetration challenge problems.

The ubiquity and continued development of global variance reduction methods
illustrates the need and desire for them in the nuclear engineering community.
Some of the
methods discussed in this section--including Becker's global weight windows,
Cooper's weight windows, Van Wijk's method, and FW-CADIS--have
been applied to large application problems and compared to other methods.
All of the methods reduce the time to a ``good'' solution--thus improving the
final FOM--when
compared to analog Monte Carlo. When compared against one another, FW-CADIS
consistently outperforms the other methods.
