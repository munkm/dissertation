\section{Automated Variance Reduction Methods for Global Solutions}
\label{sec:GlobalVR}

Variance reduction methods for global solutions are designed to obtain an even
distribution of error across several tallies or a tally map that spans the
entire problem phase-space.
The previous section detailed several methods that automate variance reduction
for localized tallies. However, for global solutions these methods do not work
well. The global tally suffers from a large range in variance across the
physical problem space, and the solution is dependent on the flux distribution
throughout the problem. Here several methods are described that provide
automated variance reduction for global solutions or multiple tallies. The
general principle that these methods follow is that by distributing particles
evenly throughout the Monte Carlo problem, a global tally will have
approximately the same sample size in each region, resulting in a uniform
variance across the tally. This often requires a forward deterministic solution
to determine the density of forward particles throughout the problem, and then
using that forward distribution to aid in generating an importance map. The
methods described here use this solution to generate a variety of variance
reduction parameters for Monte Carlo, with varying results.

\subsection{Cooper's Isotropic Weight Windows}
\label{subsec:CooperGlobal}

Cooper and Larsen developed a weight window technique to reduce the variance of
Monte Carlo global solutions \cite{cooper_automated_2001} using a calculation of
the forward flux from solutions obtained from diffusion, quasidiffusion
\cite{miften_quasi-diffusion_1993}, or pure
Monte Carlo. In their work, Cooper
and Larsen utilized a forward solution to the transport equation to generate
weight window values to uniformly distribute particles throughout the problem.
By doing so, the variance in the
scalar flux remained relatively constant throughout the problem for a
problem-wide tally, rather than
rising significantly with increasing distance from the forward source. Cooper's
``isotropic'' weight windows (named because they were not dependent on $\hat\Omega$ )
dependent on $\vec{r}$ are given by:
\begin{subequations}
\label{coopers}
\begin{equation}
  \bar{ww}(\vec{r}) = \frac{\phi(\vec{r})}{max \phi(\vec{r})}
  \label{eq:coopertarget}
\end{equation}
\begin{equation}
  ww(\vec{r})_{top} = \rho \bar{ww}(\vec{r})
\end{equation}
\begin{equation}
  ww(\vec{r})_{bottom} = \frac{\bar{ww}(\vec{r})}{\rho}
\end{equation}
where $\rho$ is the weight window scaling factor.
\end{subequations}
Note that by setting the weight window target value to be inversely
proportional to the total flux in the cell, the density of particles throughout
the problem ends up being roughly constant. Also note from Eq.
\eqref{eq:coopertarget} that the weight windows are dependent on space only.

In practice, Cooper's algorithm was to iteratively switch between solving the
diffusion equation with transport correctors (Eddington factors described by
\cite{goldin_quasi-diffusion_1964}); a process
known as quasidiffusion \cite{goldin_quasi-diffusion_1964,
miften_quasi-diffusion_1993}, and with a Monte Carlo solution. An initial
quasidiffusion solution was used to generate weight windows, and then after a
time the Monte Carlo solution was used to generate updated eddington factors for
the quasidiffusion solution.

Because Cooper's method depended on Monte Carlo to generate the Eddington
factors for the quasidiffusion problem, this method was limited by the iterative
switch between the quasidiffusion solution and the Monte Carlo solution. The
frequency by which this switching happens is entirely up to the user, but may
drastically affect the efficiency of the method. Further, Cooper notes that we
do not know at what point in time (for which number of particles N) the Monte
Carlo solution becomes more accurate than the quasidiffusion solution, which is
an important issue in choosing solution parameters.

\subsection{Becker's Global Weight Windows}
\label{subsec:BeckerGlobal}

Becker, in addition to developing a local VR method, developed a global space-
energy- weight correction method both with (Section \ref{sec:AngleVR}) and
without directional biasing: \cite{becker_hybrid_2007, becker_hybrid_2009}.

\begin{subequations}
\label{eq.beckerglobal}
Becker's global method utilizes a formulation of the energy- and space-
dependent contributon flux like eqs. \eqref{eq:beckerconributon} and
\eqref{eq:beckerconributonspace}, but defines the spatial parameter as:
$B(\vec{r})$:
\begin{equation}
  B(\vec{r}) = \tilde{\phi^c}(\vec{r})
\end{equation}
Similarly to FW-CADIS, Becker's method defines a different adjoint source
distribution depending on the response desired for the calculation:
\begin{equation}
  q^{\dagger}(\vec{r},E) = \frac{1}{\phi(\vec{r},E)}
\end{equation}
\begin{equation}
  q^{\dagger}(\vec{r},E) = \frac{\sigma_d(\vec{r},E)}
                                {\int_0^{\infty} \sigma_d(\vec{r},E) dE}
\end{equation}
The space- and energy- dependent weight windows are:
\begin{equation}
  \bar{w}(\vec{r},E) = \frac{B(\vec{r})}{\phi^{\dagger}(\vec{r},E)}
\label{eq:beckerglobalww}
\end{equation}
\end{subequations}

Becker's method is similar to both Cooper's earlier method, but like FW-CADIS
(Section \ref{subsec:FWCADIS}),
the global response weight window is proportional to the forward response, i.e.
\begin{equation}
  \bar{w}(\vec{r},E) \propto \frac{\int \sigma(\vec{r},E) \phi (\vec{r},E) dE}
                                  {\sigma (\vec{r},E)}
\end{equation}
rather than the forward flux
\begin{equation}
  \bar{w}(\vec{r},E) \propto \phi(\vec{r},E)
\end{equation}
 as it is with Cooper's method.

 In implementation, both Becker's global method and Cooper's global method
 undersampled the source (in comparison to FW-CADIS, which will be described in
 Section \ref{subsec:FWCADIS}) for a specified calculation
 time. However, Becker's method sampled ~1/3 the number of particles that
 Cooper's method did. Notably, Becker's method did obtain better relative
 uncertainties for low values in the problem.

\subsection{FW-CADIS}
\label{subsec:FWCADIS}

In 2007, Peplow, Blakeman, and Wagner \cite{peplow_advanced_2007} proposed three
methods by which variance reduction could be decreased in global mesh tallies in
deep-penetration radiation transport problems. The first method, using a CADIS
calculation where the adjoint source is defined at the problem boundary, aimed
at moving particles outward. The second method used standard CADIS but instead
defined each cell as equally important. The last method, called Forward-Weighted
CADIS (FW-CADIS), distributed the adjoint source across mesh cells as an inverse
relation to the forward response of the cell. In their work, Peplow et al. found
that the first method had large uncertainties in areas of the problem distant
from the boundary; the second method, performed slightly better than analog, but
not significantly, and the third method had the most uniform uncertainty
distribution.

FW-CADIS
\cite{wagner_forward-weighted_2007,wagner_forward-weighted_2009,wagner_forward-weighted_2010}
built off of existing theory, namely Cooper's
concept of using a forward deterministic calculation to evenly distribute
particles throughout the problem subspace, to obtain global tallies. However,
Wagner et al. took a different approach than Cooper: by using a forward
deterministic estimate of the flux or desired response as the adjoint source for
an adjoint calculation, biasing parameters for global responses could be
calculated. The adjoint source for the adjoint calculation thus depends on the
desired response for the system. The generic description for the adjoint source
is given by Eq. \eqref{eq:adjointsourcesimple} and more specific parameters are
given by Eqs. \eqref{eq:FWCadglobaldose}-\eqref{eq:FWCadglobalflux2}:

\begin{subequations}
\begin{equation}
  { q^{\dagger}} (P)=\frac{\sigma_d(P)}{R}
\end{equation}
\label{eq:adjointsourcesimple}

The adjoint source for the spatially dependent global dose, $\int
\phi(\vec{r},E)\sigma_d(\vec{r},E) dE$:
\begin{equation}
  { q^{\dagger} }(\vec { r } ,E)= \frac { \sigma _{ d }(\vec { r } ,E) }
       { \int { \sigma _{ d }(\vec { r } ,E)\psi (\vec { r } ,E,) } dE }
\end{equation}
\label{eq:FWCadglobaldose}

The adjoint source for the spatially dependent total flux $\int \phi(\vec{r},E) dE $:
\begin{equation}
  { q^{\dagger} }(\vec { r }) = \frac { 1 }{ \int { \phi (\vec { r } ,E) } dE }
\end{equation}
\label{eq:FWCadglobalflux}

The adjoint source for the energy- and spatially- dependent flux $\phi(\vec{r},E)$:
\begin{equation}
  { q^{\dagger} }(\vec { r } ,E) = \frac { 1 }{\phi (\vec { r } ,E) }
\end{equation}
\label{eq:FWCadglobalflux2}
\end{subequations}

Comparing these methods, \cite{peplow_comparison_2012}

\subsection{Other Notable Methods}
Baker and Larsen showed that the exponential transform can be used to generate
VR parameters for global low-variance solutions in Monte Carlo
\cite{baker_localexponential_1993}. In this work, Baker used a forward diffusion
solution to generate parameters for a combination of VR techniques: implicit
capture and weight cutoff, geometry splitting / Russian roulette and implicit
capture and a weight cutoff, and the exponential transform combined with
implicit capture and a weight cutoff. The exponential transform method was then
compared to the other combinations of VR techniques to quantify its success.
In their work, Baker and Larsen found that
this approach did not work well for highly scattering problems, where geometry
splitting and Russian roulette were generally better options. Their work did not
focus on generating weight window values, nor was it tested on deep-penetration
shielding problems.


Van Wijk's method, like CADIS and FW-CADIS, is a variance reduction method
designed to operate with little user input \cite{van_wijk_easy_2011}.
Van Wijk et al. applied their method to a PWR facility and observed a FOM
increase by a factor of $>$200.


The Method of Automatic Generation of Importances by Calculation (MAGIC) is a
global variance reduction technique \cite{davis_comparison_2011} proposed by
Davis in 2011.
Rather than using a deterministic solution to obtain an estimate of the forward
flux to generate an importance map, the MAGIC method uses an analog Monte Carlo
run with multigroup cross section data and (optionally) a high energy cut-off.
This initial analog run is used to generate initial importance map for a
secondary run.
This process iterates until several Monte Carlo simulations are run and a
finalized importance map is generated.
In their paper, the authors compared three variants of MAGIC to FW-CADIS in ITER
fusion energy systems.
These three variants used different weight window adjustments for importances:
weight windows in cells based on existing weight information, weight windows in
mesh cells based on flux information, and weight windows in cells based on
population density.
Davis and colleagues concluded that the most effective method for variance
reduction in their system was MAGIC's weight window in mesh based on flux
information, where FW-CADIS' FOM was 65\% that of MAGIC's.
The authors did not make it clear how many iterations were required, on average,
to generate the finalized weight window map or if the time to iteratively
generate importance maps were included in the FOM.
It is unclear if only the final Monte Carlo simulation's runtime was used to
calculate the FOM.
Furthermore, the authors did not clearly state how the biasing parameters were
calculated, other than that they used flux, population density, or weight
information.
%
%
%----------------------------------------------------------------------------------------
% While not particularly relevant to the immediate work described in this thesis, the MS-CADIS method used a sophisticated method to generate a variance-reduced source term for
% photon does in a shutdown system.

%However, MS-CADIS, like CADIS and FW-CADIS, employs exclusively energy- and space- based variance reduction methods. As a result, it still suffers from similar behaviors as CADIS and FW-CADIS in strongly anisotropic systems.

%MS-CADIS \cite{ibrahim_multi-step_2014, ibrahim_analysis_2014}
