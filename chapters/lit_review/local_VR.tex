\section{Automated Variance Reduction Methods for Local Solutions}
\label{sec:localVR}

The next several sections (\ref{sec:localVR} through \ref{sec:AngleVR})
describe different ways that deterministically-obtained importance
functions
can be applied to variance reduction methods in practice. Local variance
reduction methods are those that optimize a tally response in a localized region
of the problem phase-space. These types of problems may be the most immediately
physically intuitive to a user,
where a person standing $x$ meters away from a source may wish to know their
personal dose rate. In this section, notable
automated deterministically-driven variance reduction methods that
have been designed for such localized response optimization are described.
Recall that Booth's importance generator (Section \ref{subsec:AutomatedMCVR})
was also designed for localized tally results, but will not be elaborated upon
here.

\subsection{CADIS}
\label{sec:CADIS}

In 1997, Haghighat and Wagner introduced the Consistent Adjoint-Driven
Importance Sampling method (CADIS)
\cite{wagner_automatic_1997,wagner_automated_1998,haghighat_monte_2003} as a
tool for automatic variance reduction for local tallies in Monte Carlo. CADIS
was unique in that it used the adjoint solution from a deterministic simulation to
consistently bias the particle distribution and particle weights. Earlier
methods had not ensured the consistency between source biasing and particle
birth weights. CADIS was applied to a large number of application problems and
performed well in reducing the variance for local tallies
\cite{wagner_review_2011}.

The next several paragraphs present and discuss
the theory supporting CADIS. Note that the theory presented is specific to
space-energy CADIS, which is what is currently implemented in existing software.
The original CADIS equations
are based on space and energy ($\vec{r}, E$) dependence, but not angle, so
$\phi^{\dagger}$ can be used rather than $\psi^{\dagger}$. This does not mean
that CADIS is not applicable to angle. This is merely a choice made by the
software and method developers given the computational resources
required to calculate and store
full angular flux datasets, and the inefficiency that using angular fluxes might
pose for problems where angle dependence is not paramount.

In trying to reduce the variance for a local tally, we aim to encourage particle
movement towards the tally or detector location. In other words, we seek to
encourage particles to induce a detector response while discouraging them from
moving through unimportant regions in the problem.
Recall from  Eqs. \eqref{eq:response1} and \eqref{eq:response2} that the total
system response can be expressed as either an integral of $\psi^{\dagger}\: q_{e}$
(the adjoint flux and the forward source), or $\psi\: q_{e}^{\dagger}$
(the forward flux and the adjoint source).
Also recall that the adjoint solution is a measure for
response importance.

\begin{subequations}
\label{CADISmethod}
To generate the biased source distribution for the Monte Carlo calculation,
$\hat{q}$,
should be related to its contribution to inducing a response in the tally or
detector. It follows, then, that the biased source distribution is the ratio of
the contribution of a cell to a tally response to the tally response induced
from the entire problem. Thus, the biased source distribution for CADIS
is a function of the adjoint scalar
flux and the forward source distribtion $q$ in region $\vec{r}, E$,
and the total response $R$
\begin{equation}
\begin{split}
\hat{q}  & = \frac{\phi^{\dagger}(\vec {r} ,E)q(\vec {r}
,E)}{\iint\phi^{\dagger}(\vec {r} ,E)q(\vec {r} ,E) dE d\vec{r}} \\
         & = \frac{\phi^{\dagger}(\vec {r} ,E)q(\vec {r} ,E)}{R}.
\end{split}
\label{eq:weightedsource}
\end{equation}
The  starting weights of the particles sampled from the biased source
distribution ($\hat{q}$) must be adjusted to account for the biased source
distribution. As a result, the starting weights
are a function of the biased source distribution and the
original forward source distribution:
\begin{equation}
\begin{split}
w_0  & = \frac{q}{\hat{q}} \\
     & = \frac{R}{\phi^{\dagger}(\vec {r} ,E)}.
\end{split}
\label{eq:startingweight}
\end{equation}
Note that when Eq. \eqref{eq:weightedsource} is placed into Eq.
\eqref{eq:startingweight}, the starting weight is a function of the total
problem response and the adjoint scalar flux in $\vec{r}, E$.
The target weights for the biased particles are given by
\begin{equation}
\hat{w} = \frac{R}{\phi^{\dagger}(\vec {r} ,E)},
\label{eq:WW}
\end{equation}
\end{subequations}
where the target weight $\hat{w}$ is also a function of the total response and
the adjoint scalar flux in region $\vec{r}, E$.
The equations for $\hat{w}$ and $w_0$ match; particles are born at the same weight
of the region they are born into. Consequently, the problem limits
excessive splitting
and roulette at the particle births, in addition to consistently biasing the
particle source distribution and weights. This is the ``consistent'' feature of
the CADIS method.

CADIS supports adjoint theory by showing that using the adjoint solution
($\phi^{\dagger}$) for variance reduction parameter generation successfully
improves Monte Carlo calculation runtime. CADIS showed improvements in the FOM
when compared to analog Monte Carlo on the order of $10^2$ to $10^3$,
and on the order of five
when compared to ``expert'' determined or automatically-generated weight
windows \cite{wagner_automated_1998, wagner_automated_2002} for simple shielding
problems. For more complex
shielding problems, improvements in the FOM were on the order of $10^1$
\cite{wagner_automatic_1997, wagner_automated_1998}. Note that CADIS improvement
is dependent on the nature of the problem and that these are merely illustrative
examples.

\subsection{Becker's Local Weight Windows}
\label{sec:beckerlocal}

Becker's work in the mid- 2000s also explored generating biasing parameters for
local source-detector problems \cite{becker_hybrid_2009}. Becker noted that in
traditional weight window generating methods, some estimation of the adjoint
flux is used to bias a forward Monte Carlo calculation. The product of this
weight window biasing and the forward Monte Carlo transport ultimately
distributed particles in the problem similarly to the contributon flux.
In his work, Becker
used a formulation of the contributon flux, as described in Eq.
\eqref{eq.Cont-Flux} to optimize the flux at the forward detector location.
The
relevant equations are given by Eqs. \eqref{eq:beckerconributon} -
\eqref{eq:beckeralpha}.

\begin{subequations}
\label{eq.beckerlocal}
First, the scalar contributon flux $\phi^c$, which is a function of space and energy
is calculated with a product of the deterministically-calculated forward and
adjoint fluxes, where
\begin{equation}
\phi^c(\vec{r},E) = \phi(\vec{r},E) \phi^{\dagger}(\vec{r},E).
\label{eq:beckerconributon}
\end{equation}
This is then integrated over all energy to obtain a spatially-dependent
contributon flux
\begin{equation}
\tilde{\phi^c}(\vec{r}) = C_{norm}\int_0^{\infty } \phi^c(\vec{r},E) dE,
\label{eq:beckerconributonspace}
\end{equation}
where the normalization constant, $C_{norm}$, for a given detector volume,
$V_D$,
is:
\begin{equation}
C_{norm} = \frac{V_D}{\int_{V_D}\int_0^{\infty } \phi^c(\vec{r},E) dE dV}.
\end{equation}
The space- and energy-dependent weight windows are given by:
\begin{equation}
  \bar{w}(\vec{r},E) = \frac{B(\vec{r})}{\phi^{\dagger}(\vec{r},E)}\:,
\label{eq:beckerlocalww}
\end{equation}
where
\begin{equation}
B(\vec{r}) = \alpha(\vec{r}) \tilde{\phi^c}(\vec{r}) + 1 -  \alpha(\vec{r})\:,
\end{equation}
and
\newcommand{\firs}{\cfrac{\tilde{\phi}^c_{max}}{\tilde{\phi}^c(\vec{r})}}
\newcommand{\seco}{\cfrac{\tilde{\phi}^c(\vec{r})}{\tilde{\phi}^c_{max}}}
\begin{equation}
  \alpha (\vec{r}) = \bigg[ 1 + \exp \bigg( \firs - \seco \bigg) \bigg]^{-1} .
  \label{eq:beckeralpha}
\end{equation}
\end{subequations}

Becker found that this methodology compared similarly to CADIS
for local solution variance reduction
for a large challenge problem comprised of nested
cubes. The particle density throughout the problem was similar
between CADIS and Becker's local weight window. The FOMs were also relatively
similar, but were reported only with Monte Carlo calculation runtimes (meaning
that the deterministic runtimes were excluded). Note that
Becker's method requires both a forward and an adjoint
calculation to calculate the contributon fluxes, while CADIS requires only an
adjoint calculation.
