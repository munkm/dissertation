\section{Automated Variance Reduction Methods for Local Solutions}
\label{sec:localVR}

The next several sections (\ref{sec:localVR} through \ref{sec:AngleVR})
will describe different ways that deterministically-obtained importance
functions
can be applied to variance reduction methods in practice. Local variance
reduction methods are those that optimize a tally response in a localized region
of the problem phase-space. These types of problems may be the most immediately
physically intuitive to a user,
where a person standing $x$ meters away from a source may wish to know their
personal dose rate. In this section, notable
automated deterministically-driven variance reduction methods that
have been designed for such localized response optimization are described.
Recall that Booth's importance generator (Section \ref{subsec:AutomatedMCVR})
was designed for localized tally results.

\subsection{CADIS}
\label{sec:CADIS}

In 1997, Haghighat and Wagner introduced the Consistent Adjoint-Driven
Importance Sampling Method (CADIS)
\cite{wagner_automatic_1997,wagner_automated_1998,haghighat_monte_2003} as a
tool for automatic variance reduction for local tallies in Monte Carlo. CADIS
was unique in used the adjoint solution from a deterministic solution to
consistently bias the particle distribution and particle weights. Earlier
methods had not ensured the consistency between source biasing and particle
births.

Recall from  Eqs. \eqref{eq:response1} and \eqref{eq:response2} that the total
system response can be expressed as either an integral of the product of the
adjoint flux and the forward source, or from the product of the forward flux and
the adjoint source. By using the solution for the adjoint scalar
flux, $\phi^{\dagger}$, obtained from a deterministic calculation
and knowing the adjoint source distribution, $q^{\dagger}$ for the same problem,
source biasing and weight window variance reduction parameters could be obtained
for the forward Monte Carlo calculation. Note that the original CADIS equations
are based on space and energy ($\vec{r}, E$) dependence, but not angle, so
$\phi^{\dagger}$ can be used rather than $\psi^{\dagger}$. This does not mean
that CADIS is not applicable to angle. It was merely a choice made by the
developers given the computational resources required to calculate and store
full angular flux datasets, and the inefficiency that using angular fluxes might
pose for problems where angle dependence is not paramount.

\begin{subequations}
\label{CADISmethod}
To generate the biased source distribution for the Monte Carlo calculation,
$\hat{q}$,
should be related to it's contribution to inducing a response in the tally or
detector. It follows, then, that the biased source distribution is the ratio of
the contribution of a cell to a tally response to the tally response induced
from the entire problem. Thus, the biased source distribution
is a function of the adjoint scalar
flux and the forward source distribtion $q$ in region $\vec{r}, E$,
and the total response $R$
\begin{equation}
\begin{split}
\hat{q}  & = \frac{\phi^{\dagger}(\vec {r} ,E)q(\vec {r}
,E)}{\iint\phi^{\dagger}(\vec {r} ,E)q(\vec {r} ,E) dE d\vec{r}} \\
         & = \frac{\phi^{\dagger}(\vec {r} ,E)q(\vec {r} ,E)}{R}.
\end{split}
\label{eq:weightedsource}
\end{equation}
The  starting weights of the particles sampled from the biased source
distribution, $\hat{q}$ must be adjusted to account for the biased source
distribtution. As a result, the starting weights
are a function of the biased source distribution and the
original forward source distribution:
\begin{equation}
\begin{split}
w_0  & = \frac{q}{\hat{q}} \\
     & = \frac{R}{\phi^{\dagger}(\vec {r} ,E)}.
\end{split}
\label{eq:startingweight}
\end{equation}
Note that when Eq. \eqref{eq:weightedsource} is placed into Eq.
\eqref{eq:startingweight}, the starting weight is a fucntion of the the total
problem response and the adjoint scalar flux in $\vec{r}, E$.
The target weights for the biased particles are given by
\begin{equation}
\hat{w} = \frac{R}{\phi^{\dagger}(\vec {r} ,E)},
\label{eq:WW}
\end{equation}
\end{subequations}
where the target weight $\hat{w}$ is also a function of the total response and
the adjoint scalar flux in region $\vec{r}, E$.
The equations for $\hat{w}$ and $w_0$ match; so particles are born at the same weight
of the region they are born into. Consequently, the problem limits
excessive splitting
and roulette at the particle births, in addition to consistently biasing the
particle source distribution and weights.

\subsection{Becker's Local Weight Windows}
\label{sec:beckerlocal}

Becker's work in the mid- 2000s also explored generating biasing parameters for
local source-detector problems \cite{becker_hybrid_2009}. In his work, Becker
utilized a formulation of the contributon flux, as described in Eq.
\eqref{eq.Cont-Flux} to optimize the flux at the forward detector location. The
relevant equations are given by Eqs. \eqref{eq:beckerconributon} -
\eqref{eq:beckeralpha}.
\begin{subequations}
\label{eq.beckerlocal}
Becker's local method utilizes a formulation of the energy- and space- dependent
contributon flux:
\begin{equation}
\phi^c(\vec{r},E) = \phi(\vec{r},E) \phi^{\dagger}(\vec{r},E)
\label{eq:beckerconributon}
\end{equation}
and the spatially dependent contributon flux:
\begin{equation}
\tilde{\phi^c}(\vec{r}) = C_{norm}\int_0^{\infty } \phi^c(\vec{r},E) dE
\label{eq:beckerconributonspace}
\end{equation}
where the normalization constant, $C_{norm}$ for a given detector volume $V_D$
\begin{equation}
C_{norm} = \frac{V_D}{\int_{V_D}\int_0^{\infty } \phi^c(\vec{r},E) dE dV}
\end{equation}
The space- and energy- dependent weight windows are given by:
\begin{equation}
  \bar{w}(\vec{r},E) = \frac{B(\vec{r})}{\phi^{\dagger}(\vec{r},E)}
\label{eq:beckerlocalww}
\end{equation}
where
\begin{equation}
B(\vec{r}) = \alpha(\vec{r}) \tilde{\phi^c}(\vec{r}) + 1 -  \alpha(\vec{r})
\end{equation}
and
\begin{equation}
  \alpha (\vec{r}) = \bigg[ 1 + \exp \big(  \frac{\tilde{\phi}^c_{max}} \
  {\tilde{\phi}^c(\vec{r})} - \frac{\tilde{\phi}^c(\vec{r})} \
  {\tilde{\phi}^c_{max}} \big) \bigg]^{-1} \
  \label{eq:beckeralpha}
\end{equation}
\end{subequations}
