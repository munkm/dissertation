\section{Automated Variance Reduction Methods for Local Solutions}
\label{sec:localVR}

\subsection{CADIS}
\label{sec:CADIS}

In 1997, Haghighat and Wagner introduced the Consistent Adjoint-Driven
Importance Sampling Method (CADIS)
\cite{wagner_automatic_1997,wagner_automated_1998,haghighat_monte_2003} as a
tool for automatic variance reduction for local tallies in Monte Carlo. CADIS
was unique in that built off of adjoint theory for variance reduction, and
consistently biased the particle distribution and particle weights. In CADIS, a
deterministic estimate of the adjoint problem is solved, where the adjoint
source distribution $q^{\dagger}$ is the forward problem's detector or tally
response function, or $\Sigma_{tally}$. The solution for $\phi^{\dagger} $ could
be used to generate a biased source distribution for the forward problem:

\begin{subequations}
\label{CADISmethod}
Biasing Parameters Used by CADIS:
\begin{equation}
\begin{split}
\hat{q}  & = \frac{\phi^{\dagger}(\vec {r} ,E)q(\vec {r}
,E)}{\iint\phi^{\dagger}(\vec {r} ,E)q(\vec {r} ,E) dE d\vec{r}} \\
         & = \frac{\phi^{\dagger}(\vec {r} ,E)q(\vec {r} ,E)}{R}
\end{split}
\label{eq:weightedsource}
\end{equation}
The  starting weights of the particles sampled from the biased source
distribution, $\hat{q}$ are:
\begin{equation}
\begin{split}
w_0  & = \frac{q}{\hat{q}} \\
     & = \frac{R}{\phi^{\dagger}(\vec {r} ,E)}
\end{split}
\label{eq:startingweight}
\end{equation}
and the target weights for the particle are:
\begin{equation}
\hat{w} = \frac{R}{\phi^{\dagger}(\vec {r} ,E)}
\label{eq:WW}
\end{equation}
\end{subequations}

The equations for $w$ and $w_0$ match; so particles are born at the same weight
of the region they are born into, thus the problem limits excessive splitting
and roulette at the particle births.

CADIS has been implemented in a number of software packages, which are
summarized in Table [Table Reference Once Created]

\subsection{Becker's Local Weight Windows}
\label{sec:beckerlocal}

Becker's work in the mid- 2000s also explored generating biasing parameters for
local source-detector problems \cite{becker_hybrid_2009}. In his work, Becker
utilized a formulation of the contributon flux, as described in Eq.
\eqref{eq.Cont-Flux} to optimize the flux at the forward detector location. The
relevant equations are given by Eqs. \eqref{eq:beckerconributon} -
\eqref{eq:beckeralpha}.
\begin{subequations}
\label{eq.beckerlocal}
Becker's local method utilizes a formulation of the energy- and space- dependent
contributon flux:
\begin{equation}
\phi^c(\vec{r},E) = \phi(\vec{r},E) \phi^{\dagger}(\vec{r},E)
\label{eq:beckerconributon}
\end{equation}
and the spatially dependent contributon flux:
\begin{equation}
\tilde{\phi^c}(\vec{r}) = C_{norm}\int_0^{\infty } \phi^c(\vec{r},E) dE
\label{eq:beckerconributonspace}
\end{equation}
where the normalization constant, $C_{norm}$ for a given detector volume $V_D$
\begin{equation}
C_{norm} = \frac{V_D}{\int_{V_D}\int_0^{\infty } \phi^c(\vec{r},E) dE dV}
\end{equation}
The space- and energy- dependent weight windows are given by:
\begin{equation}
\bar{w}(\vec{r},E) = \frac{B(\vec{r}}{\phi^{\dagger}(\vec{r},E)}
\label{eq:beckerlocalww}
\end{equation}
where
\begin{equation}
B(\vec{r}) = \alpha(\vec{r}) \tilde{\phi^c}(\vec{r}) + 1 -  \alpha(\vec{r})
\end{equation}
and
% \begin{equation}
% \alpha(\vec{r}) = \left[ 1 + exp \left(  % \frac{\tilde{\phi^c_{max}}}{\tilde{\phi^c}(\vec{r})} - % \frac{\tilde{\phi^c}(\vec{r})}{\tilde{\phi^c_{max}} \right) \right]^{-1}
% \label{eq:beckeralpha}
%\end{equation}
\end{subequations}
