The following literature review aims to contextualize the work described in this
dissertation within the realm of hybrid methods for deep-penetration
neutron transport. In doing
so, the pertinent theoretical information that is relevant to this
topic is described.
This description is supplemented by a
discussion of the various efforts to implement
these methods for applied problems, and the degree to which those efforts
succeeded.
First, a brief overview of variance reduction for Monte Carlo radiation transport
is described in Section \ref{sec:MCvar}.
Then, Section \ref{subsec:AutomatedMCVR} expands on the
various efforts to automate variance reduction techniques in Monte Carlo.
Section \ref{sec:Importance} follows with an introduction of the concept of
importance and how that relates to variance reduction. This section also
focuses specifically on how the adjoint solution of the neutron transport equation
relates to importance.

From this point, the chapter transitions
from theory into existing implementations
of variance reduction techniques used in modern software in the nuclear
engineering community. Beginning in Section \ref{sec:CADIS}, a description of
the consistent, adjoint-driven importance sampling method, or CADIS, which has
been optimized for variance reduction of local solutions is presented.
Next, Section
\ref{sec:GlobalVR} discusses the methods implemented to reduce the variance
for global solutions. This discussion includes a description of
the forward-weighted CADIS (FW-CADIS) method.
The last section, \ref{sec:AngleVR}, details the efforts to incorporate
angular information into variance reduction methods for Monte Carlo.
Sections \ref{sec:CADIS}-\ref{sec:AngleVR} are each
concluded with a description of the
various software in which these methods have been implemented and the degree to
which they improved the variance reduction for their target applications.
