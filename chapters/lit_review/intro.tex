The following literature review aims to contextualize the work described in this
dissertation within the realm of hybrid methods for deep-penetration
neutron transport. In doing
so, it will describe pertinent theoretical information that is relevant to this
topic. This will be supplemented by a discussion of the various efforts
to use these methods practically and the degree to which those methods succeeded.
First a brief overview of variance reduction for monte carlo radiation transport
will be described in section \ref{sec:MCvar}.
This will then transition in section \ref{sec:AutomatedVR} into an elaboration
of various efforts to automate variance reduction techniques within monte carlo.
Following that, section \ref{sec:Importance} will introduce the concept of
importance and how that relates to variance reduction. This section will also
focus specifically on how the adjoint solution of the neutron transport equation
relates to importance.
From this point, the chapter will transition
from theory into existing implementations
of variance reduction techniques used in modern software in the nuclear
engineering community. This will begin in \ref{sec:CADIS} with a description of
the consistent, adjoint-driven importance sampling, or CADIS, and
forward-weighted CADIS (FW-CADIS) methods.
The following section, \ref{sec:AngleVR}, will detail the efforts to incorporate
angular information into variance reduction methods for Monte Carlo.
The sections on CADIS, FW-CADIS, and angle-specific variance reduction
techniques will be concluded with with a description of the
various software in which these methods have been implemented and the degree to
which they have been successful.
