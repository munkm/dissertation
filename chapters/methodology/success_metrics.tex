\section{Computational Success Metrics}
\label{sec:successmetrics}
\subsection{Anisotropy Quantification}

As the $\Omega$-methods are analyzed, it will be important to determine the
types of problems in which the methods are successful. In addition to describing
the physics that induce anisotropy in the flux, quantifying the degree of
anisotropy of the problem will be useful in characterizing the method. In this
section, a number
of methods by which the anisotropy can be quantified in these problems are
proposed. A brief description of how these methods capture anisotropy in the
problem is also included. While each method proposes a metric by which the
problem can be compared, there are certainly other methods that one may propose
that could be used as an alternative. The methods described in the following
subsections were chosen to reflect a relatively low computational cost
informative quantity.

\subsubsection{The Scalar Contributon Ratio}

If response is selected as an output option in ADVANTG, a silo file containing
the contributon fluxes for each discretized cell in space and energy  is
created after
each simulation run. This is useful for problem analysis as the user may see
preferential streaming paths for particles in the problem using this metric.
However, the contributon flux generated in this process is given by the product
of the scalar adjoint and forward fluxes (Eq. \eqref{eq:adv_contributon}).

As mentioned in SECTIONREF HERE, the contributon flux can be calculated by using
the product of the forward and adjoint fluxes. In methods like METHOD REFERENCE
and METHOD REFERENCE, the scalar contributon flux is calculated by the product
of the scalar adjoint and forward fluxes.

\begin{equation}
  \phi^{c}(\vec {r} ,E)  = \phi^{\dagger}(\vec {r} ,E)\phi(\vec {r} ,E)
\label{eq:adv_contributon}
\end{equation}

A more precise calculation of the contributon flux could be generated from
integrating the angular contributon flux over all angle.

\begin{equation}
  \begin{split}
    \Phi^{c}(\vec {r} ,E)  & = \int_{\Omega}{\psi^{c}(\vec{r}, E,\Omega)}
                               d\Omega \\
             & = \int_{\Omega}{\psi^{\dagger}(\vec{r}, E, \hat\Omega)
                 \psi(\vec{r}, E, \hat\Omega)} d\Omega
  \end{split}
\label{eq:full_contributon}
\end{equation}

Equation \ref{eq:full_contributon} incorporates the angular contributon flux
to calculate the scalar contributon flux
while Eq. \ref{eq:adv_contributon} utilizes the product of the scalar
adjoint and forward fluxes to calculate the scalar contributon flux. While both
equations give the contributon flux as a function of space and energy, the
differences in their calculation is addressed in their notation, namely using
$\phi^{c}$ or $\Phi^{c}$.

\begin{equation}
  M_{1} = \frac{\phi^{c}(\vec {r} ,E)}{\Phi^{c}(\vec {r} ,E)}
  \label{eq:metric_one}
\end{equation}

The first measure of anisotropy quantification that will be investigated is the
ratio between these two quantities, as described by Eq. \ref{eq:metric_one}.
If the adjoint or forward angular flux is significantly
peaked in $\Omega$, this will result in a deviation between $\phi^{c}$ and
$\Phi^{c}$. The more isotropic the flux in $\vec{r}$ and $E$, the closer these
values will be and the quantity will approach unity.

\subsubsection{The Maximum to Average Flux Ratio}

An alternative metric to quantify anisotropy is to calculate the ratio between
the maximum and average angular $\Omega$-flux in each $\vec{r} , E$ voxel (Eq.
\ref{eq:metric_two}).
The higher this quantity, the more peaked the angular $\Omega$-flux is in $\Omega$.

\begin{equation}
  M_{2} = \frac{\psi_{\Omega, Max}^{\dagger}(\vec {r} ,E)}{\psi_{\Omega,
                 Avg}^{\dagger}(\vec {r} ,E)}
  \label{eq:metric_two}
\end{equation}

While \ref{eq:metric_two} directly measures the anisotropy in the problem using
the $\Omega$-fluxes, it doesn't compare the difference between the fluxes used
in the $\Omega$ adjoint and the standard adjoint methods. Metric two can be
reformulated to incorporate this information using

\begin{equation}
  \begin{split}
    M_{3} & = \frac{\frac{\psi_{\Omega, Max}^{\dagger}(\vec {r} ,E)}{\psi_{\Omega,
                  Avg}^{\dagger}(\vec {r} ,E)}}{\frac{\psi_{\Omega,
                  Max}(\vec {r} ,E)}{\psi_{\Omega, Avg}(\vec {r} ,E)}} \\
          & = \frac{M_{2}}{\frac{\psi_{\Omega, Max}(\vec {r} ,E)}
                  {\psi_{\Omega, Avg}(\vec {r} ,E)}}
  \end{split}
  \label{eq:metric_three}
\end{equation}

as a measure between these standard and $\Omega$ fluxes.
This equation is a logical progression from \ref{eq:metric_two}. Here the ratio
of the max to average -$\Omega$ fluxes are compared to the max to average of the
original adjoint fluxes. In this metric more information on how perturbed the
$\Omega$-flux is when compared to the original adjoint flux that is normally
used in CADIS and FW-CADIS is reflected in the metric.
As a result, a strongly anisotropic
forward flux would significantly change the distribution of the $\Omega$-flux,
thus deviating this ratio from unity. In a region where the forward flux is low,
the $\Omega$ flux will match the original adjoint flux, and the value will be
unity.

Both equations \ref{eq:metric_two} and \ref{eq:metric_three} compare the maximum
angular flux in a cell to the average flux in the same cell. Because the average
angular flux is the normalization factor, the maximum flux in the cell is
compared to some relative measure of the total flux behavior in that cell. If,
for example, the flux has several directional peaks, the average will reflect
that.

\subsubsection{The Maximum to Minimum Flux Ratio}

An additional metric to quantify anisotropy in the $\Omega$ flux distribution
is to calculate the ratio between the maximum and minimum angular fluxes for
each $\vec{r}, E$ voxel, as described in metric four \ref{eq:metric_four}. This
quantity incorporates information about the
behavior of the maximum relative to the local minimum angular flux in each cell.

\begin{equation}
  M_{4} = \frac{\psi_{\Omega, Max}^{\dagger}(\vec {r} ,E)}{\psi_{\Omega,
                 Min}^{\dagger}(\vec {r} ,E)}
  \label{eq:metric_four}
\end{equation}

This metric may be more appropriate to describe the anisotropy of the flux in
cells where the distribution of flux values in the cell are not distributed in a
manner where the average is a good measure for the global behavior of the flux
in that cell. As with metric two \eqref{eq:metric_two}, metric four
\eqref{eq:metric_four} only quantifies
the anisotropy of the $\Omega$ flux relative to the $\Omega$-flux behavior in
that cell. To compare it to the anisotropy of the flux in the standard adjoint
problem, a ratio similiar to that of Eq. \ref{eq:metric_three} may be
formulated:

\begin{equation}
  \begin{split}
    M_{5} & = \frac{\frac{\psi_{\Omega, Max}^{\dagger}(\vec {r} ,E)}{\psi_{\Omega,
                  Min}^{\dagger}(\vec {r} ,E)}}{\frac{\psi_{\Omega,
                  Max}(\vec {r} ,E)}{\psi_{\Omega, Min}(\vec {r} ,E)}} \\
        & = \frac{M_{4}}{\frac{\psi_{\Omega, Max}(\vec {r} ,E)}
                  {\psi_{\Omega, Min}(\vec {r} ,E)}} .
  \end{split}
  \label{eq:metric_five}
\end{equation}

As with Eq. \ref{eq:metric_three}, Eq. \ref{eq:metric_five} uses a ratio from
the standard adjoint formulation to normalize the anisotropy of the
$\Omega$-flux. However, Eq. \ref{eq:metric_five} utilizes the maximum to minimum
ratio of angular fluxes of both the $\Omega$- and adjoint fluxes. These two
metrics will show the relative behavior of the flux in the cell, but because it
does not incorporate any information about the total flux behavior within the
cell, it may be very sensitive to the variance of the angular flux within the
cell. Using the ratio of both the $\Omega$- and adjoint fluxes may
help to smooth this if the variance of flux distributions within the $\Omega$-
and standard adjoint is similar in an $\vec{r}, E$ voxel. However, if these two
differ significantly, then metric five \eqref{eq:metric_five} may have a synergistic effect
and will over emphasize the variance when quantifying the anisotropy of the
cell.

\subsubsection{The Contributon Flux Anisotropy}

The last group of metrics that can be used to quantify anisotropy in the problem
are formulated using the contributon angular flux. As with Eqs.
\ref{eq:metric_two} and \ref{eq:metric_four}, Eqs \ref{eq:metric_six} and
\ref{eq:metric_seven} utilize the ratio of the maximum to average and maximum to
minimum angular fluxes to quantify anisotropy. In formulating these ratios with
the contributon flux, however, the anisotropy of the angular flux is quantified.
The difference in this formulation  will reveal regions of strong anisotropy
in the contributon flux, rather than the $\Omega$-flux,
which will help to inform in which regions both the forward and adjoint angular
fluxes are anisotropic. Incorporating information into the metrics on where the
forward angular flux is also anisotropic may make this metric more precise in
predicting the $\Omega$-method success, as strong streaming directional
importance in the forward flux will change the $\Omega$ method's performance,
but is not explicitly captured in the other metrics.

\begin{equation}
  M_{6} = \frac{\psi^{c}_{Max}(\vec {r} ,E,\Omega)}{\psi^{c}_{Avg}(\vec {r}
                ,E,\Omega)}
  \label{eq:metric_six}
\end{equation}

\begin{equation}
  M_{7} = \frac{\psi^{c}_{Max}(\vec {r} ,E,\Omega)}{\psi^{c}_{Min}(\vec {r}
                ,E,\Omega)}
  \label{eq:metric_seven}
\end{equation}

Metrics one through seven quantify anisotropy
in the problem solved by using different parameters to capture the problem
physics. These metrics will be compared to one another to determine which is the
most consistently correlated with predicting the $\Omega$-method's success. A
user may want to know if the $\Omega$-method will effectively generate variance
reduction parameters for a Monte Carlo simulation, and this may be a
prescriptive solution for that issue. However, all of these metrics do require
full angular flux solutions for both the forward- and adjoint- problem, so some
computational burden will be required to generate this solution metric for the
problem. That said, because the Monte Carlo solution is more computationally
demanding, generating these metrics from the deterministic solution should be
substantially less.

\subsection{Figure of Merit}
\subsection{Timing}
\subsection{Scalability}
