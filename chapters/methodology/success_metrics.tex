\newcommand{\anglevars}{x,y,z,E_g,\theta,\varphi}
\newcommand{\scalarvars}{x,y,z,E_g}


\section{Computational Success Metrics}
\label{sec:successmetrics}
\subsection{Anisotropy Quantification}
\label{sec:anisotropy_quant}

As the $\Omega$-methods are analyzed, it will be important to determine the
types of problems in which the methods are successful. In addition to describing
the physics that induce anisotropy in the flux, quantifying the degree of
anisotropy of the problem will be useful in characterizing the method. In this
section, a number
of methods by which the anisotropy can be quantified in these problems are
proposed. A brief description of how these methods capture anisotropy in the
problem is also included. While each metric proposes an avenue by which the
problem can be analyzed, there are certainly other methods that one may propose
that could be used as an alternative. The methods described in the following
subsections were proposed because they use data generated from the existing
method. The degree to which they impose a computational burden will be addressed
in their analysis.

\subsubsection{The Scalar Contributon Ratio}

The hybrid methods software that will be used for this project is ADVANTG,
developed at ORNL. Section \ref{sec:software} explains how the software used
interacts with other pieces of software
and how they were modified to execute this method.
The standard release of ADVANTG provides the contributon flux as an output
option, which can then be used to analyze problem physics by a user.
If this option selected as an output, a silo file containing
the contributon fluxes for each discretized cell in space and energy  is
created after
a run. This is useful for problem analysis as the user may see
preferential streaming paths for particles in the problem using this metric.
However, the contributon flux generated in this process is given by the product
of the scalar adjoint and forward fluxes (Eq. \ref{eq:adv_contributon}).

As mentioned in section \ref{sec:ContributonImportance}, the
contributon flux can be calculated by using
the product of the forward and adjoint fluxes. In standard software packages
that calculate the contributon flux, like ADVANTG, the scalar contributon flux is calculated by the product
of the scalar adjoint and forward fluxes. This can be written as
%
\begin{equation}
  \phi^{c}(\vec {r} ,E)  = \phi^{\dagger}(\vec {r} ,E)\phi(\vec {r} ,E) .
\label{eq:adv_contributon}
\end{equation}
%
A more precise calculation of the contributon flux could be generated from
integrating the angular contributon flux over all angle, as
%
\begin{equation}
  \begin{split}
    \Phi^{c}(\vec {r} ,E)  & = \int_{\Omega}{\psi^{c}(\vec{r}, E, \hat \Omega)}
                               d\hat\Omega \\
             & = \int_{\Omega}{\psi^{\dagger}(\vec{r}, E, \hat\Omega)
                 \psi(\vec{r}, E, \hat\Omega)} d\hat\Omega .
  \end{split}
\label{eq:full_contributon}
\end{equation}

Both equations \ref{eq:adv_contributon} and \ref{eq:full_contributon} calculate
the contributon flux as a function of space and energy, but the
differences in their calculation is addressed in their notation, namely using
$\phi^{c}$ or $\Phi^{c}$. The standard release of ADVANTG only has
access to the scalar
fluxes, so Eq. \ref{eq:full_contributon} is not an accessible option for a
user. Because the $\Omega$ calculations require full angular flux maps for the
problems, the scalar contributon flux can be calculated with the latter
formulation, rather than the former in the modified version developed to support
this work.

The first measure of anisotropy quantification that will be evaluated is the
ratio between these two quantities, as described by Eq. \ref{eq:metric_one}.
The ratio between these two values is evaluated for every cell, $x,y,z$, and energy
group, $E_g$.
If the adjoint or forward angular flux is significantly
peaked in $\Omega$, this will result in a deviation between $\phi^{c}$ and
$\Phi^{c}$, because there will be a multiplicative effect in the angular flux
captured in $\Phi^{c}$ but not $\phi^{c}$. The more isotropic the
flux in $\vec{r}$ and $E$, the closer these
values will be and the quantity will approach unity.

\begin{equation}
  % M_{1} = \frac{\phi^{c}_{\vec {r} ,E}}{\Phi^{c}_{\vec {r} ,E}}
  M_{1} = \frac{\phi^{c}}{\Phi^{c}}\bigg\rvert_{\scalarvars}
  \label{eq:metric_one}
\end{equation}

\subsubsection{The Ratio of Adjoint Fluxes}

As discussed in previous sections, the $\Omega$-methods use the omega- scalar
flux in place of the standard adjoint scalar flux. Therefore the ratio between
these two quantities would also provide a useful metric for comparing which
regions have significantly differing bias parameters in standard-adjoint and
omega-adjoint situations. This metric will deviate from unity if the forward
flux is anisotropic. This metric is calculated for
every cell and every energy group in the problem, as shown in Eq.
\ref{eq:metric_two}.
%
\begin{equation}
  M_{2} = \frac{\phi^{\dagger}_{\Omega}}{\phi^{\dagger}}\bigg\rvert_{\scalarvars}
  \label{eq:metric_two}
\end{equation}
%
Metrics one and two both reasonably appear to compute the anisotropy in the flux
using versions of the contributon and adjoint fluxes, respectively. However, by
expanding the $\Omega$-adjoint scalar flux in Metric two,
%
\begin{align*}
% \begin{equation*}
%  \begin{split*}
  M_{2} & = \frac{\phi^{\dagger}_{\Omega}}{\phi^{\dagger}}\bigg\rvert_{\scalarvars} \\
  & = \frac{\int_{\Omega}{\psi^{\dagger}(\hat\Omega) \psi(\hat\Omega)}
      d\hat{\Omega}}{\int_{\Omega}{\psi(\hat\Omega) d\hat{\Omega}}}
           \frac{1}{\phi^{\dagger}}\bigg\rvert_{\scalarvars} \\
\intertext{integrating the forward angular flux over all angle,} \\
  & = \frac{\int_{\Omega}{\psi^{\dagger}(\hat\Omega) \psi(\hat\Omega)}
      d\hat{\Omega}}{\phi}
           \frac{1}{\phi^{\dagger}}\bigg\rvert_{\scalarvars} \\
\intertext{and rearranging the terms,}\\
  & = \frac{\int_{\Omega}{\psi^{\dagger}(\hat\Omega) \psi(\hat\Omega)}
      d\hat{\Omega}}{\phi \phi^{\dagger}}\bigg\rvert_{\scalarvars}\\
  & = \frac{\Phi^{c}}{\phi^{c}}\bigg\rvert_{\scalarvars}\\
  & = \frac{1}{M_{1}}\bigg\rvert_{\scalarvars},
%  \end{split*}
% \end{equation*}
\end{align*}
%
it becomes evident that the ratio of adjoint fluxes is the inverse of the scalar
contributon ratio. As a result, metric one will not be used in the analyses of
the characterization problems.


\subsubsection{The Maximum to Average Flux Ratio}

An alternative metric to quantify anisotropy is to calculate the ratio between
the maximum and average angular contributon flux in each $\vec{r} , E$ voxel.
The higher this quantity, the more peaked the contributon flux is in $\Omega$.
Note here that while using the $\Omega$-flux would seem like the natural choice,
no angular information is directly accessible once the $\Omega$ scalar flux has
been calculated. One can compare the standard adjoint scalar flux and the omega
adjoint scalar flux and infer how anisotropic the flux in the cell might be, but
due to the normalization that occurs in Eq. \ref{eq:omega_basic}, the
variation of angular omega fluxes throughout $\Omega$ for a cell
in $\scalarvars$ is not
calculated. As such, the contributon flux must be relied
upon as a next-best evaluator of that metric:
%
\begin{equation}
  M_{3} = \frac{\psi^{c}_{Max}}{\psi^{c}_{
          Avg}}\bigg\rvert_{\scalarvars}  .
  \label{eq:metric_three}
\end{equation}

While \ref{eq:metric_three} directly measures the anisotropy in the problem using
the angular contributon fluxes, it doesn't compare the difference
between the fluxes used
in the $\Omega$-  and the standard adjoint methods. Metric three
can be
reformulated to incorporate this information using
%
\begin{equation}
  \begin{split}
    M_{4} & = \frac{\frac{\psi^{c}_{Max}}{\psi^{c}_{
                  Avg}}}{\frac{\psi^{\dagger}_{
                  Max}}{\psi^{\dagger}_{Avg}}} \Bigg\rvert_{\scalarvars} \\
          & = \frac{M_{3}}{\frac{\psi^{\dagger}_{Max}}
                  {\psi^{\dagger}_{Avg}}} \Bigg\rvert_{\scalarvars}
  \end{split}
  \label{eq:metric_four}
\end{equation}
%
as a measure between the anisotropies of the standard and contributon fluxes.
This equation is a logical progression from metric two and
metric three. This metric contains more information on how perturbed the
contributon flux is when compared to the original adjoint flux that is normally
used in CADIS and FW-CADIS.
In the case of a strongly anisotropic
forward flux, the forward flux  would significantly
change the distribution of the contributon
fluxes in a cell, but it would not affect the flux distribution of the standard
adjoint angular fluxes. By comparing the anisotropy in the contributon
fluxes to those in the standard adjoint, the perturbation of the omega flux by
the forward flux
in the cell can be evaluated. In regions where the forward flux is
not anisotropic, then the contributon anisotropy ratio should approximately be
the same as the standard adjoint anisotropy ratio.

Further, becuase the
contributon flux incorporates directionality of the forward and adjoint fluxes,
the maximum to average ratio of the contributon flux can differ from the adjoint
flux. In regions where the adjoint
angular flux and the forward angular flux are traveling in the same direction,
the contributon ratio should be greater than the adjoint ratio, and this metric
will be greater than one. In regions where they are travelling in opposite or
perpindicular directions, the contributon flux will evaluate to a more isotropic
state, and metric four will be less than unity. This metric
provides substantially more information than metric two because
it compares the behavior of the directional contributon and adjoint fluxes,
rather than comparing the overall behavior of the flux in the cell.

Both equations \ref{eq:metric_three} and \ref{eq:metric_four} compare the maximum
angular flux in a cell to the average flux in the same cell. Because the average
angular flux is the normalization factor, the maximum flux in the cell is
compared to some relative measure of the total flux behavior in that cell. If,
for example, the flux has several directional peaks, the average will reflect
that.
The fact that Eq. \ref{eq:metric_four} contains information on the global
behavior in the contributon and average cell, the directionality of the fluxes,
and the degree of isotropy of the forward flux is attractive. However,
this is also a fairly computationally expensive calculation and it may
not be worth the computational cost when compared to metrics two and three.

\subsubsection{The Maximum to Minimum Flux Ratio}

An additional metric to quantify anisotropy in the contributon flux distribution
is to calculate the ratio between the maximum and minimum angular fluxes for
each region of $\scalarvars$ phase-space, as described in metric
four, or Eq. \ref{eq:metric_five}. This
quantity incorporates information about the
behavior of the local maximum relative to the local
minimum angular flux in each cell.

\begin{equation}
  M_{5} = \frac{\psi^{c}_{Max}}{\psi^{c}_{
          Min}}\bigg\rvert_{\scalarvars}  .
  \label{eq:metric_five}
\end{equation}

This metric may be more appropriate to describe the anisotropy of the flux in
cells where the distribution of flux values in the cell are not well
reflected by the average flux in the cell. As with metric three \eqref{eq:metric_three}, metric five
\eqref{eq:metric_five} only quantifies
the anisotropy of the contributon flux in the cell. There is no comparison or
normalization to compare the anisotropy with respect to another method. To
compare it to the anisotropy of the flux in the standard adjoint
problem, a ratio similar to that of Eq. \ref{eq:metric_four} may be
formulated:

\begin{equation}
  \begin{split}
    M_{6} & =  \frac{\frac{\psi^{c}_{Max}}{\psi^{c}_{
                  Min}}}{\frac{\psi^{\dagger}_{
                  Max}}{\psi^{\dagger}_{Min}}} \Bigg\rvert_{\scalarvars} \\
          & = \frac{M_{5}}{\frac{\psi^{\dagger}_{Max}}
                  {\psi^{\dagger}_{Min}}} \Bigg\rvert_{\scalarvars}  .
  \end{split}
  \label{eq:metric_six}
\end{equation}

As with Eq. \ref{eq:metric_four}, Eq. \ref{eq:metric_six} uses a ratio from
the standard adjoint formulation to normalize the anisotropy of the
contributon flux. However, Eq. \ref{eq:metric_six} is consistent with Eq.
\eqref{eq:metric_five} and normalizes using the maximum to minimum
ratio of angular fluxes of the adjoint. These two
metrics will show the relative behavior of the flux in the cell, but because it
does not incorporate any information about the total flux behavior within the
cell, it may be very sensitive to the variance of the angular flux within the
cell. Using the ratio of both the contributon and adjoint fluxes may
help to smooth this if the variance of flux distributions within the contributon
and standard adjoint is similar a particular cell. However, if these two
differ significantly, then metric six \eqref{eq:metric_six} may have a
synergistic effect
and will over emphasize the variance when quantifying the anisotropy of the
cell.

Metrics one through six quantify anisotropy
in the problem solved by using different parameters to capture the problem
physics. These metrics will be compared to one another to determine which is the
most consistently correlated with predicting the $\Omega$-method's success. A
user may want to know if the $\Omega$-method will effectively generate variance
reduction parameters for a Monte Carlo simulation, and this may be a
prescriptive solution for that issue. However, all of these metrics do require
full angular flux solutions for both the forward- and adjoint- problem, so some
computational burden will be required to generate this solution metric for the
problem. The analysis of using these metrics will include some information of
benefit to burden, which likely will come at the cost of time.
That said, because the Monte Carlo solution is more computationally
demanding, generating these metrics from the deterministic solution should be
substantially less of an obstacle.

\subsection{Figure of Merit}

The figure of merit (FOM) is a commonly used metric to measure Monte Carlo
runtimes and to gauge the effectiveness of various hybrid methods. As discussed
in section \ref{sec:MCvar}, the FOM relates the relative error of a solution to the
time required required to achieve that variance. This was introduced in Eq.
\ref{eq:FOM} as:
\begin{equation*}
  \text{FOM} = \frac{1}{R^{2}T} ,
\end{equation*}
where $T$ is the time and $R^{2}$ is the square of the relative error.

\subsubsection{Relative Error}

In tallies with multiple regions and/or energy bins, the FOM is normally calculated
from the tally average relative error, or $R_{avg}$. This value is meaningful as
it reflects the overall tally behavior. However, it is often desirable that all
portions of the tally lie below a desired relative error threshold. A region
with very low particle contribution may have a much higher relative error than
the tally average, and may also converge much slower to a desired relative
error. This results in a substantially different FOM than
the tally average. In the results presented in later chapters, both relative
errors will be used to calculate different FOMs, respectively
\begin{subequations}
  \begin{equation}
    \text{FOM}_{avg} = \frac{1}{R_{avg}^{2}T}
  \label{eq:FOMavg}
  \end{equation}
and
  \begin{equation}
    \text{FOM}_{max} = \frac{1}{R_{max}^{2}T} .
  \label{eq:FOMmax}
  \end{equation}
  \label{eq:FOMerror}
\end{subequations}

In addition to reporting both FOMs for the entire problem, comparing the
distribution of values of the relative error for problems will be a useful
metric in method characterization. If, for example, FW-CADIS works successfully
in a problem then the problem should have a relatively even uncertainty
distribution for all cells. Comparing the distribution of relative errors
between the analog case and the hybrid cases reveals whether the method is
effectively generating variance reduction parameters for the entire problem or
if it is more effective in particular regions.

\subsubsection{Timing}

The previous section described two different means by which the FOM could be
calculated using different relative errors. The question that one must
now consider is: what time should be used to
calculate the FOM? In an analog Monte Carlo simulation, this time is the runtime
of the Monte Carlo simulation, $T = T_{MC}$. In a hybrid method, one could choose
either
\begin{equation}
  T_{Hybrid} = T_{MC} + T_{Deterministic}
\label{eq:hybridtime}
\end{equation}
or
\begin{equation}
  T_{Hybrid} = T_{MC} .
\end{equation}

The FOM should remain a constant--with the exception of very early on in an MC
calculation where statistics are very poor--for a problem. The issue with using
Eq. \ref{eq:hybridtime} to calculate the FOM is that the deterministic runtime
does not change the relative error of the Monte Carlo simulation. Thus, the FOM
is not a constant throughout the Monte Carlo Simulation when using Eq.
\ref{eq:hybridtime} as the time.
However, it would be disingenuous to not include the deterministic runtime into
reports for the hybrid method, as the total computational time required to
achieve some desired relative error is ultimately what the user is seeking. As
such, two reports of the FOM are included with the results for each simulation:
\begin{subequations}
  \begin{equation}
    \text{FOM}_{MC} = \frac{1}{R^{2}T_{MC}}
  \label{eq:FOMMC}
  \end{equation}
and
  \begin{equation}
    \text{FOM}_{Hybrid} = \frac{1}{R^{2}(T_{MC} + T_{Deterministc})} .
  \label{eq:FOMHybrid}
  \end{equation}
  \label{eq:FOMtime}
\end{subequations}
Note that the deterministic time used in Eq. \ref{eq:FOMHybrid} is the time only
to run the transport and generate source biasing and weight window values for
each problem. It will not include the time used to quantify the anisotropy as
outlined in section \ref{sec:anisotropy_quant}, as those parameters will be
computationally demanding but not normally included in a hybrid method
computation.

In this section four different equations calculate the figure of merit were presented: two
using different relative errors, and two using different quantities
for time. In analyzing the method, all four will be presented as FOM$_{MC,avg}$,
FOM$_{MC,max}$, FOM$_{Det,avg}$, and FOM$_{Det,max}$. Further, the improvement
in the FOM for each problem will be reported as those values normalized by
FOM$_{analog,avg}$ for the two FOMs calculated with the tally average relative
error and FOM$_{analog,max}$ for the FOMs calculated with the tally maximum
relative error. The success of the method will depend on its ability to improve
each one of these FOM values.

