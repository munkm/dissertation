\section{Software}
\label{sec:software}

In this section, the software in which the methods presented in Section
\ref{sec:omegaintro} are implemented is described.
A brief summary of each piece of
software and what was added in each is discussed.
While the details of the inner-workings of the software will not be described
here, both pieces of software have rich documentation and user guides which an
interested reader may reference.
% Consider maybe writing a brief summary of how ADVANTG and Denovo interact
% together for a particular problem in this intro.
% Rachel, what do you think?

\subsection{Denovo}

Denovo $S_N$ is a three-dimensional discrete ordinates transport solver
developed at Oak Ridge National Laboratory \cite{evans_denovo:_2010}. Denovo is a module in the larger
Exnihilo massively-parallel radiation trasnsport code suite. There exist several
other modules in Exnihilo. In addition to Denovo, the most pertinent package being
Omnibus, a frontend pre- and post-processing module. The
$\Omega$-fluxes are generated by running two independent (a forward and an
adjoint) determinstic solves
in Denovo. The setup and generation of each
simulation input is automated through ADVANTG (see Section \ref{sec:advantg}).
After the calculation has reached the desired convergence criteria, the
full angular flux maps for the
forward and adjoint solves are saved to an HDF5 \cite{hdf5} file. Denovo was
modified to output the full angular flux maps for a simulation. The
$\Omega$-fluxes are then generated by passing the angular flux maps through the
postprocessing module in Omnibus. Using this module, the integration described in Eq.
\eqref{eq:omega_basic} is performed, the scalar $\Omega$-fluxes are saved
to a SILO file, and the scalar fluxes are passed to ADVANTG for
variance reduction parameter
generation. Appendix \ref{sec:omega_code} contains the code added to Omnibus
to perform this calculation.

\subsection{ADVANTG}
\label{sec:advantg}

ADVANTG \cite{mosher_new_2010} is a software package
originally designed to automatically
generate variance reduction
parameters for the Monte Carlo radiation transport solver
MCNP \cite{brown_mcnp_2002}
using the CADIS
and FW-CADIS methods. For this project, the ADVANTG functionality was extended to
process the $\Omega$-fluxes provided by Denovo through Omnibus and to generate
variance reduction
parameters for CADIS and FW-CADIS using said fluxes. In addition to the
modifications required to perform CADIS and FW-CADIS, ADVANTG was further modified
to generate anisotropy quantification metrics and a modified version of
the scalar contributon flux, both of which were summarized in
Section \ref{sec:anisotropy_quant}. The piece of code used to
reroute the $\Omega$-fluxes
through CADIS and FW-CADIS as well as to generate the anisotropy metrics in
ADVANTG is included in Appendix \ref{sec:anisotropy_code}.
