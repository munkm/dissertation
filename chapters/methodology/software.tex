\section{Software}
\label{sec:software}

In this section the software in which the methods presented in section
\ref{sec:omegaintro} will be implemented. A brief summary of each piece of
software and what was added in each will be presented.
While the details of the inner-workings of the software will not be described
here, both pieces of software have rich documentation and user guides which an
interested reader may pursue at their leisure.
% Consider maybe writing a brief summary of how ADVANTG and Denovo interact
% together for a particular problem in this intro.
% Rachel, what do you think?

\subsection{The Denovo Discrete Ordinates Software}

Denovo $S_N$ is a three-dimensional discrete ordinates transport solver
developed at Oak Ridge National Laboratory \cite{evans_denovo:_2010}. Denovo is a module in the larger
Exnihilo code suite, which in addition to Denovo,
includes: Omnibus, a frontend pre- and post-processing module; Transcore,
components used for radiation transport codes, including cross sections and
quadrature sets; Nemesis, software infrastructure and general algorithms;
Geometria, the geometry packages; Shift, a Monte Carlo radiation transport
solver; and Insilico, the front end used to couple Denovo and Shift. The
$\Omega$-fluxes are generated by running two independent-a forward and an
adjoint-determinstic solves in Denovo. The setup and generation of each
simulation input is automated through ADVANTG (see Section \ref{sec:advantg}).
After the calculation has reached the desired convergence criteria, the
full angular flux maps for the
forward and adjoint solves are saved to an hdf5 \cite{hdf5} file. Denovo was
modified to output the full angular flux maps for a simulation. The
$\Omega$-fluxes are then generated by passing the angular flux maps through the
postprocessing module in Omnibus, the integration described in Eq.
\ref{eq:omega_basic} is performed, and then the scalar $\Omega$-fluxes are saved
to a .silo file and passed on to ADVANTG for variance reduction parameter
generation. Appendix \ref{sec:omega_code} contains the code added to Omnibus
used to perform this calculation.

\subsection{ADVANTG}
\label{sec:advantg}

ADVANTG \cite{mosher_new_2010} is a software package
originally designed to automatically
generate variance reduction
parameters for the Monte Carlo radiation transport solver
MCNP \cite{brown_mcnp_2002}
using the CADIS
and FW-CADIS methods. For this project, the ADVANTG functionality was extended to
process the omega fluxes provided by Denovo through Omnibus and to generate
variance reduction
parameters for CADIS and FW-CADIS using said fluxes. In addition to the
modifications required to perform CADIS and FW-CADIS, ADVANTG was further modified
modified to generate anisotropy quantification metrics and a modified version of
the scalar contributon flux, both of which were summarized in
section \ref{sec:anisotropy_quant}. The piece of code used to
reroute the $\Omega$-fluxes
through CADIS and FW-CADIS as well as to generate the anisotropy metrics in
ADVANTG is included in Appendix \ref{sec:anisotropy_code}.
