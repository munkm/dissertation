\section*{Summary}

In summary, this chapter presented the novel theory behind the $\Omega$-methods;
the metrics by which the $\Omega$-methods will be compared with existing hybrid
methods; and the software that was modified to implement the $\Omega$-methods
into an existing codebase. Two variants of the $\Omega$-methods were presented:
CADIS-$\Omega$ and FW-CADIS-$\Omega$, which are referred to together as
FW/CADIS-$\Omega$. CADIS-$\Omega$ is a modification of CADIS, and is designed to
generate VR parameters for local solutions in problems with strong anisotropy.
FW-CADIS-$\Omega$ is a modification of FW-CADIS, and is designed for generating
VR parameters for global solutions in problems with strong anisotropy.

Both CADIS-$\Omega$ and FW-CADIS-$\Omega$ are implemented in well-used,
well-documented, massively-parallel, state-of-the-art radiation
transport and hybrid methods software. The
radiation transport code suite Exnihilo is modified to generate the
$\Omega$-fluxes. The hybrid methods package ADVANTG is modified to generate
VR parameters for the $\Omega$-methods using the $\Omega$-fluxes.

To understand the performance of the $\Omega$-methods and compare it
consistently to existing methods, several performance metrics were proposed. First,
a few variants of the FOM were described. They include: FOM$_{MC,avg}$,
FOM$_{MC,max}$, FOM$_{hybrid,avg}$, FOM$_{hybrid,avg}$. Together, they provide
an overall picture of the performance of the $\Omega$-method's performance with
respect to relative error and time, rather than of a
single criteria. Because anisotropy has the ability to affect energy groups
differently, resulting in different relative errors achieved in different energy
bins, separating out different FOMs helps to isolate interesting behavior
in the methods.

The $\Omega$-methods are designed to work in problems with strong
anisotropies in the flux. As a result, several anisotropy metrics
with which to investigate flux anisotropy were proposed.
Using these metrics and comparing them to
the relative errors or FOMs in each
tally region, we can try to understand the effect that anisotropy has on the
$\Omega$-method performance. Each metric quantifies the anisotropy in cells
differently, so each has the potential to capture different information. Denovo
was modified to output angular fluxes to generate the $\Omega$-flux for the
$\Omega$-methods.
As a result, the anisotropy metrics use data generated from the existing
$\Omega$-method calculation.

Using the methodology described in this chapter, the $\Omega$-methods'
performance can be fully characterized. Further, the characterization presented
in this chapter has been
extended from standard FOM performance metrics to include anisotropy
quantification.
By implementing the $\Omega$-methods into production-level software, it is
accessible to any user beyond the author. The generation of the anisotropy
metrics is also incorporated into the codebase, meaning that any user could
feasibly perform an investigation of the $\Omega$-performance consistent with
what is proposed herein. The use of the various FOMs and of the aniostropy
metrics helps the understanding of the $\Omega$-method performance as a function
of time, error, and anisotropy.
