\section{Theory: Angle-Informed Importance Maps for CADIS and FW-CADIS}
\label{sec:methodtheory}

A number of methods have been developed to generate variance reduction parameters
for deep penetration radiation transporti problems. Section \ref{sec:omegabknd} will
briefly review the
relevant background work to support methods development of angle-informed variance reduction
parameters. Note that this is merely a summary of that supporting work. A more
thorough review of it is presented in Chapter \ref{sec:lit_review}. This work provided the foundation for the $\Omega$-methods, which
will be introduced in the following section, \ref{sec:omegaintro}. In
these sections the mathematical foundation for the $\Omega$-methods will be
presented and accompanied by a discussion of how these methods deviate from the
original formulations of CADIS and FW-CADIS, and other angle-informed methods.

\subsection{Previous Work}
\label{sec:omegabknd}

As discussed in sections \ref{sec:CADIS}
through \ref{sec:AngleVR}, the existing gold standard for automatically
generating variance reduction parameters for deep penetration
fixed-source radiation transport problems are
CADIS and FW-CADIS. Both of these methods are very effective at
generating variance reduction parameters for local and global solutions,
respectively. However, CADIS and FW-CADIS were developed for optimization in
space and energy, not angle.

A number of angle-informed automatic variance reduction methods have been
investigated in the past, most notably
* AVATAR
  * description of method
  * pro/con summary
* LIFT
  * description of method
  * pro/con summaryy

In an attempt to adapt CADIS and FW-CADIS to include angular information into
the VR parameters, Peplow et. al. formulated an adjustment to CADIS in the ORNL
code suite \cite{peplow_consistent_2012}. Two different primary
formulations were investigated:
CADIS with directional source biasing, and CADIS without directional source
biasing.

In section \ref{sec:resultsangle} the success of earlier methods
that were developed for automated variance reduction with angular information
was evaluated.

\subsection{The $\Omega$ Methods}
\label{sec:omegaintro}

The foundation of the $\Omega$-methods is built upon CADIS and FW-CADIS. Namely,
the $\Omega$-methods will use a version of then adjoint scalar flux to
consistently bias a monte carlo problem with the intention of reducing the
variance. In section \ref{sec:Importance} the concept of importance was
introduced. Notably, it was shown that the adjoint flux is a good marker for
importance for particles to contribute to a tally. It was also shown that the
product of the forward and adjoint flux generates a pseudo-particle flux called
the contributon flux, where contributons are ``importance particles''.
These importance particles can be used to show preferential flow paths from a
source to a tally or desired location.

By using a version of the adjoint scalar flux that has been formulated with the
contributon flux, the direction of particle flow is then incorporated into the
variance reduction parameters.

\begin{equation}
  \phi^{\dagger}_{\Omega}(\vec {r} ,E)  = \frac{\int_{\Omega}{\psi^{\dagger}
                             (\vec{r}, E, \hat\Omega)
                             \psi(\vec{r}, E, \hat\Omega)} d\Omega}
                             {\int_{\Omega}\psi(\vec{r}, E, \hat\Omega) d\Omega}
\label{eq:omega_basic}
\end{equation}

\subsubsection{CADIS-$\Omega$}
\label{sec:cadomega}


\begin{subequations}
\label{CADISomegamethod}
Biasing parameters used by CADIS-$\Omega$:
\begin{equation}
\begin{split}
  \hat{q}  & = \frac{\phi_{\Omega}^{\dagger}(\vec {r} ,E)q(\vec {r} ,E)}
               {\iint\phi_{\Omega}^{\dagger}(\vec {r} ,E)
               q(\vec {r} ,E) dE d\vec{r}} \\
           & = \frac{\phi_{\Omega}^{\dagger}(\vec {r} ,E)q(\vec {r} ,E)}{R}
\end{split}
\end{equation}
The  starting weights of the particles sampled from the
biased source distribution, $\hat{q}$ are:
\begin{equation}
\begin{split}
w_0  & = \frac{q}{\hat{q}} \\
     & = \frac{R}{\phi_{\Omega}^{\dagger}(\vec {r} ,E)}
\end{split}
\end{equation}
and the target weights for the particle are:
\begin{equation}
  \hat{w} = \frac{R}{\phi_{\Omega}^{\dagger}(\vec {r} ,E)}
\end{equation}
\label{eq:cadomega_eqns}
\end{subequations}

\subsubsection{FW-CADIS-$\Omega$}
\label{sec:fwcadomega}

FW-CADIS differs from CADIS in that it requires a forward deterministic
calculation to generate $q^{\dagger}$, rather than setting
$q^{\dagger}=\sigma_d$ as with standard CADIS. Depending on the type of global
response desired, FW-CADIS runs a deterministic forward calculation to
approximate the global response in the problem. The inverse of these responses
is then used to generate the biased adjoint source distribution for the adjoint
deterministic run. Therefore, the behavior of FW-CADIS-$\Omega$
in the forward biasing
portion of the calculation will remain the same as with FW-CADIS:

\begin{subequations}
\begin{equation}
{ q^{\dagger}} (P)=\frac{\sigma_d(P)}{R}
\end{equation}

The adjoint source for the spatially dependent global dose, $\int \phi(\vec{r},E)\sigma_d(\vec{r},E) dE$:
\begin{equation}
{ q^{\dagger} }(\vec { r } ,E)=\frac { \sigma _{ d }(\vec { r } ,E) }{ \int { \sigma _{ d }(\vec { r } ,E)\psi (\vec { r } ,E,) } dE }
\end{equation}

The adjoint source for the spatially dependent total flux $\int \phi(\vec{r},E) dE $:
\begin{equation}
{ q^{\dagger} }(\vec { r })=\frac { 1 }{ \int { \phi (\vec { r } ,E) } dE }
\end{equation}

The adjoint source for the energy- and spatially- dependent flux $\phi(\vec{r},E)$:
\begin{equation}
{ q^{\dagger} }(\vec { r } ,E)=\frac { 1 }{\phi (\vec { r } ,E) }
\end{equation}
\label{eq:fwcadomega_eqns}
\end{subequations}

One advantage of FW-CADIS-$\Omega$ is that, transport-focused, the
$\Omega$-method is no more expensive than standard FW-CADIS. Because both
versions require a forward and adjoint deterministic calculation, an extra
transport step is not required as it is for CADIS-$\Omega$. This is attractive,
but the nature of FW-CADIS might not be the most well-suited for the
$\Omega$-methods. Because FW-CADIS attempts to evenly distribute particles
throughout the problem using the forward-biased adjoint fluxes,
the additional forward normalization with the $\Omega$-methods will likely skew
the particle distribution in the problem, and it may place too great of
importance on the forward-moving particles in generating the variance reduction
parameters.
