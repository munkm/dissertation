\section{Theory: Angle-Informed Importance Maps for CADIS and FW-CADIS}
\label{sec:methodtheory}

Methods exist that have been developed to generate
variance reduction parameters
for deep penetration radiation transport problems with strong anisotropy in the
flux. These methods have shown to have varying success, and may not be fully
automated. The solution proposed in this dissertation is a formulation that we
have named the $\Omega$-methods. 
% Maybe \Omega-CADIS-methods? \Omega-methods seems insufficiently specific...?
The foundational research
on which the $\Omega$-methods are built will be briefly described to begin this
section. That discussion will serve as a primer for the subsequent section, which
is an introduction to the $\Omega$-methods and a discussion on how they differ
from their predecessors.

\subsection{Previous Work}
% I this whole section seems to be lit review. I'm not clear about why it is here.
% I haven't read the lit review section, so maybe the framing makes more sense. 
% This is why the big outline is helpful...
\label{sec:omegabknd}

%A thing to note and think carefully about: we probably want to use the words
%"speed up" about improving calculations. Both acceleration and optimization
%have specific mathematical meaning; VR methods typically don't do either. I
%bring this up bc it's something I did not used to be careful about and I got
%dinged about it by reviewers. If we are careful with wording now then we won't
%have to worry about anyone being nit-picky later (computational science; all
%about precision...)
As discussed in Sections \ref{sec:CADIS}
through \ref{sec:AngleVR}, the existing gold standard for automatically
generating variance reduction parameters for deep penetration
fixed-source radiation transport problems are
CADIS and FW-CADIS. Both of these methods are very effective at
generating variance reduction parameters for local and global solutions,
respectively. However, CADIS and FW-CADIS have only been implemented to
do variance reduction in space and energy, not angle. 
% This isn't true. The original documents talk about how it can easily be extended to include angle.
% It is simply that they have only been implemented for space and energy for a bunch of reasons. 
As a result, solutions for problems with
strong anisotropy in the flux are not always improved very efficiently with these methods,
resulting low Figure of Merit (FOM) values.
To address this, a number of angle-informed variance reduction
methods have been
investigated in the past, most notably AVATAR, LIFT, and a modified version of
CADIS using AVATAR-type angular parameters.
% Note: some of these methods were invented -before- CADIS and FW-CADIS. 
% I think you need to reframe this part as ``we need VR methods that work for problems
% with strong anisotropies; here are some past attempts.''
% However, all of that is lit-review discussion... To have it be here you 

The AVATAR method
\cite{van_riper_generation_1995, van_riper_avatarautomatic_1997} used an
approximation of the angular flux--without explicitly calculating it--to
generate angle-dependent weight windows. It operated under the approximation
that
the angular flux was separable and symmetric about the average current vector.
The angular flux was then calculated using
a product of a deterministically-obtained
scalar flux and an exponential function, derived from the
maximum entropy distribution, that was a function of the scalar flux and the
current. Space-, energy-, and angle-dependent weight windows for
the Monte Carlo problem were then generated from the inverse of the angular
flux. AVATAR improved the FOM for sample problems from 2 to 5 times, but did not
apply to problems where the flux was not azimuthally symmetric.

The LIFT method \cite{turner_automatic_1997, turner_automatic_1997-1}, like
AVATAR, calculated the angular flux for a region by assuming the angular flux
was a product of the scalar flux and an exponential function. The angular flux
values were then used to generate values for the exponential transform variance
reduction
technique to bias the particles in space, energy, and angle. Like AVATAR, LIFT
also generated weight window parameters. However, generating a full
angle-dependent weight window map and running Monte Carlo transport with those
weight windows was computationally limiting, and the authors chose to only
generate space- and energy- dependent weight windows. Turner showed that LIFT
outperformed AVATAR for several example problems, but both methods performed
poorly in voids and low-density regions.

In an attempt to adapt CADIS and FW-CADIS to include angular information into
the variance reduction parameters,
Peplow et. al. formulated an adjustment to CADIS in the ORNL
code suite \cite{peplow_consistent_2012}. Two different
methods to generate weight windows and source biasing parameters
were investigated:
CADIS with directional source biasing, and CADIS without directional source
biasing. These methods were referred to as Simple Angular CADIS. Like AVATAR and
LIFT, Simple Angular CADIS approximated the angular flux as a product of the
scalar flux and an exponential. Like AVATAR, the angular flux values
were used to
generate angle-dependent weight windows but also consistently generated source
biasing parameters. For the method without
directional source biasing, the biased source distribution matched that of the
original CADIS, but the weight window values were directionally-dependent. The
method with directional source biasing used the transform function to obtain
directionally-dependent weight windows and directional source biasing.
Peplow and his colleagues found
that these methods generally increased the FOM by a factor of 1-5 as compared to
traditional CADIS, but in some
cases decreased the FOM. This was attributed to the P$_1$ type assumption used
to calculate the angular flux, which limited the physical applicability of the
method, as it did with AVATAR.

LIFT, AVATAR, and Simple Angular CADIS all showed that by including angular
information into Monte Carlo variance reduction parameters the FOM can be
improved. However, none of these methods used the actual angular flux to
calculate the variance reduction parameters for the problem they were
optimizing. Without explicitly using the angular flux solutions they were
limited in which types of problems they were applicable, because some assumption
of the degree of anisotropy of the flux was made. Further, LIFT and
Simple Angular CADIS showed that by including substantial angular biasing in the
weight windows in problems where the approximation to the angular flux is not
sufficient, the FOM can decrease not unsubstantially, defeating the purpose of using
these methods.

\subsection{The $\Omega$ Methods}
\label{sec:omegaintro}

The foundation of the $\Omega$-methods is built upon CADIS and FW-CADIS. As with
both methods,
the $\Omega$-methods will use a version of the adjoint scalar flux to
consistently bias a Monte Carlo problem with the intention of reducing the
variance. In section \ref{sec:Importance} the concept of importance was
introduced. Notably, it was shown that the adjoint flux is a good marker 
for the likelihood of particles to contribute to a tally, which is the
particle's importance. It was also shown that the
product of the forward and adjoint fluxes generates a pseudo-particle flux called
the contributon flux, where contributons are ``importance particles''.
These importance particles can be used to show preferential flow paths from a
source to a tally or desired location.
% Yes, this is the kind of material for this section. Good description!

By using a version of the adjoint scalar flux that has been formulated with the
contributon flux, the direction of particle flow will be incorporated into the
importance map and, consequently, the
variance reduction parameters. By using this variant of the adjoint scalar flux,
the method will maintain the importance relationship between particles and
their proximity to the adjoint source location, or the forward response
location. 
% I'm a little confused by the above sentence...
The adjusted adjoint scalar flux quantity, or the $\Omega$-adjoint
scalar flux, is:
%
\begin{equation}
  \phi^{\dagger}_{\Omega}(\vec {r} ,E)  = \frac{\int_{\Omega}{\psi^{\dagger}
                             (\vec{r}, E, \hat\Omega)
                             \psi(\vec{r}, E, \hat\Omega)} d\Omega}
                             {\int_{\Omega}\psi(\vec{r}, E, \hat\Omega)
                           d\Omega}.
\label{eq:omega_basic}
\end{equation}
% should the d \Omega have a hat? I'm not sure...

The $\Omega$-flux is a hybridization of the adjoint scalar flux and the contributon
flux. It is both a normalized contributon flux and a forward-weighted
adjoint flux. Theoretically, it should inherit some of the advantages of
each of the traditional
adjoint and the contributon fluxes. Because it maintains dimensionality of the
traditional adjoint scalar flux, it can be used in place of the standard adjoint
scalar flux in both
CADIS and FW-CADIS variance reduction parameter generation. 
This means that nothing about the CADIS and FW-CADIS implementations need to 
change, making adoption of the new method straightforward.

\subsubsection{CADIS-$\Omega$}
\label{sec:cadomega}

As with CADIS, CADIS-$\Omega$ will consistently bias the problem's source and
particle weights according to their importance. However, CADIS-$\Omega$ will
use the
$\Omega$-adjoint scalar flux rather than the standard adjoint scalar flux
to generate the biased source distribution, weight windows, and particle birth weights.
The consistency between the biased source and particle weights is maintained
when using $\phi_{\Omega}^{\dagger}$ rather than $\phi^{\dagger}$. 
% I don't understand the above sentence. 
The adjusted
formulation of CADIS using the $\Omega$ fluxes is given by Eqs.
\eqref{eq:CADISomegamethod}.
% note: always use eqref--will format equation references more clearly.
%
\begin{subequations}
\label{eq:CADISomegamethod}
%
The biased source distribution used by CADIS-$\Omega$ is formulated just as
it is in CADIS, except the new adjoint fluxes are used:
% I would refer to all of our quantities with the sub omega to make it clear
\begin{equation}
\begin{split}
  \hat{q}_{\Omega}  & = \frac{\phi_{\Omega}^{\dagger}(\vec {r} ,E)q(\vec {r} ,E)}
               {\iint\phi_{\Omega}^{\dagger}(\vec {r} ,E)
               q(\vec {r} ,E) dE d\vec{r}} \\
           & = \frac{\phi_{\Omega}^{\dagger}(\vec {r} ,E)q(\vec {r} ,E)}{R_{\Omega}}.
\end{split}
\end{equation}
% I think this is correct about R? You probably want to call out that R is now
% made with the omega flux
The  starting weights of the particles sampled from the new
biased source distribution, $\hat{q}_{\Omega}$ are given by
\begin{equation}
\begin{split}
w_{0,\Omega}  & = \frac{q}{\hat{q}_{\Omega}} \\
     & = \frac{R_{\Omega}}{\phi_{\Omega}^{\dagger}(\vec {r} ,E)} ,
\end{split}
\end{equation}
%
and the new target weights for the particle are:
\begin{equation}
  \hat{w}_{\Omega} = \frac{R_{\Omega}}{\phi_{\Omega}^{\dagger}(\vec {r} ,E)} .
\end{equation}
\label{eq:cadomega_eqns}
\end{subequations}

\subsubsection{FW-CADIS-$\Omega$}
\label{sec:fwcadomega}

FW-CADIS differs from CADIS in that it requires a forward deterministic
calculation to generate $q^{\dagger}$, which is used as the source distribution
in the adjoint deterministic problem (recall that CADIS sets
$q^{\dagger}=\sigma_d$). Depending on the type of global
response desired, FW-CADIS runs a deterministic forward calculation to
approximate the global response in the problem. The inverse of these responses
is then used to generate the biased adjoint source distribution for the adjoint
deterministic run. Therefore, the behavior of FW-CADIS-$\Omega$
in the forward biasing
portion of the calculation is unchanged from FW-CADIS.

\begin{subequations}
The generalized form for the adjoint source definition is given by the fraction
of the response in a region of phase space, P, over the total response in the
problem, or
\begin{equation*}
{ q^{\dagger}} (P)=\frac{\sigma_d(P)}{R} .
\end{equation*}
%
The adjoint source for the spatially dependent global dose, $\int
\phi(\vec{r},E)\sigma_d(\vec{r},E) dE$ is:
\begin{equation*}
{ q^{\dagger} }(\vec { r } ,E)=\frac { \sigma _{ d }(\vec { r } ,E) }{ \int {
  \sigma _{ d }(\vec { r } ,E)\psi (\vec { r } ,E,) } dE } .
\end{equation*}
%
The adjoint source for the spatially dependent total flux $\int \phi(\vec{r},E)
dE $ is:
\begin{equation*}
{ q^{\dagger} }(\vec { r })=\frac { 1 }{ \int { \phi (\vec { r } ,E) } dE } .
\end{equation*}
%
The adjoint source for the energy- and spatially- dependent flux
$\phi(\vec{r},E)$ is
\begin{equation*}
{ q^{\dagger} }(\vec { r } ,E)=\frac { 1 }{\phi (\vec { r } ,E) }\:.
\end{equation*}
\label{eq:fwcadomega_eqns}
\end{subequations}

One advantage of FW-CADIS-$\Omega$ is that, from a transport perspective, the
$\Omega$-method is no more expensive than standard FW-CADIS. Because both
versions require a forward and adjoint deterministic calculation, an extra
transport step is not required as it is for CADIS-$\Omega$. This is attractive,
but the nature of FW-CADIS might not be the most well-suited for the
$\Omega$-methods. Because FW-CADIS attempts to evenly distribute particles
throughout the problem using the forward-biased adjoint fluxes,
the additional forward normalization with the $\Omega$-methods will likely skew
the particle distribution in the problem in the forward direction,
and it may place too great of
importance on the forward-moving particles in generating the variance reduction
parameters.
% May need to update this paragraph after we see what happens in practice.
