\section{Theory: Angle-Informed Importance Maps for CADIS and FW-CADIS}
\label{sec:methodtheory}

There exist methods to generate
variance reduction parameters
for deep penetration radiation transport problems with strong anisotropy in the
flux. These methods have shown to have varying success, and may not be fully
automated. The solution proposed in this dissertation is a formulation that we
have named the $\Omega$-CADIS-methods. This section will commence with a brief
discussion of the foundational research
on which the $\Omega$-CADIS-methods are built.
That discussion serves as a primer for the subsequent section, which
is an introduction to the $\Omega$-CADIS-methods and a discussion on how they differ
from their predecessors.

\subsection{Previous Work}
\label{sec:omegabknd}

As discussed in Sections \ref{sec:CADIS}
through \ref{sec:AngleVR}, the existing gold standard for automatically
generating variance reduction parameters for deep penetration
fixed-source radiation transport problems are
CADIS and FW-CADIS. Both of these methods are very effective at
generating variance reduction parameters for local and global solutions,
respectively. However, CADIS and FW-CADIS have only been implemented
to perform variance reduction in
space and energy, not angle. As a result, solutions for problems with
strong anisotropy in the flux are not always optimized with these methods,
resulting in slow convergence times and low FOM values.
Problems with strong anisotropies in the flux require more than just
space- and energy- variance reduction techniques.
A number of angle-informed variance reduction methods have been
investigated, most notably AVATAR, LIFT, and a modified version of
CADIS using AVATAR-type angular parameters.

LIFT, AVATAR, and Simple Angular CADIS all showed that by including angular
information into Monte Carlo variance reduction parameters the FOM can be
improved. However, none of these methods used the actual angular flux to
calculate the variance reduction parameters for the problem they were
optimizing. Without explicitly using the angular flux solutions they were
limited in which types of problems they were applicable, because some assumption
of the degree of anisotropy of the flux was made. Further, LIFT and
Simple Angular CADIS showed that by including substantial angular biasing in the
weight windows in problems where the approximation to the angular flux is not
sufficient, the FOM can decrease not unsubstantially, defeating the purpose of using
these methods.

\subsection{The $\Omega$ Methods}
\label{sec:omegaintro}

The foundation of the $\Omega$-methods is built upon CADIS and FW-CADIS. As with
both methods,
the $\Omega$-methods will use a version of the adjoint scalar flux to
consistently bias a Monte Carlo problem with the intention of reducing the
variance. In Section \ref{sec:Importance} the concept of importance was
introduced. Notably, it was shown that the adjoint flux is a good marker
for the likelihood of particles to contribute to a tally, which is the
particle's importance. It was also shown that the
product of the forward and adjoint fluxes generates a pseudo-particle flux called
the contributon flux, where contributons are ``importance particles''.
These importance particles can be used to show preferential flow paths from a
source to a tally or desired location.

By using a version of the adjoint scalar flux that has been formulated with the
contributon flux, the direction of particle flow will be incorporated into the
importance map and, consequently, the
variance reduction parameters. By using this variant of the adjoint scalar flux,
the method, like traditional CADIS, will show increasing importance as the
particles travel near the adjoint source. However, because this variant of the adjoint
flux incorporates directionality of the particle flow, not all regions near the
adjoint source are equally important. In this way, the adjusted flux
incorporates features from both the adjoint- and contributon- fluxes.

The adjusted adjoint scalar flux quantity, or the $\Omega$-adjoint
scalar flux, is
%
\begin{equation}
  \phi^{\dagger}_{\Omega}(\vec {r} ,E)  = \frac{\int_{\Omega}{\psi^{\dagger}
                             (\vec{r}, E, \hat\Omega)
                             \psi(\vec{r}, E, \hat\Omega)} d\hat{\Omega}}
                             {\int_{\Omega}\psi(\vec{r}, E, \hat\Omega)
                             d\hat{\Omega}}.
\label{eq:omega_basic}
\end{equation}
%
The $\Omega$-flux is a hybridization of the adjoint scalar flux and the contributon
flux. It is both a normalized contributon flux and a forward-weighted
adjoint flux. As a result, it should inherit some of the advantages of
each of the traditional
adjoint and the contributon fluxes. Because it maintains dimensionality of the
traditional adjoint scalar flux, it can be used in place of the standard adjoint
scalar flux in both
CADIS and FW-CADIS variance reduction parameter generation.
This means that the method can capitalize on existing infrastructure used to
generate variance reduction parameters for CADIS and FW-CADIS, and only the
software handling the transport and flux-generation requires modification.

\subsubsection{CADIS-$\Omega$}
\label{sec:cadomega}

As with CADIS, CADIS-$\Omega$ consistently biases a problem's source and
particle weights according to their importance. However, CADIS-$\Omega$
uses the
$\Omega$-adjoint scalar flux rather than the standard adjoint scalar flux
to generate the biased source distribution, weight windows,
and the particle birth weights. Furthermore, because $\phi_{\Omega}^{\dagger}$
is used to calculate these values in CADIS-$\Omega$,
the consistent-biasing hallmark for
which CADIS is known is maintained.
The adjusted
formulation of CADIS using the $\Omega$ fluxes is given by Eqs.
\eqref{eq:CADISomegamethod}.
%
\begin{subequations}
\label{eq:CADISomegamethod}
%
The biased source distribution used by CADIS-$\Omega$ is formulated just as it
is in CADIS, except the adjusted adjoint fluxes are used:
\begin{equation}
\begin{split}
  \hat{q}_{\Omega}  & = \frac{\phi_{\Omega}^{\dagger}(\vec {r} ,E)q(\vec {r} ,E)}
               {\iint\phi_{\Omega}^{\dagger}(\vec {r} ,E)
               q(\vec {r} ,E) dE d\vec{r}} \\
               & = \frac{\phi_{\Omega}^{\dagger}(\vec {r} ,E)q(\vec {r},E)}
               {R_{\Omega}}.
\end{split}
\end{equation}
%
The  starting weights of the particles sampled from the
biased source distribution, $\hat{q}$ are given by
\begin{equation}
\begin{split}
  w_{0, \Omega}  & = \frac{q}{\hat{q}_{\Omega}} \\
  & = \frac{R_{\Omega}}{\phi_{\Omega}^{\dagger}(\vec {r} ,E)} ,
\end{split}
\end{equation}
%
and the new target weights for the particle are
\begin{equation}
  \hat{w}_{\Omega} = \frac{R_{\Omega}}{\phi_{\Omega}^{\dagger}(\vec{r} ,E)} .
\end{equation}
\label{eq:cadomega_eqns}
\end{subequations}

\subsubsection{FW-CADIS-$\Omega$}
\label{sec:fwcadomega}

FW-CADIS differs from CADIS in that it requires a forward deterministic
calculation to generate $q^{\dagger}$, which is used as the source distribution
in the adjoint deterministic problem (recall that CADIS sets
$q^{\dagger}=\sigma_d$). Depending on the type of global
response desired, FW-CADIS runs a deterministic forward calculation to
approximate the global response in the problem. The inverse of these responses
is then used to generate the biased adjoint source distribution for the adjoint
deterministic run. Therefore, the behavior of FW-CADIS-$\Omega$
in the forward biasing
portion of the calculation will remain unchanged from FW-CADIS.
%
\begin{subequations}
The generalized form for the adjoint source definition is given by the fraction
of the response in a region of phase space, $P$, over the total response in the
problem, or
\begin{equation*}
  { q^{\dagger}_{\Omega}} (P) = { q^{\dagger}} (P)=\frac{\sigma_d(P)}{R} .
\end{equation*}
%
When applied to the spatially-dependent global dose, $\int
\phi(\vec{r},E)\sigma_d(\vec{r},E) dE$, the adjoint source will be
\begin{equation*}
  { q^{\dagger}_{\Omega} }(\vec { r } ,E) = { q^{\dagger} }(\vec { r } ,E)
  =\frac { \sigma _{ d }(\vec { r } ,E) }{ \int {
  \sigma _{ d }(\vec { r } ,E)\psi (\vec { r } ,E,) } dE } .
\end{equation*}
%
The adjoint source for the spatially-dependent total flux $\int \phi(\vec{r},E)
dE $ is
\begin{equation*}
  { q^{\dagger}_{\Omega} }(\vec { r }) = { q^{\dagger} }(\vec { r })
  =\frac { 1 }{ \int { \phi (\vec { r } ,E) } dE } .
\end{equation*}
%
The adjoint source for the energy- and spatially-dependent flux
$\phi(\vec{r},E)$ is
\begin{equation*}
  { q^{\dagger}_{\Omega} }(\vec { r } ,E) = { q^{\dagger} }(\vec { r } ,E)=\frac { 1 }{\phi (\vec { r } ,E) } .
\end{equation*}
\label{eq:fwcadomega_eqns}
\end{subequations}

One advantage of FW-CADIS-$\Omega$ is that, from a transport perspective, the
$\Omega$-method is no more expensive than standard FW-CADIS. Because both
versions require a forward and adjoint deterministic calculation, an extra
transport step is not required as it is for CADIS-$\Omega$. This is attractive,
but the nature of FW-CADIS might not be the most well-suited for the
$\Omega$-methods. Because FW-CADIS attempts to evenly distribute particles
throughout the problem using the forward-biased adjoint fluxes,
the additional forward normalization with the $\Omega$-methods will likely skew
the particle distribution in the problem in the forward direction,
and it may place too great of
importance on the forward-moving particles in generating the variance reduction
parameters.
