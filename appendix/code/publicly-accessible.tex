\section{Inputs and Scripts}
\label{sec:github_codes}

\subsection{Parametric Study Problems}

The characterization problem MCNP and ADVANTG inputs are all available at the
public repository: https://github.com/munkm/caskmodels

The problems were executed from commit hash no.:
\begin{verbatim}
0bc315bf2f49f83627a563f710bf4f586ec3e489
\end{verbatim}
Directions on how to run these problems are available in the repository as well.

\subsection{Postprocessing Scripts}
\label{sec:postprocess}

A suite of postprocessing tools were created to create the figures herein. The
tools, the datasets, and directions on how to recreate the figures in this
dissertation are available at: https://github.com/munkm/thesiscode

The figures in this dissertation were created with the suite of tools at commit
hash no.: \begin{verbatim} 28329c86b939d30d2ac236bfccaf026d7e57556d
\end{verbatim}

This repository has three folders:
\begin{itemize}
  \item notebooks
  \item scripts
  \item submission\_scripts
  \item data
\end{itemize}

\textbf{Submission scripts} is the folder containing the .pbs runscripts used to run
these problems on the remote ORNL machine remus. If one has access to a similar
machine with the same queueing system, these will run them on the same number of
cores as run in the studies of this dissertation. Alternatively, there are
directions on how to run each problem on a local machine in that directory.

\textbf{Scripts} contains the scripting tools used to parse the datasets generated by
ADVANTG and Denovo. This suite has plotting tools, MCNP postprocessing tools,
statistical analysis tools to analyze anisotropy metrics, and HDF5
postprocessing tools. The plotting tools can be used to generate the tally
result and relative error histograms, the categorical violin, strip, or boxplots
for each metric, and the comparitive histograms from the angle-informed study.
The scripts folder also has a tool for automatically setting up the parametric
study, as outlined in Section \ref{sec:AngleResults}. The tool includes an
automatically generated submission script, for ease of use for future parametric
studies.
The flux maps for each problem were generated with VisIT, which is not an
automated process. Future work will be to make these reproducible as well.

\textbf{Notebooks} contains a few Jupyter notebooks that provide examples on using the
plotting tools used in this suite. An interested user can understand how the
postprocessing tools work and categorize data to compare the adjoint, CADIS, and
CADIS-$\Omega$ methods.

The \textbf{data} folder is mostly empty, but contains a makefile that a user can execute
to automatically download the datasets from this project. This folder also has
instructions on how to use the plotting tools to analyze the data just
downloaded. One should be aware that the datasets from these few problems are
still significant, at roughly 800Gb.

These tools are by no means comprehensive and cover all types of comparative
hybrid methods figures. For example, they will likely not be the best tools with
which to analyze the Forward-Weighted methods. The
top level repository contains a README file with the future features added
to the repository.

\subsection{Supporting Repositories}
\label{subsec:supportinrepos}

The violin and scatterplots used throughout this thesis were made with a
modified version of the seaborn frontend to matplotlib. That modification
includes adjusting categorials to be over logrithmic scales. The URL for the
modified repository is: \\
https://github.com/munkm/seaborn

