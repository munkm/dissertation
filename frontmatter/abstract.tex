\begin{abstract}

The development of methods for deep-penetration radiation transport is of continued
importance for radiation shielding, nonproliferation, nuclear threat reduction,
and medical applications. As these applications become more ubiquitous, the need
for transport methods that can accurately and reliably model the systems' behavior
will persist. For these types of systems, hybrid methods are often the best
choice to obtain a reliable answer in a short amount of time.
Hybrid methods leverage
the speed and uniform uncertainty distribution of a deterministic solution to
bias Monte Carlo transport to reduce the variance in the solution.
At present, the Consistent Adjoint-Driven Importance Sampling
(CADIS) and Forward-Weighted CADIS (FW-CADIS) hybrid
methods are the gold standard by
which to model systems that have deeply-penetrating radiation. They use an
adjoint scalar flux to generate variance reduction parameters for Monte Carlo.
However, in
problems where there exists strong anisotropy in the flux, CADIS and FW-CADIS
are not as effective at reducing the problem variance as isotropic
problems.

This dissertation covers the theoretical background, implementation of, and
characterization of a set of angle-informed hybrid methods
that can be applied to
strongly anisotropic deep-penetration radiation transport problems. These methods
use a forward-weighted adjoint angular flux to generate variance
reduction parameters for Monte Carlo. As a result, they leverage both adjoint
and contributon theory for variance reduction. They have been named
CADIS-$\Omega$ and FW-CADIS-$\Omega$.

To characterize CADIS-$\Omega$, several characterization problems with flux
anisotropies were devised. These problems contain different physical
mechanisms by which flux anisotropy is induced. Additionally, a series of novel
anisotropy metrics by which to quantify flux anisotropy are used
to characterize the
methods beyond standard Figure of Merit (FOM) and relative error metrics. As a
result, a more thorough investigation into the effects of anisotropy and the
degree of anisotropy on Monte Carlo convergence is possible.

The results from the characterization of CADIS-$\Omega$ show that it performs
best in strongly anisotropic problems that have preferential particle flowpaths,
but only if the flowpaths are not comprised of air.
Further, the characterization of
the method's sensitivity to deterministic angular discretization showed that
CADIS-$\Omega$ has less sensitivity to discretization than CADIS for both
quadrature order and P$_{N}$ order. However, more variation in the results were
observed in response to changing quadrature order than P$_N$ order. Further, as
a result of the forward-normalization in the $\Omega$-methods, ray effect
mitigation was observed in many of the characterization problems.

The characterization of the CADIS-$\Omega$-method in this dissertation serves to
outline a path forward for further hybrid methods development. In particular,
the response that the $\Omega$-method has with changes in quadrature order,
P$_{N}$ order, and on ray effect mitigation are strong indicators that the
method is more resilient than its predecessors to strong anisotropies in the
flux.
With further method characterization, the full potential of the $\Omega$-methods
can be realized.
The method can then be applied to geometrically complex, materially
diverse problems and help to advance system modelling in deep-penetration
radiation transport problems with strong anisotropies in the flux.

\end{abstract}
